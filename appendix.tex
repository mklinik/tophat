%% For double-blind review submission, w/o CCS and ACM Reference (max submission space)
%\documentclass[sigconf,review,anonymous]{acmart}\settopmatter{printfolios=true,printccs=false,printacmref=false}
%% For double-blind review submission, w/ CCS and ACM Reference
% \documentclass[sigconf,review,anonymous]{acmart}\settopmatter{printfolios=true}
%% For single-blind review submission, w/o CCS and ACM Reference (max submission space)
% \documentclass[sigconf,review]{acmart}\settopmatter{printfolios=true,printccs=false,printacmref=false}
%% For single-blind review submission, w/ CCS and ACM Reference
% \documentclass[sigconf,review]{acmart}\settopmatter{printfolios=true}
%% For final camera-ready submission, w/ required CCS and ACM Reference
\documentclass[sigconf]{acmart}\settopmatter{}


% !TEX root=../main.tex


%% Basics %%%%%%%%%%%%%%%%%%%%%%%%%%%%%%%%%%%%%%%%%%%%%%%%%%%%%%%%%%%%%%%%%%%%%%

%% Fixes %%

\usepackage{underscore}


%% Fonts %%

\usepackage[utf8]{inputenc}
\usepackage[T1]{fontenc}

\usepackage{stmaryrd}

% \usepackage{tgpagella}
% \usepackage{lucidabr}
\usepackage{libertine}
\usepackage[varqu]{zi4}
\usepackage[libertine]{newtxmath}


%% Programming %%

\usepackage{ifthen}


%% Layout %%

% \usepackage[final]{microtype}


%% Additions %%%%%%%%%%%%%%%%%%%%%%%%%%%%%%%%%%%%%%%%%%%%%%%%%%%%%%%%%%%%%%%%%%%

%% Textual %%

\usepackage{paralist}
\usepackage{quoting}


%% Maths %%

\usepackage{amsmath}
\usepackage{mathpartir}


%% Graphics %%

\usepackage{graphicx}
% \usepackage{xcolor}


%% Tabulations %%

\usepackage{booktabs}
\usepackage{array}


%% Listings %%

% \usepackage[final]{listings}


%% References %%

\usepackage{cleveref}

% !TEX root=../main.tex


%% Fixes %%

\frenchspacing

%% NOTE: uses the same lengths as in `tufte-common.def` and `article.cls`...
\setlength{\bigskipamount}   {3.25ex plus .2ex} %% ...before (sub)section
\setlength{\medskipamount}   {2.3ex  plus .2ex} %% ...after section
\setlength{\smallskipamount} {1.5ex  plus .2ex} %% ...after subsection


%% Numbering %%

% \setcounter{secnumdepth}{2}


%% Compact lists %%
%% NOTE: requires `paralist`

\setlength{\pltopsep}{\smallskipamount}
\setlength{\plpartopsep}{\parskip}
\setlength{\plitemsep}{\parskip}
\setlength{\plparsep}{\parskip}

% !TEX root=../main.tex


%% Helpers %%%%%%%%%%%%%%%%%%%%%%%%%%%%%%%%%%%%%%%%%%%%%%%%%%%%%%%%%%%%%%%%%%%%%

\let\newoperator\DeclareMathOperator


%% Text %%%%%%%%%%%%%%%%%%%%%%%%%%%%%%%%%%%%%%%%%%%%%%%%%%%%%%%%%%%%%%%%%%%%%%%%

\newcommand*{\alert}[1]
  {\textbf{#1}}
\newcommand*{\enquote}[1]
  {``#1''}
\newcommand*{\todo}[1]
  {\ensuremath{\star}\marginnote{\ensuremath{\star}#1}}


%% Lists %%
%% NOTE: requires `paralist`

%% Use compact lists by default
\renewenvironment{itemize}
  {\begin{compactitem}}
  {\end{compactitem}}
\renewenvironment{enumerate}
  {\begin{compactenum}}
  {\end{compactenum}}
\renewenvironment{description}
  {\begin{compactdesc}}
  {\end{compactdesc}}
%% Define starred versions as in-paragraph-lists
\newenvironment{itemize*}
  {\begin{inparitem}}
  {\end{inparitem}}
\newenvironment{enumerate*}
  {\begin{inparenum}}
  {\end{inparenum}}
\newenvironment{description*}
  {\begin{inpardesc}}
  {\end{inpardesc}}


%% Blocks %%

\newenvironment{block}
  {\smallskip}
  {\smallskip}


%% Column types %%
%% NOTE: requires `array`

\newcolumntype{L}{>{$}l<{$}}
\newcolumntype{C}{>{$}c<{$}}
\newcolumntype{R}{>{$}r<{$}}
\newcolumntype{T}{>{\ttfamily}l}
\newcolumntype{S}{>{\sffamily}l}


%% References %%
%% NOTE: requires `cleveref`

\let\see\cref
\let\See\Cref
\let\at\cpageref


%% Citations %%
%% NOTE: requires `natbib`

\let\cite\citep
\let\textcite\citet


%% Math %%%%%%%%%%%%%%%%%%%%%%%%%%%%%%%%%%%%%%%%%%%%%%%%%%%%%%%%%%%%%%%%%%%%%%%%

%% NOTE: change this to \emptyset when using a font that includes a nice standard emptyset
\let\nothing\varnothing


%% Braces %%

\let\<\langle
\let\>\rangle

\newcommand*{\llbrace}
  {\{\!|}
\newcommand*{\rrbrace}
  {|\!\}}


%% Operators %%

\let\lt<
\let\gt>
\let\eq\equiv

\newcommand*{\pp}
  {+\!\!+}


%% Shortcuts %%

\newcommand*{\powerset}
  {\mathcal{P}}

\newcommand*{\NN}{\mathbb{N}}
\newcommand*{\ZZ}{\mathbb{Z}}
\newcommand*{\RR}{\mathbb{R}}
\newcommand*{\CC}{\mathbb{C}}

\newcommand*{\LL}{\mathbb{L}}
\newcommand*{\UU}{\mathbb{U}}
\newcommand*{\BB}{\mathbb{B}}
\renewcommand*{\SS}{\mathbb{S}}

\newoperator{\downto}
  {\;\rightarrow\!\shortmid\;}

\let\to\rightarrow
\let\implies\Rightarrow
\let\infers\vdash

% !TEX root=../main.tex


%% Styles %%%%%%%%%%%%%%%%%%%%%%%%%%%%%%%%%%%%%%%%%%%%%%%%%%%%%%%%%%%%%%%%%%%%%%

\lstdefinestyle{natural}
  {columns=fullflexible
  ,gobble=2
  ,breaklines=true
  ,breakatwhitespace=true
  ,literate=
    %{.}{{$\cdot$}}1
    %{.}{{\ }}1
    {<<}{{$\<$}}1
    {>>}{{$\>$}}1
    {->}{{$\to$\ }}2
    % {--}{{--}}1
    %{_}{{\ }}1
    %{\ "}{{\ \textquotedblleft}}2
    %{"\ }{{\textquotedblright\ }}2
  ,basicstyle={\sffamily}
  ,keywordstyle=[1]{\bfseries}
  ,keywordstyle=[2]{\scshape}
  ,keywordstyle=[3]{}
  %,commentstyle={\itshape}
  %,identifierstyle={\itshape}
  ,emphstyle={\itshape}
  %,stringstyle={\rmfamily}
  ,showstringspaces=false
  ,texcl=true
  ,mathescape=true
  %,escapechar=\$
  %,escapeinside={\{\}}
  ,xleftmargin=1\parindent
  }

\lstdefinestyle{flexible}
  {columns=flexible
  ,gobble=2
  ,fontadjust=true
  ,basicstyle={\ttfamily\small}
  ,commentstyle={\itshape}
  ,keywordstyle={\bfseries}
  %,identifierstyle={\itshape}
  %,stringstyle={\ttfamily}
  ,emphstyle={\itshape}
  ,showstringspaces=false
  ,texcl=true
  ,mathescape=true
  %,escapechar=\$
  %,escapeinside={\{\}}
  ,xleftmargin=1\parindent
  }

\lstdefinestyle{literate}
  {literate=
    {\\}{{$\lambda$}}1
    {\\\$}{{\$}}1 %NOTE: otherwise eaten by `\`, NOTE: prevents \$ to be parsed as math escape
    {\\/}{{$\vee$}}1
    {/\\}{{$\wedge$}}1
    {A.}{{$\forall$}}1
    {E.}{{$\exist$}}1
    {->}{{$\rightarrow$ }}1
    {<-}{{$\leftarrow$}}1
    {<=}{{$\leq$}}1
    {>=}{{$\geq$}}1
    {>>=}{{>>=}}3 %NOTE: otherwise eaten by `>=`
    {\{|}{{$\{\!|\!$}}1
    {|\}}{{$\!|\!\}$}}1
    {\{|*|\}}{{$\{\!|\!\!\star\!\!|\!\}$}}3
  }


%% Definitions %%%%%%%%%%%%%%%%%%%%%%%%%%%%%%%%%%%%%%%%%%%%%%%%%%%%%%%%%%%%%%%%%

%% Tasks %%

\lstdefinelanguage{tasks}
  {sensitive=true
  ,morekeywords=[1]{if,then,else,case,of}
  ,morekeywords=[2]{Bool,Int,String,Store,List}
  ,morestring=[b]"
  ,morecomment=[l]--
  ,morecomment=[n]{\{-}{-\}}
  }[keywords,strings,comments]
\lstdefinestyle{tasks}
  {style=natural
  ,literate=
    {\\}{{$\lambda$}}1
    {>>=}{{$\Then$\ }}1
    {>>?}{{$\Next$\ }}1
    {<&>}{{$\And$\ }}1
    {<|>}{{$\Or$\ }}1
    {<?>}{{$\Xor$\ }}1
    {edit}{{$\Edit$}}1
    {enter}{{$\Enter$}}1
    {store}{{$\Change$}}1
    {fail}{{$\Fail$}}1
    {==}{{$\equiv$\ }}1
    {/=}{{$\nequiv$\ }}1
  }

\lstnewenvironment{TASK}[1][]
  {\lstset{language=tasks,style=tasks,#1}}
  {}
\newmacro{TS}[1][1]
  {\lstinline[language=tasks,style=tasks,#1]}
\newmacro{includeTASK}[2][]
  {\lstinputlisting[language=tasks,style=tasks,#1]{#2}}


%% Flows %%

\lstdefinelanguage{flows}
  {sensitive=true
  ,morekeywords=[1]{module,where,define,using,as,yielding,share,holding,with,do,for,fork,then,when,next,done,on,and,or,not,readonly,writeonly,readwrite}
  ,morekeywords=[2]{Bool,Int,String,Shared,List, Date,Document,Photo, Citizen,Company,Declaration}
  ,morekeywords=[3]{True,False,Just,Nothing,List}
  ,morestring=[b]"
  ,morecomment=[l]--
  ,morecomment=[n]{\{-}{-\}}
  }[keywords,strings,comments]

% \lstMakeShortInline[language=flows,style=natural] | % |
\lstnewenvironment{FLOW}[1][]
  {\lstset{language=flows,style=natural,#1}}
  {}
\newmacro{FL}[1][1]
  {\lstinline[language=flows,style=natural,#1]}
\newmacro{includeFLOW}[2][]
  {\lstinputlisting[language=flows,style=natural,#1]{#2}}


%% Clean %%

\lstdefinelanguage{clean}
  {sensitive=true
  %,alsoletter={ABCDEFGHIJKLMNOPQRSTUVWXYZabcdefghijklmnopqrstuvwxyz_`}
  %,alsoletter={~!@\#$\%^\&*-+=?<>:|\\} %$
  ,morekeywords={from,definition,implementation,import,module,system,code,inline,if,case,of,let,let!,in,where,with,class,instance,generic,derive,dynamic,infix,infixl,infixr}
  ,morestring=[b]"
  ,morestring=[b]'
  ,morecomment=[l]//
  ,morecomment=[n]{/*}{*/}
  }[keywords,strings,comments]

\lstnewenvironment{CLEAN}[1][]
  {\lstset{language=clean,style=flexible,#1}}
  {}
\newmacro{CL}[1][1]
  {\lstinline[language=clean,style=flexible,#1]}
\newmacro{includeCLEAN}[2][1]
  {\lstinputlisting[language=clean,style=flexible,#1]{#2}}


\newlogo[ITASKS]{iTasks}
\newlogo{BPMN}
\newlogo{BPEL}
\newlogo{UML}
\newlogo{WFN}

\newlogo{EDSL}
\newlogo{GUI}

\newlogo{STW}
\newlogo{NWO}

% !TEX root=../main.tex



%% Host language %%%%%%%%%%%%%%%%%%%%%%%%%%%%%%%%%%%%%%%%%%%%%%%%%%%%%%%%%%%%%%%


\newkeyword[IF]  {if}
\newkeyword[THEN]{then}
\newkeyword[ELSE]{else}

\newkeyword[Let]{let}
\newkeyword[In]{in}

\newkeyword[Ref] {ref}


\newmacro{If}[3]
  {\IF #1 \THEN #2 \ELSE #3}



%% Values %%


\newmathcommand{unit}{\<\>}


\newvalue{True}
\newvalue{False}
\newvalue[Not]{not}


\newmacro{str}[1]
  {\text{``#1''}}

\newoperator{Length}{\mathrm{length}}


\newvalue[Map]{map}
\newvalue[Fst]{fst}
\newvalue[Snd]{snd}
\newvalue[Assoc]{assoc}



%% Types %%


\newtype{Unit}
\newtype{Bool}
\newtype{Nat}
\newtype{Int}
\newtype{String}
\newtype[Reference]{Ref}
\newtype{Task}
\newtype{Maybe}

\newtype{Euro}



%% Object language %%%%%%%%%%%%%%%%%%%%%%%%%%%%%%%%%%%%%%%%%%%%%%%%%%%%%%%%%%%%%


\let\And\relax
\newoperator{Then}  {\blacktriangleright}
\newoperator{Next}  {\vartriangleright}
\newoperator{And}   {\Join}
\newoperator{Or}    {\blacklozenge}
\newoperator{Xor}   {\lozenge}
\newoperator{Edit}  {\square}
\newoperator{View}  {\overline{\square}}
\newoperator{Enter} {\boxtimes}
\newoperator{Update}{\blacksquare}
\newoperator{Watch} {\overline{\blacksquare}}
\newoperator{Fail}  {\lightning}

\newoperator{AndOr} {\DEPRECATED}



%% Events %%


\newvalue[Left]   {L}
\newvalue[Right]  {R}


\newvalue[Clear]   {C}
\newvalue[Continue]{N}
\newvalue[Pick]    {P}


\newvalue[First]  {F}
\newvalue[Second] {S}
\newvalue[Here]   {H}



%% Semantic functions %%%%%%%%%%%%%%%%%%%%%%%%%%%%%%%%%%%%%%%%%%%%%%%%%%%%%%%%%%


\newmathcommand{evaluate}[rel]
  {\;\downarrow\;}
\newmathcommand{normalise}[rel]
  % {\;\rightarrow\!\shortmid\;}
  {\;\Downarrow\;}
\newmacro{handle}[1]
  {\mathrel{\;\xrightarrow{\;#1\;}\;}}
\newmacro{drive}[1]
  {\mathrel{\;\xRightarrow{\;#1\;}\;}}


\newmathcommand{Value}[cal]
  {V}
\newmathcommand{Firsts}[cal]
  {F}
\newmathcommand{Interface}[cal]
  {U}
\newmathcommand{Succeeding}[cal]
  {S}



%% Depricated %%%%%%%%%%%%%%%%%%%%%%%%%%%%%%%%%%%%%%%%%%%%%%%%%%%%%%%%%%%%%%%%%%

\newvalue[Execute] {<depricated>}

% !TEX root=pldi2019.tex


%% Helpers %%%%%%%%%%%%%%%%%%%%%%%%%%%%%%%%%%%%%%%%%%%%%%%%%%%%%%%%%%%%%%%%%%%%%

\newmacro{newrule}[4][2]
  {\newmacro{#1}{\inferrule*[lab={#1},right={$#2$}]
    {#3}
    {#4}}}
\newmacro{userule}
  {\usemacro}
\newmacro{refrule}[1]
  {\ifthenelse{\isundefined{#1}}
    {\GenericError{}{Rule `#1` is not defined}{}{}}
    {\textsc{#1}}}


\newif\ifstateful
\statefulfalse
\newmacro{st}[2][1]
  {\ifthenelse{\isempty{#1}}
    {\ifstateful{,\:s#2}\else{}\fi}
    {\ifstateful{,\:[#1]s#2}\else{}\fi}}
\newmacro{St}[1]
  {\ifstateful{,\:\Sigma#1}\else{}\fi}



%% Typing %%%%%%%%%%%%%%%%%%%%%%%%%%%%%%%%%%%%%%%%%%%%%%%%%%%%%%%%%%%%%%%%%%%%%%


\newmacro{RelationT}
  {\Gamma,\Sigma \infers e : \tau}


\newrule{T-Var}
  {x:\tau\in\Gamma}
  {\Gamma,\Sigma\infers x:\tau}


\newrule{T-Abs}
  {\Gamma[x:\tau_1] ,\Sigma \infers e:\tau_2}
  {\Gamma,\Sigma \infers \lambda x : \tau_1 . e :\tau_1 \to \tau_2}

\newrule{T-App}
  {\Gamma,\Sigma \infers e_1:\tau_1\to\tau_2\\
   \Gamma,\Sigma \infers e_2:\tau_1}
  {\Gamma,\Sigma \infers e_1 e_2 :\tau_2}


\newrule{T-If}
  {\Gamma,\Sigma \infers e_1:\Bool\\
   \Gamma,\Sigma \infers e_2:\tau\\
   \Gamma,\Sigma \infers e_3:\tau}
  {\Gamma,\Sigma \infers \If{e_1}{e_2}{e_3}:\tau}


\newrule{T-Pair}
    {\Gamma,\Sigma \infers e_1 : \tau_1 \\
     \Gamma,\Sigma \infers e_2 : \tau_2}
    {\Gamma,\Sigma \infers \tuple{e_1, e_2} :\tau_1 \times \tau_2}


\newrule{T-Ref}
  {\Gamma,\Sigma \infers e:\tau}
  {\Gamma,\Sigma \infers \Ref e :\Reference \tau}

\newrule{T-Deref}
  {\Gamma,\Sigma \infers e:\Reference \tau}
  {\Gamma,\Sigma\infers\ !e:\tau}

\newrule{T-Assign}
  {\Gamma,\Sigma\infers e_1:\Reference \tau\\
   \Gamma,\Sigma\infers e_2:\tau}
  {\Gamma,\Sigma\infers e_1 := e_2:\Unit}

\newrule{T-Loc}
  {\Sigma(l) = \tau}
  {\Gamma,\Sigma\infers l:\Reference \tau}


\newrule{T-Edit}
  {\Gamma,\Sigma \infers e : \tau}
  {\Gamma,\Sigma \infers \Edit e : \Task \tau}

\newrule{T-Fill}
  {\ }
  {\Gamma,\Sigma \infers \Enter \tau : \Task \tau}

\newrule{T-Update}
  {\Gamma,\Sigma \infers e : \Reference \tau}
  {\Gamma,\Sigma \infers \Update e : \Task \tau}


\newrule{T-Fail}
  {\ }
  {\Gamma,\Sigma \infers \Fail : \Task \tau}


\newrule{T-Then}
  {\Gamma,\Sigma \infers e_1 : \Task \tau_1 \\
   \Gamma,\Sigma \infers e_2 : \tau_1 \to \Task \tau_2}
  {\Gamma,\Sigma \infers e_1 \Then e_2 : \Task \tau_2}


\newrule{T-Next}
  {\Gamma,\Sigma \infers e_1 : \Task \tau_1 \\
   \Gamma,\Sigma \infers e_2 : \tau_1 \to \Task \tau_2}
  {\Gamma,\Sigma \infers e_1 \Next e_2 : \Task \tau_2}


\newrule{T-And}
  {\Gamma,\Sigma \infers e_1 : \Task \tau_1 \\
   \Gamma,\Sigma \infers e_2 : \Task \tau_2}
  {\Gamma,\Sigma \infers e_1 \And e_2 : \Task\,(\tau_1 \times \tau_2)}


\newrule{T-Or}
  {\Gamma,\Sigma \infers e_1 : \Task \tau \\
   \Gamma,\Sigma \infers e_2 : \Task \tau }
  {\Gamma,\Sigma \infers e_1 \Or e_2 : \Task \tau}


\newrule{T-Xor}
  {\Gamma,\Sigma \infers e_1 : \Task \tau \\
   \Gamma,\Sigma \infers e_2 : \Task \tau }
  {\Gamma,\Sigma \infers e_1 \Xor e_2 : \Task \tau}


\newrule{T-Appoint}
  {\Gamma,\Sigma\infers e:\Task\tau}
  {\Gamma,\Sigma\infers u \At e:\Task\tau}



%% Evaluation %%%%%%%%%%%%%%%%%%%%%%%%%%%%%%%%%%%%%%%%%%%%%%%%%%%%%%%%%%%%%%%%%%


\newmacro{RelationE}
  {e,s \evaluate v,s'}


\newrule{E-Value}[v\in\set{\lambda x:\tau.e, \unit, l, B, I, S}]
  {\ }
  {v,s\evaluate v,s}

\newrule{E-App}
  {e_1,s\evaluate \lambda x:\tau.e_1',s'\\
   e_2,s'\evaluate v_2,s''\\
   e_1'[x\mapsto v_2],s''\evaluate v_1,s'''}
  {e_1 e_2,s \evaluate v_1,s'''}


\newrule{E-IfTrue}
    {e_1,s\evaluate \True,s'\\
     e_2,s'\evaluate v,s''}
    {\If{e_1}{e_2}{e_3},s\evaluate v,s''}

\newrule{E-IfFalse}
  {e_1,s\evaluate \False,s'\\
   e_3,s'\evaluate v,s''}
  {\If{e_1}{e_2}{e_3},s\evaluate v,s''}


\newrule{E-Pair}
  {e_1,s\evaluate v_1,s'\\
   e_2,s'\evaluate v_2,s''}
  {\tuple{e_1,e_2},s\evaluate\tuple{v_1,v_2},s''}


\newrule{E-Ref}
  {e,s\evaluate v,s'\\
   l\not\in Dom(s)}
  {\Ref e,s\evaluate l,s'[l\mapsto v]}

\newrule{E-Deref}
  {e,s\evaluate l,s'}
  {!e,s\evaluate s'(l),s'}

\newrule{E-Assign}
  {e_1,s\evaluate l,s'\\
   e_2,s'\evaluate v_2,s''}
  {e_1:=e_2,s\evaluate \unit,s''[l\mapsto v_2]}


\newrule{E-Sequence}
  {?}
  {e_1;e_2 \evaluate ?}

\newrule{E-Edit}
  {e,s \evaluate v,s'}
  {\Edit e , s\evaluate \Edit v,s'}

\newrule{E-Fill}
  {\ }
  {\Enter \tau,s \evaluate \Enter \tau,s}

\newrule{E-Update}
  {e,s\evaluate l,s'}
  {\Update e ,s\evaluate \Update l,s'}


\newrule{E-Fail}
  {\ }
  {\Fail,s \evaluate \Fail,s}


\newrule{E-Then}
  {e_1 ,s\evaluate t_1,s'}
  {e_1 \Then e_2,s \evaluate t_1 \Then e_2,s'}

\newrule{E-Next}
  {e_1 ,s\evaluate t_1,s'}
  {e_1 \Next e_2 ,s\evaluate t_1 \Next e_2,s'}


\newrule{E-And}
  {e_1 ,s\evaluate t_1 ,s'\\
   e_2 ,s'\evaluate t_2,s''}
  {e_1 \And e_2 ,s\evaluate t_1 \And t_2,s''}


\newrule{E-Or}
  {e_1 ,s\evaluate t_1 ,s'\\
   e_2 ,s'\evaluate t_2,s''}
  {e_1 \Or e_2 ,s\evaluate t_1 \Or t_2,s''}

\newrule{E-Xor}
  {\ }
  {e_1 \Xor e_2 ,s\evaluate e_1 \Xor e_2,s}


\newrule{E-Appoint}
    {e,s\evaluate t,s'}
    {u \At e,s\evaluate u \At t,s'}



%% Normalisation %%%%%%%%%%%%%%%%%%%%%%%%%%%%%%%%%%%%%%%%%%%%%%%%%%%%%%%%%%%%%%%


\newmacro{RelationS}
  {t,s \stride t',s'}


\newrule{S-Edit}
  { }
  {\Edit v,s \stride \Edit v,s}

\newrule{S-Fill}
  { }
  {\Enter \tau,s \stride \Enter \tau,s}

\newrule{S-Update}
  { }
  {\Update l,s \stride \Update l,s}


\newrule{S-Fail}
  { }
  {\Fail,s \stride \Fail,s}


\newrule{S-ThenStay}[\Value(t_1',s') = \bot]
  {t_1,s \stride t_1',s'}
  {t_1 \Then e_2,s \stride t_1' \Then e_2,s'}

\newrule{S-ThenFail}[\Value(t_1',s') = v_1 \land \Failing(t_2,s'')]
  {t_1,s \stride t_1',s' \\
   e_2\ v_1,s' \evaluate t_2,s''}
  {t_1 \Then e_2,s \stride t_1' \Then e_2,s'}

\newrule{S-ThenCont}[\Value(t_1',s') = v_1 \land \lnot\Failing(t_2,s'')]
  {t_1,s \stride t_1',s' \\
   e_2\ v_1,s' \evaluate t_2 ,s'' \\
   t_2,s'' \stride t_2',s'''}
  {t_1 \Then e_2,s \stride t_2',s'''}

\newrule{S-Next}
  {t_1,s \stride t_1',s'}
  {t_1 \Next e_2,s \stride t_1' \Next e_2,s'}


\newrule{S-And}
  {t_1,s  \stride t_1',s' \\
   t_2,s' \stride t_2',s''}
  {t_1 \And t_2,s \stride t_1' \And t_2',s''}


\newrule{S-OrLeft}[\Value(t_1',s') = v_1]
  {t_1,s  \stride t_1',s'}
  {t_1 \Or t_2,s \stride t_1',s'}

\newrule{S-OrRight}[\Value(t_1',s') = \bot \land \Value(t_2',s'') = v_2]
  {t_1,s  \stride t_1',s' \\
   t_2,s' \stride t_2',s''}
  {t_1 \Or t_2,s \stride t_2',s''}

\newrule{S-OrNone}[\Value(t_1',s') = \bot \land \Value(t_2',s'') = \bot]
  {t_1,s  \stride t_1',s' \\
   t_2,s' \stride t_2',s''}
  {t_1 \Or t_2,s \stride t_1' \Or t_2',s''}


\newrule{S-Xor}
  { }
  {e_1 \Xor e_2,s \stride e_1 \Xor e_2,s}

\newrule{S-Eval}[e \neq e']
    {e,s \evaluate e',s' \\
     e',s' \stride e'',s''}
    {e,s \stride e'',s''}


\newrule{S-Appoint}
  {t,s\stride t',s'}
  {u \At t,s\stride u \At t',s'}


% \newrule{S-Next}[t_1' = \Edit v]
%   {t_1,\Sigma \stride t_1'\st{'}   \\
%    e\ v \evaluate t_2       \\
%    t_2\st{'} \stride t_2'\st{''} }
%   {t_1 \Next e,\Sigma \stride t_2'\st{''}}

% \newrule{S-NextEval}
%   {e_1,\Sigma \stride u_1\st{'}}
%   {e_1 \Next e_2,\Sigma \stride u_1 \Next e_2\st{'}}




%% Normalisation %%


\newmacro{RelationN}
  {e,s \normalise t,s'}


\newrule{N-Done}[\Dirty(s,s'') \cap \Watching(t') = \nothing]
    {e,s \evaluate t,s' \\
     t,s' \stride t',s''}
    {t,s \normalise t',s''}


\newrule{N-Stride}[\Dirty(s,s'') \cap \Watching(t') \neq \nothing]
    {e,s \evaluate t,s' \\
     t,s' \stride t',s'' \\
     t',s'' \normalise t'',s'''}
    {e,s \normalise t'',s''}



%% Handling %%


\newmacro{RelationH}
  {t,s \handle{i} t',s'}

\newrule{H-Change}[v, v' : \tau]
  { }
  {\Edit v,s \handle{v'} \Edit v',s}

\newrule{H-Empty}[v : \tau]
  { }
  {\Edit v,s \handle{\Empty} \Enter \tau,s}

\newrule{H-Fill}[v' : \tau]
  { }
  {\Enter \tau,s \handle{v'} \Edit v',s}

\newrule{H-Update}[s(l), v' : \tau]
  { }
  {\Update l,s \handle{v'} \Update l,s[l \mapsto v']{}}


\newrule{H-PassThen}
  {t_1,s \handle{i} t_1',s'}
  {t_1 \Then e_2,s \handle{i} t_1' \Then e_2,s'}

\newrule{H-PassNext}
  {t_1,s \handle{i} t_1',s'}
  {t_1 \Next e_2,s \handle{i \neq \Continue} t_1' \Next e_2,s'}

\newrule{H-Next}[\Value{(t_1,s)}\equiv v_1 \wedge \neg\Failing{(t_2,s')}]
  {e_2 v_1,s\stride t_2,s'}
  {t_1 \Next e_2,s \handle{\Continue} t_2,s'}


\newrule{H-FirstAnd}
  {t_1,s \handle{i} t_1',s' }
  {t_1 \And t_2,s \handle{\First i} t_1' \And t_2,s'}

\newrule{H-SecondAnd}
  {t_2,s \handle{i} t_2',s'}
  {t_1,s \And t_2 \handle{\Second i} t_1 \And t_2',s'}


\newrule{H-FirstOr}
  {t_1,s \handle{i} t_1',s'}
  {t_1 \Or t_2,s \handle{\First i} t_1' \Or t_2,s'}

\newrule{H-SecondOr}
  {t_2,s \handle{i} t_2',s' }
  {t_1 \Or t_2,s \handle{\Second i} t_1 \Or t_2',s'}


\newrule{H-PickLeft}[e_1,s\evaluate t_1,s'\wedge \neg\Failing(t_1,s')]
  { }
  {e_1 \Xor e_2,s \handle{\Left} e_1,s}

\newrule{H-PickRight}[e_2,s\evaluate t_2,s'\wedge \neg\Failing(t_2,s')]
  { }
  {e_1 \Xor e_2,s \handle{\Right} e_2,s}


\newrule{H-Appoint}
  {t,s \handle{i} t',s'}
  {u \At t,s\handle{i} u \At t',s'}



%%%%


% \newrule{H-Stay'}[\Value(t_1) = \bot]
%   {\ }
%   {t_1 \Next e,\Sigma \handle{\Next} t_1 \Next e,\Sigma}

% \newrule{H-Next'}[\Value(t_1) = v]
%   {e\ v \evaluate t_2    \\
%    t_2,\Sigma \stride t_2'\st{'} }
%   {t_1 \Next e,\Sigma \handle{\Next} t_2'\st{'}}

% \newrule{H-Stay}[\Value(t_1) = \bot]
%   {\ }
%   {t_1 \Then e,\Sigma \handle{\Execute \pi} t_1 \Then e,\Sigma}

% \newrule{H-Fail'}[\Value(t_1) = v \land t_2 = \Fail]
%   {e\ v \evaluate t_2    \\
%    t_2,\Sigma \handle{\Pick \pi} t_2'\st{'} }
%   {t_1 \Then e,\Sigma \handle{\Execute \pi} t_1 \Then e,\Sigma}

% \newrule{H-Next}[\Value(t_1) = v \land t_2 \neq \Fail]
%   {e\ v \evaluate t_2    \\
%    t_2,\Sigma \handle{\Pick \pi} t_2'\st{'} }
%   {t_1 \Then e,\Sigma \handle{\Execute \pi} t_2'\st{'}}

% \newrule{H-PassS}
%   {t_1,\Sigma \handle{i} t_1'\st{'}}
%   {t_1 \Next e,\Sigma \handle{i} t_1' \Next e\st{'}}

% \newrule{H-Pass}[i \neq \Execute \pi]
%   {t_1,\Sigma \handle{i} t_1'\st{'}       \\
%    t_1' \Then e\st{'} \stride t_2\st{''} }
%   {t_1 \Then e,\Sigma \handle{i} t_2\st{''}}

% \newrule{H-Fallback}
%   { }
%   {t,\Sigma \handle{i} t,\Sigma}



%% Driving %%


\newmacro{RelationD}
  {t,s \drive{i} t',s'}


\newrule{D-Handle}
  {t,s \handle{i} t',s' \\
   t',s' \normalise t'',s''}
  {t,s \drive{i} t'',s''}

% !TEX root=pldi2019.tex


%% Language %%%%%%%%%%%%%%%%%%%%%%%%%%%%%%%%%%%%%%%%%%%%%%%%%%%%%%%%%%%%%%%%%%%%

\newmacro{G-Language}{
  \begin{grammar}
    Expressions
      & e    &::= & \lambda x:\tau.\ e   & – abstraction \\
      &      &\mid& e_1\ e_2             & – application \\
      &      &\mid& x                    & – variable \\
      &      &\mid& c                    & – constant \\
    \addlinespace
      &      &\mid& e_1 \star e_2        & – operation \\
      &      &\mid& \If{e_1}{e_2}{e_3}   & – branch \\
      &      &\mid& \tuple{e_1, e_2}     & – pair \\
      &      &\mid& \Fst e               & – first projection \\
      &      &\mid& \Snd e               & – second projection \\
      &      &\mid& \unit                & – unit \\
    \addlinespace
      &      &\mid& \Ref e               & – reference \\
      &      &\mid& !e                   & – dereference \\
      &      &\mid& e_1 := e_2           & – assignment \\
      % &      &\mid& e_1; e_2             & – sequence \\
      &      &\mid& l                    & – location \\
    \addlinespace
      &      &\mid& p                    & – pretask \\
    \addlinespace
    Constants
      & c    &::= & B                    & – boolean \\
      &      &\mid& I                    & – integer \\
      &      &\mid& S                    & – string
  \end{grammar}
}

\newmacro{G-Language-Compact}{
  \begin{grammar*}
    e ::= &                                                     & Expressions \\
    \mid  & \lambda x:\tau.\ e \Mid  e_1\ e_2                   & – abstraction, application \\
    \mid  & x \Mid  c                                           & – variable, constant \\
    \mid  & \If{e_1}{e_2}{e_3} \Mid e_1 \star e_2               & – branch, operation \\
    \mid  & \tuple{e_1, e_2} \Mid \Fst e \Mid \Snd e \Mid \unit & – pair, projections, unit \\
    \mid  & \Ref e \Mid  !e \Mid  e_1 := e_2 \Mid  l            & – references and locations \\
    \mid  & p                                                   & – pretask \\
    c ::= &                                                     & Constants \\
    \mid  & B \Mid  I \Mid  S                                   & – booleans, integers, strings
  \end{grammar*}
}

\newmacro{G-Pretasks}{
  \begin{grammar}
    Pretasks
      & p    &::= & \Edit e              & – valued editor \\
      &      &\mid& \Enter \tau          & – unvalued editor \\
      &      &\mid& \Update e            & – stored editor \\
    \addlinespace
      &      &\mid& e_1 \Then e_2        & – step \\
      &      &\mid& e_1 \Next e_2        & – user step \\
    \addlinespace
      &      &\mid& e_1 \And e_2         & – composition \\
    \addlinespace
      &      &\mid& e_1 \Or e_2          & – choice \\
      &      &\mid& e_1 \Xor e_2         & – user choice \\
    \addlinespace
      &      &\mid& u \At e              & – appoint \\
      &      &\mid& \Fail                & – fail task
  \end{grammar}
}

\newmacro{G-Pretasks-Compact}{
  \begin{grammar*}
    p ::= &                                              & Pretasks \\
    \mid  & \Edit e \Mid   \Enter \tau  \Mid   \Update e & – editors: valued, unvalued, stored \\
    \mid  & e_1 \Then e_2 \Mid   e_1 \Next e_2           & – steps: system, user \\
    \mid  & e_1 \And e_2                                 & – pairing \\
    \mid  & e_1 \Or e_2 \Mid   e_1 \Xor e_2              & – choice: system, user \\
    \mid  & u \At e \Mid   \Fail                         & – appoint, fail
  \end{grammar*}
}

\newmacro{G-Types}{
  \begin{grammar}
    Types
      & \tau &::= & \tau_1 \to \tau_2    & – function type \\
      &      &\mid& \tau_1 \times \tau_2 & – product type \\
      &      &\mid& \Unit                & – unit type \\
      &      &\mid& \Reference \tau      & – reference type \\
      &      &\mid& \Task \tau           & – task type \\
      &      &\mid& \beta                & – basic type \\
      % &      &\mid& \alpha               & – universal type \\
    Basic types
      &\beta &::= & \Bool                & – boolean type \\
      &      &\mid& \Int                 & – integer type \\
      &      &\mid& \String              & – string type \\
  \end{grammar}
}

\newmacro{G-Types-Compact}{
  \begin{grammar*}
    \tau ::=  &                                               & Types \\
    \mid      & \tau_1 \to \tau_2 \Mid   \tau_1 \times \tau_2 & – function type, product type \\
    \mid      & \Unit \Mid   \Reference \tau                  & – unit type, reference type \\
    \mid      & \Task \tau \Mid   \beta                       & – task type, basic type \\
    \beta ::= &                                               & Basic types \\
    \mid      & \Bool \Mid   \Int \Mid   \String              & – boolean, integer, string
  \end{grammar*}
}

\newmacro{G-Values}{
  \begin{grammar}
    Values
      & v    &::= & \lambda x:\tau.\ e   & – abstraction \\
      &      &\mid& \tuple{v_1, v_2}     & – pair value \\
      &      &\mid& \unit                & – unit \\
      &      &\mid& c                    & – constant \\
      &      &\mid& l                    & – location \\
      &      &\mid& t                    & – task \\
    Tasks
      & t    &::= & \Edit v              & – valued editor \\
      &      &\mid& \Enter \tau          & – unvalued editor \\
      &      &\mid& \Update l            & – stored editor \\
      &      &\mid& \Fail                & – fail task \\
      &      &\mid& t_1 \Then e_2        & – step \\
      &      &\mid& t_1 \Next e_2        & – user step \\
      &      &\mid& t_1 \And t_2         & – composition \\
      &      &\mid& t_1 \Or t_2          & – choice \\
      &      &\mid& t_1 \Xor t_2         & – user choice
  \end{grammar}
}

\newmacro{G-Values-Compact}{
  \begin{grammar*}
    v ::= &                                          & Values \\
      \mid& \lambda x:\tau.\ e                       & – abstraction \\
      \mid& \tuple{v_1, v_2} \Mid \unit              & – pair value, unit \\
      \mid& c \Mid l \Mid t                          & – constant, location, task \\
    t ::= &                                          & Tasks \\
      \mid& \Edit v \Mid \Enter \tau \Mid \Update l  & – editor tasks \\
      \mid& \Fail                                    & – fail task \\
      \mid& t_1 \Then e_2 \Mid t_1 \Next e_2         & – steps \\
      \mid& t_1 \And t_2                             & – composition \\
      \mid& t_1 \Or t_2 \Mid e_1 \Xor e_2            & – choices
  \end{grammar*}
}

\newmacro{G-Inputs}{
  \begin{grammar}
    Inputs
      & i    & ::=& a                    & – action \\
      &      &\mid& \First i             & – pass to first \\
      &      &\mid& \Second i            & – pass to second \\
    Actions
      & a    & ::=& v                    & – change editor to value \\
      &      &\mid& \Empty               & – empty an editor \\
      &      &\mid& \Continue            & – continue with next task \\
      &      &\mid& \Pick                & – pick route \\
    Routes
      & r    & ::=& \Left                & – go left \\
      &      &\mid& \Right               & – go right \\
  \end{grammar}
}

\newmacro{G-Inputs-Compact}{
  \begin{grammar*}
    i ::= &                                 & Inputs \\
    \mid  & a \Mid \First i \Mid \Second i  & – action, pass to first or second \\
    a ::= &                                 & Actions \\
    \mid  & v \Mid \Empty                   & – change to value, empty editor \\
    \mid  & \Continue \Mid \Pick r          & – continue, pick route \\
    r ::= &                                 & Routes \\
    \mid  &\Left \Mid \Right                & – go left or right \\
  \end{grammar*}
}

% !TEX root=pldi2019.tex


%% Language %%%%%%%%%%%%%%%%%%%%%%%%%%%%%%%%%%%%%%%%%%%%%%%%%%%%%%%%%%%%%%%%%%%%

\newmacro{O-Value}{
  \begin{flalign*}
    \begin{array}{lcl}
      \multicolumn{3}{l}{\Value : \mathrm{Tasks} \rightharpoonup \mathrm{Values}} \\
      \Value(\Edit v, s)       &=& v \\
      \Value(\Enter \tau, s)   &=& \bot \\
      \Value(\Update l, s)  &=& s(l) \\
      \Value(\Fail, s)         &=& \bot \\
      \Value(t_1 \Then e_2, s) &=& \bot \\
      \Value(t_1 \Next e_2, s) &=& \bot \\
      \Value(t_1 \And t_2, s)  &=& \left\{
        \begin{array}{ll}
          \tuple{v_1, v_2}  & \when\ \Value(t_1, s) = v_1 \land \Value(t_2, s) = v_2 \\
          \bot              & \otherwise
        \end{array}
      \right. \\
      \Value(t_1 \Or t_2, s)   &=& \left\{
        \begin{array}{ll}
          v_1               & \when\ \Value(t_1, s) = v_1 \\
          v_2               & \when\ \Value(t_1, s) = \bot \land \Value(t_2, s) = v_2 \\
          \bot              & \otherwise
        \end{array}
      \right. \\
      \Value(t_1 \Xor t_2, s)  &=& \bot\\
      \Value(u \At t, s)  &=& \Value(t,s)
    \end{array} & &&
  \end{flalign*}
}

\newmacro{O-Inputs}{
  \begin{flalign*}
    \begin{array}{lcl}
      \multicolumn{3}{l}{\Inputs : \mathrm{Tasks} \to \powerset(\mathrm{Inputs})} \\
      \Inputs(\Edit v:\Task\tau)   &=& \set{v':\tau, \Empty} \\
      \Inputs(\Enter \tau)         &=& \set{v':\tau} \\
      \Inputs(\Update l:\Task\tau) &=& \set{v':\tau} \\
      \Inputs(\Fail)               &=& \nothing \\
      \Inputs(t_1 \Then e_2)       &=& \Inputs(t_1) \\
      \Inputs(t_1 \Next e_2)       &=& \Inputs(t_1) \cup \set{\Continue \mid \Value{(t_1)} = v_1 \land \lnot\Failing(e_2 v_1)} \\
      \Inputs(t_1 \And t_2)        &=& \set{\First\ i \mid i \in \Inputs(t_1)} \cup \set{\Second\ i \mid i \in \Inputs(t_2)} \\
      \Inputs(t_1 \Or t_2)         &=& \set{\First\ i \mid i \in \Inputs(t_1)} \cup \set{\Second\ i \mid i \in \Inputs(t_2)} \\
      \Inputs(e_1 \Xor e_2)        &=& \set{\Left, \Right} \\
      \Inputs(u_1 \At t)           &=& \set{u_2 \At i\mid u_2 \At i \in \Inputs(t)} \cup \set{u_1 \At i\mid i\in \Inputs(t)}
    \end{array} & &&
  \end{flalign*}
}

\newmacro{O-Failing}{
  \begin{flalign*}
    \begin{array}{lcl}
      \multicolumn{3}{l}{\Failing : \mathrm{Tasks} \to \mathrm{Booleans}} \\
      \Failing(\Edit v,s)       &=& \False \\
      \Failing(\Enter \tau,s)   &=& \False \\
      \Failing(\Update l,s)     &=& \False \\
      \Failing(\Fail,s)         &=& \True \\
      \Failing(t_1 \Then e_2,s) &=& \Failing(t_1,s) \\
      \Failing(t_1 \Next e_2,s) &=& \Failing(t_1,s) \\
      \Failing(t_1 \And t_2,s)  &=& \Failing(t_1,s) \land \Failing(t_2,s) \\
      \Failing(t_1 \Or t_2,s)   &=& \Failing(t_1,s) \land \Failing(t_2,s) \\
      \Failing(e_1 \Xor e_2,s)  &=& \Failing(t_1,s') \land \Failing(t_2,s') \\
      \multicolumn{3}{l}{\qquad\where\ e_1,s\stride t_1,s' \quad e_2,s\stride t_2,s'} \\
      \Failing(u \At t,s)       &=& \Failing(t,s)
    \end{array} & &&
  \end{flalign*}
}

\newmacro{O-Watching}{
  \begin{flalign*}
    \begin{array}{lcl}
      \multicolumn{3}{l}{\Watching : \mathrm{Tasks} \to \powerset(\mathrm{Locations})} \\
      \Watching(\Edit v)       &=& \nothing \\
      \Watching(\Enter \tau)   &=& \nothing \\
      \Watching(\Update l)     &=& \set{l} \\
      \Watching(\Fail)         &=& \nothing\\
      \Watching(t_1 \Then e_2) &=& \Watching(t_1) \\
      \Watching(t_1 \Next e_2) &=& \Watching(t_1) \\
      \Watching(t_1 \And t_2)  &=& \Watching(t_1) \cup \Watching(t_2) \\
      \Watching(t_1 \Or t_2)   &=& \Watching(t_1) \cup \Watching(t_2) \\
      \Watching(e_1 \Xor e_2)  &=& \nothing \\
      \Watching(u \At t)       &=& \Watching(t)
    \end{array} & &&
  \end{flalign*}
}




%% Journal information
%% Supplied to authors by publisher for camera-ready submission;
%% use defaults for review submission.
% \acmJournal{PACMPL}
% \acmVolume{1}
% \acmNumber{ICFP} % CONF = POPL or ICFP or OOPSLA
% \acmArticle{1}
% \acmYear{2018}
% \acmMonth{1}
% \acmDOI{} % \acmDOI{10.1145/nnnnnnn.nnnnnnn}
% \startPage{1}

%% Copyright information
%% Supplied to authors (based on authors' rights management selection;
%% see authors.acm.org) by publisher for camera-ready submission;
%% use 'none' for review submission.
%\setcopyright{none}
%\setcopyright{acmcopyright}
\setcopyright{acmlicensed}
%\setcopyright{rightsretained}
%\copyrightyear{2018}           %% If different from \acmYear

%% Bibliography style
\bibliographystyle{ACM-Reference-Format}
%% Citation style
%% Note: author/year citations are required for papers published as an
%% issue of PACMPL.
\citestyle{acmnumeric}   %% For author/year citations



\begin{document}


\copyrightyear{2019}
\acmYear{2019}
\acmConference[PPDP '19]{Principles and Practice of Programming Languages 2019}{October 7--9, 2019}{Porto, Portugal}
\acmBooktitle{Principles and Practice of Programming Languages 2019 (PPDP '19), October 7--9, 2019, Porto, Portugal}
\acmPrice{15.00}
\acmDOI{10.1145/3354166.3354182}
\acmISBN{978-1-4503-7249-7/19/10}


%% Title information
\title{TopHat: A formal foundation for task-oriented programming}
% \title{TopHat: A language for modular interactive workflows}
                                        %% [Short Title] is optional;
                                        %% when present, will be used in
                                        %% header instead of Full Title.
%\titlenote{with title note}             %% \titlenote is optional;
                                        %% can be repeated if necessary;
                                        %% contents suppressed with 'anonymous'
\subtitle{Appendices}                   %% \subtitle is optional
%\subtitlenote{with subtitle note}       %% \subtitlenote is optional;
                                        %% can be repeated if necessary;
                                        %% contents suppressed with 'anonymous'


%% Author information
%% Contents and number of authors suppressed with 'anonymous'.
%% Each author should be introduced by \author, followed by
%% \authornote (optional), \orcid (optional), \affiliation, and
%% \email.
%% An author may have multiple affiliations and/or emails; repeat the
%% appropriate command.
%% Many elements are not rendered, but should be provided for metadata
%% extraction tools.

\author{Tim Steenvoorden}
%\authornote{with author1 note}          %% \authornote is optional; can be repeated if necessary
%\orcid{nnnn-nnnn-nnnn-nnnn}             %% \orcid is optional
\affiliation{
  %\position{PhD}
  \department{Software Science}
  %\department{Institute for Computing and Information Sciences}
                                        %% \department is recommended
  \institution{Radboud University}      %% \institution is required
  \streetaddress{Toernooiveld 212}
  \postcode{6525 EC}
  \city{Nijmegen}
  %\state{State1}
  \country{The Netherlands}
}
\email{tim@cs.ru.nl}                     %% \email is recommended

\author{Nico Naus}
%\authornote{with author1 note}          %% \authornote is optional; can be repeated if necessary
%\orcid{nnnn-nnnn-nnnn-nnnn}             %% \orcid is optional
\affiliation{
  %\position{PhD}
  \department{Information and Computing Sciences}
                                        %% \department is recommended
  \institution{Utrecht University}      %% \institution is required
  \streetaddress{Princetonplein 5}
  \postcode{3584 CC}
  \city{Utrecht}
  %\state{State1}
  \country{The Netherlands}
}
\email{n.naus@uu.nl}                    %% \email is recommended

\author{Markus Klinik}
%\authornote{with author1 note}          %% \authornote is optional; can be repeated if necessary
%\orcid{nnnn-nnnn-nnnn-nnnn}             %% \orcid is optional
\affiliation{
  %\position{PhD}
  \department{Software Science}
  %\department{Institute for Computing and Information Sciences}
                                        %% \department is recommended
  \institution{Radboud University}
                                        %% \institution is required
  \streetaddress{Toernooiveld 212}
  \postcode{6525 EC}
  \city{Nijmegen}
  %\state{State1}
  \country{The Netherlands}
}
\email{m.klinik@cs.ru.nl}               %% \email is recommended

%% Paper note
%% The \thanks command may be used to create a "paper note" ---
%% similar to a title note or an author note, but not explicitly
%% associated with a particular element.  It will appear immediately
%% above the permission/copyright statement.
%\thanks{with paper note}                %% \thanks is optional
                                        %% can be repeated if necesary
                                        %% contents suppressed with 'anonymous'


%% Abstract
%% Must come before \maketitle command
% \begin{abstract}
%   % !TEX root=../pldi2019.tex

Task-Oriented Programming (\TOP) is a programming paradigm that focusses on modelling real world collaborations between people.
It prescribes a declarative programming style to specify multi-user workflows.
Workflows can be higher-order.
They communicate through typed values on a local or global level.
Such specifications can be turned into interactive applications for different platforms,
supporting collaboration during execution.

In this paper we decompose the rich features of \TOP into elementary language elements,
which makes them suitable for formal treatment.
We use the simply typed lambda calculus, extended with pairs and references, as a base language.
On top of this language, we develop \TOPHAT (TopHat), a calculus for modular interactive workflows.

We describe \TOPHAT by means of a layered semantics.
These layers consist of multiple big step evaluations on expressions,
and two labelled transition systems, handling user inputs.
We show some interesting properties of this machinery.
This approach allows for comparison with other work in the field.
We place \TOPHAT in perspective with the process calculus \CSP.

% \end{abstract}

% \begin{teaserfigure}
%    \includegraphics[width=\textwidth]{figures/declrequest-part.pdf}
%    \caption{This is a teaser}
%    \label{fig:teaser}
% \end{teaserfigure}

%% 2012 ACM Computing Classification System (CSS) concepts
%% Generate at 'http://dl.acm.org/ccs/ccs.cfm'.

%% End of generated code


%% Keywords
%% comma separated list, optional
% \keywords{formal methods, operational semantics, functional programming language, task oriented programming, workflow modelling, program generation}

%% Note: \maketitle command must come after title commands, author
%% commands, abstract environment, Computing Classification System
%% environment and commands, and keywords command.
\maketitle

% %% Acknowledgments
% \begin{acks}                            %% acks environment is optional
%                                         %% contents suppressed with 'anonymous'
%   %% Commands \grantsponsor{<sponsorID>}{<name>}{<url>} and
%   %% \grantnum[<url>]{<sponsorID>}{<number>} should be used to
%   %% acknowledge financial support and will be used by metadata
%   %% extraction tools.
%   \small
%   % !TEX root=../icfp2019.tex

The authors are grateful to Johan Jeuring, Sjaak Smetsers, and Andreas Vinter-Hviid for fruitful discussions,
and Rinus Plasmeijer, Peter Achten and Pieter Koopman for proofreading.

This research is supported by the Dutch Technology Foundation STW, which is part
of the Netherlands Organisation for Scientific Research (NWO), and which is
partly funded by the Ministry of Economic Affairs.

% \end{acks}


%% Bibliography
%\bibliography{bibliography}


%% Appendix
\appendix

\onecolumn
% !TEX root=../main.tex

\section{Additional rules}

\subsection{Evaluation rules}

% \begin{figure}

  \begin{gather*}
    \boxed{\RelationE} \Break
    \userule{E-App} \Quad
    \userule{E-IfTrue} \Quad
    \userule{E-Ref} \Break
    \userule{E-IfFalse} \Quad
    \userule{E-Deref} \Quad
    \userule{E-Value} \Quad
    \userule{E-Assign} \Break
    \userule{E-Pair} \Quad
    \userule{E-Edit} \Quad
    \userule{E-Enter} \Quad
    \userule{E-Update} \Quad
    \userule{E-Then} \Break
    \userule{E-Next} \Quad
    \userule{E-And} \Quad
    \userule{E-Fail} \Quad
    \userule{E-Or} \Quad
    \userule{E-Xor}
    %\userule{E-Appoint}
  \end{gather*}

%   \caption{Evaluation semantics} \label{fig:evaluation-semantics}
% \end{figure}

\subsection{Typing rules}

% \begin{figure}[h]

  \begin{gather*}
    \boxed{\RelationT} \Break
    \userule{T-Var} \Quad
    \userule{T-Loc} \Quad
    \userule{T-Pair} \Quad
    \userule{T-Abs} \Quad
    \userule{T-App} \Break
    \userule{T-If} \Quad
    \userule{T-Ref} \Quad
    \userule{T-Deref} \Quad
    \userule{T-Assign}
  \end{gather*}

%   \caption{Additional typing rules}
% \end{figure}

\pagebreak
% !TEX root=../icfp2019.tex

\section{Proofs}

\subsection{Theorem~\ref{thm:pres-eval}}

\begin{proof}
  We prove Theorem~\ref{thm:pres-eval} by induction on $e$:\\

  \case
    {$e=\lambda x:\tau.e, e_1 e_2, x, c, l, e_1 \star e_2,\If{e_1}{e_2}{e_3},\tuple{e_1, e_2},\unit,\Ref e,!e,e_1 := e_2$}
    {Preservation has been proven for these cases by Pierce~\cite{books/Pierce02TAPL}.}

  \case
    {$\userule{E-Edit}$}
    {Given that $\Gamma,\Sigma\infers\Edit e:\Task \tau$ and $\Sigma\infers s$,\refrule{T-Edit} gives us that $\Gamma,\Sigma\infers e:\tau$.
    The induction hypothesis gives us that $e,s\evaluate v,s'$ also preserves, and thus $\Gamma,\Sigma\infers v:\tau$ and $\Sigma\infers s'$.
    Therefore $\Gamma,\Sigma\infers\Edit v:\Task\tau$.}

  \case
    {$\userule{E-Enter}$}
    {Evaluation does not alter $e$ and $s$, therefore this case holds trivially.}

  \case
    {$\userule{E-Update}$}
    {Given that $\Gamma,\Sigma\infers \Edit e:\Task \tau$ and $\Sigma\infers s$, \refrule{T-Update} gives us that $\Gamma,\Sigma\infers e:\Ref \tau$.
    The induction hypothesis gives us that $e,s\evaluate l,s'$ also preserves, and thus $\Gamma,\Sigma\infers l:\Ref\tau$ and $\Sigma\infers s'$.
    Therefore $\Gamma,\Sigma\infers\Update l:\Task\tau$.}

  \case
    {$\userule{E-Fail}$}
    {Evaluation does not alter $e$ and $s$, therefore this case holds trivially.}

  \case
    {$\userule{E-Then}$}
    {Given that $\Gamma,\Sigma\infers e_1\Then e_2:\Task \tau$ and $\Sigma\infers s$, \refrule{T-Then} gives us that $\Gamma,\Sigma\infers e_1:\Task\tau_1$ and $\Gamma,\Sigma\infers e_2:\tau_1 \to \Task \tau$.
    By the induction hypothesis, we know that $e_1,s\evaluate t_1,s'$ preserves and thus $\Gamma,\Sigma\infers t_1:\Task\tau_1$ and $\Sigma\infers s'$.
    Therefore $\Gamma,\Sigma\infers t_1\Then e_2:\Task\tau$.}

  \case
    {$\userule{E-Next}$}
    {Given that $\Gamma,\Sigma\infers e_1\Next e_2:\Task \tau$ and $\Sigma\infers s$, \refrule{T-Next} gives us that $\Gamma,\Sigma\infers e_1:\Task\tau_1$ and $\Gamma,\Sigma\infers e_2:\tau_1 \to \Task \tau$.
    By the induction hypothesis, we know that $e_1,s\evaluate t_1,s'$ preserves and thus $\Gamma,\Sigma\infers t_1:\Task\tau_1$ and $\Sigma\infers s'$.
    Therefore $\Gamma,\Sigma\infers t_1\Next e_2:\Task\tau$.}

  \case
    {$\userule{E-And}$}
    {Given that $\Gamma,\Sigma\infers e_1\And e_2:\Task(\tau_1\times\tau_2)$ and $\Sigma\infers s$, \refrule{T-And} gives us that $\Gamma,\Sigma\infers e_1:\Task\tau_1$ and $\Gamma,\Sigma\infers e_2:\Task\tau_2$.
    By the induction hypothesis, we know that both $e_1,s\evaluate t_1,s'$ and $e_2,s'\evaluate t_2,s''$ preserve and thus $\Gamma,\Sigma\infers t_1:\Task\tau_1$, $\Sigma\infers s'$, $\Gamma,\Sigma\infers t_2:\Task\tau_2$ and $\Sigma\infers s''$.
    Therefore $\Gamma,\Sigma\infers t_1\And t_2:\Task(\tau_1\times\tau_2)$.}

  \case
    {$\userule{E-Or}$}
    {Given that $\Gamma,\Sigma\infers e_1\Or e_2:\Task\tau$ and $\Sigma\infers s$, \refrule{T-Or} gives us that $\Gamma,\Sigma\infers e_1:\Task\tau$ and $\Gamma,\Sigma\infers e_2:\Task\tau$.
    By the induction hypothesis, we have that both $e_1,s\evaluate t_1,s'$ and $e_2,s'\evaluate t_2,s''$ preserve and thus $\Gamma,\Sigma\infers t_1:\Task\tau$, $\Sigma\infers s'$, $\Gamma,\Sigma\infers t_2:\Task\tau$ and $\Sigma\infers s''$.
    Therefore $\Gamma,\Sigma\infers t_1\Or t_2:\Task\tau$.}

  \case
    {$\userule{E-Xor}$}
    {Evaluation does not alter $e$ and $s$, therefore this case holds trivially.}

  % \noindent\textbf{Case} $\userule{E-Appoint}$
  %     Given that $\Gamma,\Sigma\infers u \At e:\Task\tau$ and $\Sigma\infers s$, the induction hypothesis gives us that $e,s\evaluate t,s'$ also preserves, and therefore by \refrule{T-Appoint} $\Gamma,\Sigma\infers u \At t:\Task\tau$.
\end{proof}



\subsection{Lemma~\ref{lem:presvalue}}

\begin{proof}
  We prove Lemma~\ref{lem:presvalue} by induction over $e$.\\

\case {$\Value{(\Edit v,s)}=v$}
      {By T-Edit, if $\Gamma,\Sigma\infers \Edit v:\Task\tau$, then $\Gamma,\Sigma\infers v:\tau$.}

\case {$\Value{(\Enter \tau,s)}=\bot$ }
      {Since this case does not lead to a value, the lemma holds trivially.}

\case {$\Value{(\Update l,s)}=s(l)$ }
      {Given that $\Gamma,\Sigma\infers\Update l:\Task \tau$ and $\Sigma\infers s$,
      we know that $\Gamma,\Sigma\infers s(l):\tau$ by definiton.}

\case {$\Value{(\Fail,s)}=\bot$ }
      {Since this case does not lead to a value, the lemma holds trivially.}

\case {$\Value{(t_1\Then e_2,s)}=\bot$}
      { Since this case does not lead to a value, the lemma holds trivially.}

\case {$\Value{(t_2\Next e_2,s)}=\bot$ }
      {Since this case does not lead to a value, the lemma holds trivially.}

\case {$\Value{(t_1\And t_2,s)}=\tuple{v_1, v_2}$ given that $\Value{(t_1,s)}=v_1\wedge\Value{(t_2,s)}=v_2$}
      {By \refrule{T-And} we have that $\Gamma,\Sigma\infers t_1\And t_2:\Task(\tau_1\times\tau_2)$ and $\Gamma,\Sigma\infers t_1:\tau_1$ and $\Gamma,\Sigma\infers t_2:\tau_2$.
      By the induction hypothesis, $ \Value{(t_1,s)}=v_1$ and $\Value{(t_2,s)}=v_2$ preserve, and thus $\Gamma,\Sigma\infers v_1:\tau_1$ and $\Gamma,\Sigma\infers v_2:\tau_2$.
      This gives us that $\Gamma,\Sigma\infers \tuple{v_1, v_2}:\Task(\tau_1\times\tau_2)$.}

\case {$\Value{(t_1\And t_2,s)}=\bot$ given that $\neg(\Value{(t_1,s)}=v_1\wedge\Value{(t_2,s)}=v_2)$}
      { Since this case does not lead to a value, the lemma holds trivially.}

\case
{$\Value{(t_1\Or t_2,s)}=v_1$ given that
  $\Value{(t_1,s)}=v_1$}{
  By \refrule{T-Or} we have that $\Gamma,\Sigma\infers t_1\Or t_2:\Task\tau$,
  and $\Gamma,\Sigma\infers t_1:\Task\tau$ and
  $\Gamma,\Sigma\infers t_2:\Task\tau$. By the induction hypothesis, we have
  that $\Gamma,\Sigma\infers v_1:\tau$.}

\case
{$\Value{(t_1\Or t_2,s)}=v_2$ given that
  $\Value{(t_1,s)}=\bot\wedge\Value{(t_2,s)}=v_2$}{
  By \refrule{T-Or} we have that $\Gamma,\Sigma\infers t_1\Or t_2:\Task\tau$,
  and $\Gamma,\Sigma\infers t_1:\Task\tau$ and
  $\Gamma,\Sigma\infers t_2:\Task\tau$. By the induction hypothesis, we have
  that $\Gamma,\Sigma\infers v_2:\tau$.}

\case
{$\Value{(t_1\Or t_2,s)}=\bot$ given that
  $\Value{(t_1,s)}=\bot\wedge\Value{(t_2,s)}=\bot$}{ Since this case does not
  lead to a value, the lemma holds trivially.}

\case{$\Value{(t_1\Xor t_2,s)}=\bot$ }{Since this case does not
  lead to a value, the lemma holds trivially.}

  % \noindent\textbf{Case} $\Value{(u\At t,s)}=\Value(t,s)$ This case follows
  % directly from the induction hypothesis.\\
\end{proof}



\subsection{Theorem~\ref{thm:pres-stride}}

\begin{proof}
  We prove Theorem~\ref{thm:pres-stride} by induction on $e$:\\

\case{$\userule{S-Fail}$}
     {Since this case does not alter the expression, the theorem holds trivially.}

\case{$\userule{S-Xor}$}
     {Since this case does not alter the expression, the theorem holds trivially.}

\case{$\userule{S-Update}$}
     {Since this case does not alter the expression, the theorem holds trivially.}

\case{$\userule{S-Fill}$}
     {Since this case does not alter the expression, the theorem holds trivially.}

\case{$\userule{S-Edit}$}
     {Since this case does not alter the expression, the theorem holds trivially.}

\case{$\userule{S-And}$}
     {Given that $\Gamma,\Sigma\infers t_1\And t_2:\Task(\tau_1\times\tau_2)$, by \refrule{T-And we} have $\Gamma,\Sigma\infers t_1:\tau_1$ and $\Gamma,\Sigma\infers t_2:\tau_2$.
     By the induction hypothesis, we also have $\Gamma,\Sigma\infers t_1':\tau_1$ and $\Gamma,\Sigma\infers t_2':\tau_2$.
     This gives us that $\Gamma,\Sigma\infers t_1'\And t_2':\Task(\tau_1\times\tau_2)$.}

\case{$\userule{S-Next}$}
       {Given that
  $\Gamma,\Sigma\infers e_1\Next e_2:\Task \tau$, \refrule{T-Then} gives us that $\Gamma,\Sigma\infers t_1:\Task\tau_1$ and
  $\Gamma,\Sigma\infers e_2:\tau_1 \to \Task \tau$. By the induction hypothesis,
  we know that $t_1\stride t_1'$ preserves and thus
  $\Gamma,\Sigma\infers t_1':\Task\tau_1$. Therefore
  $\Gamma,\Sigma\infers t_1'\Next e_2:\Task\tau$.}

\case{$\userule{S-OrLeft}$}
       {Given that
  $\Gamma,\Sigma\infers t_1\Or t_2:\Task\tau$, by \refrule{T-Or} we have
  $\Gamma,\Sigma\infers t_1:\Task\tau$. By the induction hypothesis, we know that
  $t_1\stride t_1'$ preserves and thus $\Gamma,\Sigma\infers t_1':\Task\tau$.}

\case{$\userule{S-OrRight}$ }
     {Given that $\Gamma,\Sigma\infers t_1\Or t_2:\Task\tau$, by \refrule{T-Or} we have $\Gamma,\Sigma\infers t_2:\Task\tau$.
     By the induction hypothesis, we know that $t_2\stride t_2'$ preserves and thus $\Gamma,\Sigma\infers t_2':\Task\tau$.}

  \case{$\userule{S-OrNone}$}
  {Given that $\Gamma,\Sigma\infers t_1\Or t_2:\Task\tau$, by \refrule{T-Or} we have $\Gamma,\Sigma\infers t_1:\Task\tau$ and $\Gamma,\Sigma\infers t_2:\Task\tau$.
  By the induction hypothesis, we know that $t_1\stride t_1'$ and $t_2\stride t_2'$ preserve, and thus $\Gamma,\Sigma\infers t_1'\Or t_2':\Task\tau$.}

  \case{$\userule{S-ThenStay}$ }
  {Given that $\Gamma,\Sigma\infers t_1\Then e_2:\Task\tau$, by \refrule{T-Then} we have $\Gamma,\Sigma\infers t_1:\Task\tau_1$ and $\Gamma,\Sigma\infers e_2:\tau_1\to\Task\tau$.
  By the induction hypothesis, we know that $t_1\stride t_1'$ preserves, and thus $\Gamma,\Sigma\infers t_1'\Then e_2:\Task\tau$.}

  \case{$\userule{S-ThenFail}$}
  {Given that $\Gamma,\Sigma\infers t_1\Then e_2:\Task\tau$,
  by \refrule{T-Then} we have $\Gamma,\Sigma\infers t_1:\Task\tau_1$ and
  $e_2:\tau_1\to\Task\tau$. By the induction
  hypothesis, we know that $t_1\stride t_1'$ preserves, and thus
  $\Gamma,\Sigma\infers t_1'\Then e_2:\Task\tau$.}

  \case{$\userule{S-ThenCont}$}
  {Given that $\Gamma,\Sigma\infers t_1\Then e_2:\Task\tau$, by \refrule{T-Then} we have
  $\Gamma,\Sigma\infers t_1:\Task\tau_1$ and
  $\Gamma,\Sigma\infers e_2:\tau_1\to\Task\tau$. By the induction hypothesis, we
  know that $t_1\stride t_1'$ preserves. By Lemma~\ref{lem:presvalue}, we know
  that $\Value{(t_1')}=v_1$ preserves. By Theorem~\ref{thm:pres-eval} we know
  that $e_2 v_1\evaluate t_2$ preserves. And finally by the induction hypothesis,
  we know that $t_2\stride t_2'$ preserves. Therefore
  $\Gamma,\Sigma\infers t_2':\Task\tau$.}
  %
  % \noindent\textbf{Case} $\userule{S-Appoint}$ Given that
  % $\Gamma,\Sigma\infers u\At t:\Task\tau$, by \refrule{T-Appoint} we have
  % $t:\Task\tau$. By the induction hypothesis, we know that $t,s\stride t',s'$
  % preserves, and thus $\Gamma,\Sigma\infers u\At t':\Task\tau$.\\

\end{proof}



\subsection{Theorem~\ref{thm:pres-norm}}

\begin{proof}
  We prove Theorem~\ref{thm:pres-norm} by induction on $e$:\\

  \noindent\textbf{Case} $\userule{N-Done}$ \\
  \indent Given that
  $\Gamma,\Sigma\infers e:\Task\tau$ and $\Sigma\infers s$, we know that
  $\Gamma,\Sigma\infers t:\Task\tau$ and $\Sigma\infers s'$ by
  Theorem~\ref{thm:pres-eval}. Then by Theorem~\ref{thm:pres-stride}, we have
  $\Gamma,\Sigma\infers t':\Task\tau$ and $\Sigma\infers s''$.\\

  \noindent\textbf{Case}\\
  $\userule{N-Repeat}$ \\
  \indent Given that
  $\Gamma,\Sigma\infers e:\Task\tau$ and $\Sigma\infers s$, we know that
  $\Gamma,\Sigma\infers t:\Task\tau$ and $\Sigma\infers s'$ by
  Theorem~\ref{thm:pres-eval}. Then by Theorem~\ref{thm:pres-stride}, we have
  $\Gamma,\Sigma\infers t':\Task\tau$ and $\Sigma\infers s''$. Then by the
  induction hypothesis, we finally obtain that
  $\Gamma,\Sigma\infers t'':\Task\tau$ and $\Sigma\infers s'''$.\\

\end{proof}



\subsection{Theorem~\ref{thm:pres-handle}}

We require the following Lemma for this proof.

\begin{lemma}
  Given that $\Sigma\infers s$, $\Sigma(l)=\tau$ and $\Gamma,\Sigma\infers v:\tau$, it holds that $\Sigma\infers s[l\mapsto v]$
  \label{lemmasigmaconsistent}
\end{lemma}
This lemma follows immediately from definition.

\begin{proof}
  We prove Theorem~\ref{thm:pres-handle} by induction on $e$:

  \case
    {$\userule{H-Change}$}
    {Given that $\Gamma,\Sigma\infers\Edit v:\Task\tau$ and $\Sigma\infers s$, the \refrule{H-Change} rule additionally gives us that $v,v':\tau$. Therefore by \refrule{T-Edit} we have that $\Gamma,\Sigma\infers\Edit v':\Task\tau$.}

  \case
    {$\userule{H-Empty}$}
    {Given that $\Gamma,\Sigma\infers\Edit v:\Task\beta$ and $\Sigma\infers s$, the \refrule{H-Empty} rule additionally gives us that $v:\tau$.
    Then by \refrule{T-Enter} we have $\Gamma,\Sigma\infers\Enter \tau:\Task\tau$.}

  \case
    {$\userule{H-Fill}$}
    {Given that $\Gamma,\Sigma\infers\Enter\tau$ and $\Sigma\infers s$, the \refrule{H-Fill} rule additionally gives us that $v':\tau$.
    Then by \refrule{T-Enter} we have $\Gamma,\Sigma\infers \Edit v':\Task\tau$.}

  \case
    {$\userule{H-Update}$}
    {Given that $\Gamma,\Sigma\infers\Update l:\Task\tau$ and $\Sigma\infers s$.
    This gives us that $\Sigma(l)=\tau$, and we additionally obtain $s(l),v':\tau$ by \refrule{H-Update}.
    By application of Lemma~\ref{lemmasigmaconsistent} this case holds.}

  \case
    {$\userule{H-PickLeft}$}
    {Given that $\Gamma,\Sigma\infers t_1\Xor t_2:\Task\tau$ and $\Sigma\infers s$,
    then by \refrule{T-Xor} we have $\Gamma,\Sigma\infers t_1:\Task \tau$.}

  \case
    {$\userule{H-PickRight}$}
    {Given that $\Gamma,\Sigma\infers t_1\Xor t_2:\Task\tau$ and $\Sigma\infers s$,
    then by \refrule{T-Xor} we have $\Gamma,\Sigma\infers t_2:\Task \tau$.}

  \case
    {$\userule{H-Next}$}
    {Given that $\Gamma,\Sigma\infers t_1\Next e_2 :\Task\tau$ and $\Sigma\infers s$.
    Then by \refrule{T-Next}, we have $\Gamma,\Sigma\infers t_1:\Task\tau_1$ and $\Gamma,\Sigma\infers e_2:\tau_1\to\Task\tau$.
    Then by \refrule{T-Then} we obtain $\Gamma,\Sigma\infers t_1\Then e_2:\Task\tau$.}

  \case
    {$\userule{H-PassThen}$}
    {Given that $\Gamma,\Sigma\infers t_1\Then e_2:\Task\tau$ and $\Sigma\infers s$,
    \refrule{T-Then} gives us that $\Gamma,\Sigma\infers t_1:\Task\tau_1$ and $\Gamma,\Sigma\infers e_2:\tau_1\to\Task\tau$.
    By the induction hypothesis, we know that $t_1,s\handle{i}t_1',s'$ also preserves and thus $\Gamma,\Sigma\infers t_1':\Task\tau_1$ and $\Gamma,Sigma\infers s'$.
    By \refrule{T-Then} we now obtain that $\Gamma,\Sigma\infers t_1'\Then e_2:\Task\tau$.}

  \case
    {$\userule{H-PassNext}$}
    {Given that
  $\Gamma,\Sigma\infers t_1\Next e_2:\Task\tau$ and $\Sigma\infers s$,
  \refrule{T-Next} gives us that $\Gamma,\Sigma\infers t_1:\Task\tau_1$ and
  $\Gamma,\Sigma\infers e_2:\tau_1\to\Task\tau$. By the induction hypothesis, we
  know that $t_1,s\handle{i}t_1',s'$ also preserves and thus
  $\Gamma,\Sigma\infers t_1':\Task\tau_1$ and $\Gamma,Sigma\infers s'$. By
  \refrule{T-Next} we now obtain that
  $\Gamma,\Sigma\infers t_1'\Next e_2:\Task\tau$. }

  \case
    {$\userule{H-FirstAnd}$}
    {Given that
  $\Gamma,\Sigma\infers t_1\And t_2:\Task(\tau_1\times\tau_2)$ and
  $\Sigma\infers s$, \refrule{T-And} gives us that
  $\Gamma,\Sigma\infers t_1:\Task\tau_1$ and
  $\Gamma,\Sigma\infers t_2:\Task\tau_2$. By the induction hypothesis, we know
  that $t_1,s\handle{i}t_1',s'$ also preserves and thus
  $\Gamma,\Sigma\infers t_1':\Task\tau_1$ and $\Sigma\infers s'$. Therefore by
  \refrule{T-Next} we obtain
  $\Gamma,\Sigma\infers t_1'\And t_2:\Task(\tau_1\times\tau_2)$.}

  \case
    {$\userule{H-SecondAnd}$}
    {Given that
  $\Gamma,\Sigma\infers t_1\And t_2:\Task(\tau_1\times\tau_2)$ and
  $\Sigma\infers s$, \refrule{T-And} gives us that
  $\Gamma,\Sigma\infers t_1:\Task\tau_1$ and
  $\Gamma,\Sigma\infers t_2:\Task\tau_2$. By the induction hypothesis, we know
  that $t_2,s\handle{i}t_2',s'$ also preserves and thus
  $\Gamma,\Sigma\infers t_2':\Task\tau_2$ and $\Sigma\infers s'$. Therefore by
  \refrule{T-Next} we obtain
  $\Gamma,\Sigma\infers t_1\And t_2':\Task(\tau_1\times\tau_2)$.}

  \case
    {$\userule{H-FirstOr}$}
    {Given that
  $\Gamma,\Sigma\infers t_1\Or t_2:\Task\tau$ and $\Sigma\infers s$,
  \refrule{T-Or} gives us that $\Gamma,\Sigma\infers t_1:\Task\tau$ and
  $\Gamma,\Sigma\infers t_2:\Task\tau$. By the induction hypothesis we know that
  $t_1,s\handle{i}t_1',s'$ also preserves, and therefore
  $\Gamma,\Sigma\infers t_1':\Task\tau$ and $\Sigma\infers s'$. By
  \refrule{T-Or} we now obtain $\Gamma,\Sigma\infers t_1'\Or t_2:\Task\tau$.}

  \case
    {$\userule{H-SecondOr}$}
    {Given that $\Gamma,\Sigma\infers t_1\Or t_2:\Task\tau$ and $\Sigma\infers s$, \refrule{T-Or} gives us that $\Gamma,\Sigma\infers t_1:\Task\tau$ and $\Gamma,\Sigma\infers t_2:\Task\tau$.
    By the induction hypothesis we know that $t_2,s\handle{i}t_2',s'$ also preserves, and therefore $\Gamma,\Sigma\infers t_2':\Task\tau$ and $\Sigma\infers s'$.
    By T-Or we now obtain $\Gamma,\Sigma\infers t_1\Or t_2':\Task\tau$.}

  % \noindent\textbf{Case} $\userule{H-Appoint}$ Given that
  % $\Gamma,\Sigma\infers u\At t:\Task\tau$, \refrule{T-Assign} gives us that
  % $\Gamma,\Sigma\infers t:\Task\tau$. By the induction hypothesis, we know that
  % $t,s\handle{i}t',s'$ also preserves, and therefore
  % $\Gamma,\Sigma\infers u\At t'$ and $\Sigma\infers s'$.
\end{proof}

% \subsection{Theorem~\ref{thmprogressnorm}}
%
% \begin{proof} We prove this Theorem by induction on $e$.\\
%
%   \noindent\textbf{Case} $\userule{S-Edit}$ Given that $\Gamma,\Sigma\infers \Edit v : \Task \tau$, let $e''=\Edit v'$, $s''=s$ and $i=v':\tau$. Then H-Edit gives $\Edit v,s\handle{v'}\Edit v',s$. \\
%
%
%   \noindent\textbf{Case} $\userule{S-Fill}$ Given that $\Gamma,\Sigma\infers \Enter \tau : \Task \tau$, let $e''=\Edit v$, $s''=s$ and $i=v:\tau$. Then H-Fill gives $\Enter \tau,s\handle{v}\Edit v,s$. \\
%
%   \noindent\textbf{Case} $\userule{S-Update}$ Given that $\Gamma,\Sigma\infers \Update l : \Task \tau$, let $e''=\Update l$, $s''=s[l\mapsto v]$ and $i=v:\tau$. Then H-Update gives $\Update l ,s\handle{v}\Update l,s[l\mapsto v]$. \\
%
%   \noindent\textbf{Case} $\userule{S-Fail}$ Since $\Failing(\Fail)=\True$, the theorem holds in this case. \\
%
%   \noindent\textbf{Case} $\userule{S-Xor}$ \todo{Here we have an issue with the defintion of $\Failing$ and the sideconditions of external choice. These definitions should concur, so we can finish the proof, but $\Failing$ is a static property}\\
%
%   \noindent\textbf{Case} $\userule{S-Eval}$ By induction hypothesis we have that there exists an $e'''$, $s'''$ and $i$ such that $e'',s''\handle{i}e''',s'''$.\\
%
%   \noindent\textbf{Case} $\userule{S-ThenStay}$ By the induction hyposthesis we have that there exists an $e''$, $s''$ and $i$ such that $t_1',s'\handle{i}e'',s''$. Then by H-Pass, we know that we can apply the same $i$ to obtain $t_1\Then e_2,s'\handle{i} e''\Then e,s''$.\\
%
%   \noindent\textbf{Case} $\userule{S-ThenFail}$ By the induction hyposthesis we have that there exists an $e''$, $s''$ and $i$ such that $t_1',s'\handle{i}e'',s''$. Then by H-Pass, we know that we can apply the same $i$ to obtain $t_1\Then e_2,s'\handle{i} e''\Then e,s''$.\\
%
%   \noindent\textbf{Case} $\userule{S-ThenCont}$ By the induction hypothesis, we have that there exists an $e''$, $s''''$ and $i$ such that $t_2',s''\handle{i}e'',s''''$.\\
%
%   \noindent\textbf{Case} $\userule{S-OrLeft}$ By the induction hypothesis, we have that there exists an $e''$, $s'$ and $i$ such that $t_1',s\handle{i}e'',s''$.\\
%
%   \noindent\textbf{Case} $\userule{S-OrRight}$ By the induction hypothesis, we have that there exists an $e''$, $s'$ and $i$ such that $t_2',s\handle{i}e'',s''$.\\
%
%   \noindent\textbf{Case} $\userule{S-OrNone}$ \todo{   }\\
%
%   \noindent\textbf{Case} $\userule{S-Next}$ By the induction hypothesis, we have that there exists an $e''$, $s'$ and $i$ such that $t_1',s'\handle{i}e'',s''$. Then by the H-PassNext rule, we know that we can apply the same $i$ to obtain $t_1'\Next e_2,s'\handle{i}e'',s''$.\\
%
%   \noindent\textbf{Case} $\userule{S-And}$ By the induction hypothesis, we have that there exists an $e''$, $s'$ and $i$ such that $t_1',s'\handle{i}e'',s''$. Then by the H-FirstOr rule, we know that we can apply $\First i$ to obtain $t_1'\And t_2',s'\handle{\First i} t_1''\And t_2',s''$
%
% \end{proof}

% \subsection{Lemma~\ref{lem:stride-does-not-eval}}
%
% \begin{proof}
%   We prove Lemma~\ref{lem:stride-does-not-eval} by induction on $e$. First we
%   apply evaluation, and we then find ourselves in one of the cases below.\\
%
%   \noindent\textbf{Case} $\userule{S-ThenStay}$\\
%   and $\userule{S-ThenFail}$ The \refrule{E-Then} rule gives us that evaluation
%   has been applied to $t_1$. We therefore can apply the induction hypothesis,
%   from which we obtain that $t_1',s'\evaluate t_1'',s''$ then $t_1'=t_1''$ and
%   $s'=s''$. This, together with the \refrule{E-Then} rule, proves this case.\\
%
%   \noindent\textbf{Case} $\userule{S-ThenCont}$ By application of the induction
%   hypothesis, we obtain that $t_2',s'''\evaluate t_2'',s''''$ then $t_2'=t_2''$
%   and $s'''=s''''$, which immediately proves this case.\\
%
%   \noindent\textbf{Case} $\userule{S-OrLeft}$ The \refrule{E-Or} rule gives us
%   that evaluation has been applied to $t_1$. We therefore can apply the
%   induction hypothesis, from which we obtain that $t_1',s'\evaluate t_1'',s''$
%   then $t_1'=t_1''$ and $s'=s''$, which immediately proves this case.\\
%
%   \noindent\textbf{Case} $\userule{S-OrRight}$The \refrule{E-Or} rule gives us
%   that evaluation has been applied to $t_2$. We therefore can apply the
%   induction hypothesis, from which we obtain that $t_2',s'\evaluate t_2'',s''$
%   then $t_2'=t_2''$ and $s'=s''$, which immediately proves this case.\\
%
%   \noindent\textbf{Case} $\userule{S-OrNone}$ The \refrule{E-Or} rule gives us
%   that evaluation has been applied to both $t_1$ and $t_1$. We assumed that the
%   states do not change, so we know that $s=s'=s''$. We can therefore safely
%   apply the induction hypothesis twice, to obtain that
%   $t_1',s''\evaluate t_1'',s'''$ then $t_1'=t_1''$ and $s''=s'''$, and
%   $t_2',s'''\evaluate t_2'',s''''$ then $t_2'=t_2''$ and $s'''=s''''$. This,
%   together with the \refrule{E-Or} rule proves this case.\\
%
%   \noindent\textbf{Case} $\userule{S-Edit}$ by definiton, $\Edit v$ is a value
%   with respect to evaluation.\\
%
%   \noindent\textbf{Case} $\userule{S-Fill}$ by definition, $\Enter \tau$ is a
%   value with respect to evaluation.\\
%
%   \noindent\textbf{Case} $\userule{S-Update}$ by definiton, $\Update l$ is a
%   value with respect to evaluation.\\
%
%   \noindent\textbf{Case} $\userule{S-Fail}$ by definiton, $\Fail$ is a value
%   with respect to evaluation.\\
%
%   \noindent\textbf{Case} $\userule{S-Xor}$ by definition, $e_1\Xor e_2$ is a
%   value with respect to evaluation.\\
%
%   \noindent\textbf{Case} $\userule{S-Next}$ The \refrule{E-Next} rule gives us
%   that evaluation has been applied to $t_1$. We therefore can apply the
%   induction hypothesis, from which we obtain that $t_1',s'\evaluate t_1'',s''$
%   then $t_1'=t_1''$ and $s'=s''$. This, together with the \refrule{E-Next} rule,
%   proves this case.\\
%
%   \noindent\textbf{Case} $\userule{S-And}$ The \refrule{E-And} rule gives us
%   that evaluation has been applied to both $t_1$ and $t_1$. We assumed that the
%   states do not change, so we know that $s=s'=s''$. We can therefore safely
%   apply the induction hypothesis twice, to obtain that
%   $t_1',s''\evaluate t_1'',s'''$ then $t_1'=t_1''$ and $s''=s'''$, and
%   $t_2',s'''\evaluate t_2'',s''''$ then $t_2'=t_2''$ and $s'''=s''''$. This,
%   together with the \refrule{E-And} rule proves this case.\\
%
%   \noindent\textbf{Case} $\userule{S-Appoint}$ The \refrule{E-Appoint} rule
%   gives us that evaluation has been applied to $t$. We therefore can apply the
%   induction hypothesis, from which we obtain that $t',s'\evaluate t'',s''$ then
%   $t'=t''$ and $s'=s''$. This, together with the \refrule{E-Appoint} rule,
%   proves this case.
%
% \end{proof}



\subsection{Theorem~\ref{thm:failing}}

\begin{proof}

  We prove Theorem~\ref{thm:failing} by induction on $e'$.\\

\case{$e = \Fail$}
     {$\Failing(\Fail,s)=\True$, and there is no handling rule that applies to fail.}

\case{$e = \Edit v$}
       {$\Gamma,\Sigma\infers\Edit v:\Task\tau$, $\Failing(\Edit v,s)=\False$, and there exists an input $i$, namely $v':\tau$.}

\case{$e = \Enter \tau$}
       {$\Failing(\Enter \tau)=\False$, and there exists an in put i, namely $v:\tau$.}

\case{$e = \Update l$}
       {Given that $\Gamma,\Sigma\infers\Update l:\Task\tau$, $\Failing(\Update l,s)=\False$, and there exists an input $i$, namely $v:\tau$.}

\case{$e = t_1\Then e_2$}
       {$\Failing(t_1\Then e_2,s)=\Failing(t_1,s)$. If there exists an $i$ for $t_1$, then this $i$ also applies to $t_1\Then e_2$. This case therefore holds by the induction hypothesis.}

\case{$e = t_1\Next e_2$}
       {$\Failing(t_1\Next e_2,s)=\Failing(t_1,s)$. If there exists an $i$ for $t_1$, then this $i$ also applies to $t_1\Next e_2$. This case therefore holds by the induction hypothesis.}

\case{$e = e_1\Xor e_2$}
       {We normalise the two expressions first, $e_1,s\stride t_1,s'$, $e_2,s\stride t_2,s'$ and we can then be in two situations. One, we can have that $\Failing(t_1,s')$ and $\Failing(t_2,s')$ are both true. If that is so, then by definition, we have both $\Failing(e_1\Xor e_2,s)$ and no rule of the hanlding semantics applies, and therefore there exists no input for this case.\\
                                           Or we are in the situation where one or both of the two sub expressions does not fail. In that case, we know that $\Failing(e_1\Xor e_2,s)$ does not hold, and that at least one of the handling rules applies. Therefore, there must be an input $i$, namely $\Left$, $\Right$ or both.}

\case{$e = t_1\And t_2$}
       {We can again find ourselves in one of two situations. In the first case, both sub expressions fail, $\Failing(t_1,s)$ and $\Failing(t_2,s)$. In that case, we know that $\Failing(t_1\And t_2,s)$ also fails by definition. By the induction hypothesis, we know that for both $t_1$ and $t_2$ there is no input that can be handled. Since the only two rules for $\And$ that handle input just pass this input on to one of the two expressions, we know that indeed no $i$ applies.\\
                                           In the case that one or both sub expressions do not fail, then by definition $t_1\And t_2$ not failing under $s$. Again by induction hypothesis, we know that for one or both of the expressions, there exits an $i$ that can be handled. Then by H-FirstAnd and H-SecondAnd, we know that we can pass this $i$, by prefixing it with either $\First$ or $\Second$.}

\case{$e = t_1\Or t_2$}
       {We can again find ourselves in one of two situations. In the first case, both sub expressions fail, $\Failing(t_1,s)$ and $\Failing(t_2,s)$. In that case, we know that $\Failing(t_1\Or t_2,s)$ also fails by definition. By the induction hypothesis, we know that for both $t_1$ and $t_2$ there is no input that can be handled. Since the only two rules for $\Or$ that handle input just pass this input on to one of the two expressions, we know that indeed no $i$ applies.\\
                                           In the case that one or both sub expressions do not fail, then by definition $t_1\Or t_2$ not failing under $s$. Again by induction hypothesis, we know that for one or both of the expressions, there exits an $i$ that can be handled. Then by H-FirstOr and H-SecondOr, we know that we can pass this $i$, by prefixing it with either $\First$ or $\Second$.}

  % \noindent\textbf{Case} $e = u \At t$\\
  %  This case follows directly from applying the induction hypothesis, since $\Failing(u \At t,s)=\Failing(t,s)$ and any input that applies to $t$ also applies to $u \At t$.\\
\end{proof}



\subsection{Theorem~\ref{thm:safety-i}}

\begin{proof}

  \case{$e=\Edit v:\Task\tau, i= v':\tau$}
       {Given that $\userule{H-Change}$, we have by definition that $\Inputs(\Edit v:\Task\tau,s) = \set{v':\tau,\Empty}$, which includes $v':\tau$.}

  \case{$e=\Edit v:\Task\tau, i= \Empty$}
       {Given that $\userule{H-Empty}$, we have by definition that $\Inputs(\Edit v:\Task\tau,s) = \set{v':\tau,\Empty}$, which includes $\Empty$.}

  \case{$e=\Enter \tau, i= v':\tau$}
       {Given that $\userule{H-Fill}$, we have by definition that $\Inputs(\Enter \tau,s) = \set{v':\tau }$, which includes $v':\tau$.}

  \case{$e=\Update l:\Task\tau, i= v':\tau$}
       {Given that $\userule{H-Update}$, we have by definition that $\Inputs(\Update l:\Task\tau,s)= \set{v':\tau }$, which includes $v':\tau$.}

  \case{$e=t_1\Xor t_2 , i= \Left$}
       {Given that $\userule{H-PickLeft}$, we have by definition that $\Inputs(t_1\Xor t_2,s) = \set{ \Left,\Right}$, which includes $\Left$.}

  \case{$e=t_1\Xor t_2 , i= \Right$}
       {Given that $\userule{H-PickRight}$, we have by definition that $\Inputs(t_1\Xor t_2,s) = \set{ \Left,\Right}$, which includes $\Right$.}

  \case{$e=t_1\Next e_2 , i= \Continue $}
       {Given that $\userule{H-Next}$, we have by definition that $\Inputs(t_1\Next e_2,s) = \Inputs(t_1,s) \cup \set{\Continue\mid \Value{(t_1,s)} = v_1 \wedge \neg\Failing{(e_2 v_1,s\stride)}}$.
       If the H-Next rule applies, this means that the conditions $\Value{(t_1,s)} = v_1 \wedge \neg\Failing{(e_2 v_1,s\stride)}$ are fulfilled, and therefore $\Continue$ is contained.}

  \case{$e=t_1\Next e_2 , i\neq\Continue$}
       {Given that $\userule{H-PassNext}$, we have by definition that $\Inputs(t_1\Next e_2,s) = \Inputs(t_1,s) \cup \set{\Continue\mid \Value{(t_1,s)} = v_1 \wedge \neg\Failing{(e_2 v_1,s\stride)}}$.
       By the induction hypothesis, we have that $i\in \Inputs(t_1,s)$, and by definition of $\Inputs$, $i$ is therefore also included in this case.}

  \case{$e=t_1\Then e_2 , i$}
       {Given that $\userule{H-PassThen}$, we have by definition that $\Inputs(t_1\Then e_2,s)= \Inputs(t_1,s)$.
       By the induction hypothesis, we have that $i\in \Inputs(t_1,s)$, and by definition of $\Inputs$, $i$ is therefore also included in this case.}

  \case{$e=t_1\And t_2 , i=\First i$}
       {Given that $\userule{H-FirstAnd}$ we have by definition that $\Inputs(t_1\And t_2,s) = \set{\First\ i \mid i \in \Inputs(t_1,s)} \cup \set{\Second\ i \mid i \in \Inputs(t_2,s)}$.
       By the induction hypothesis, we have that $i\in\Inputs(t_1,s)$, and by definition of $\Inputs$, $\First i$ is therefore als included in this case.}

  \case{$e=t_1\And t_2 , i=\Second i$}
       {Given that $\userule{H-SecondAnd}$ we have by definition that $\Inputs(t_1\And t_2) = \set{\First\ i \mid i \in \Inputs(t_1,s)} \cup \set{\Second\ i \mid i \in \Inputs(t_2,s)}$.
       By the induction hypothesis, we have that $i\in\Inputs(t_2,s)$, and by definition of $\Inputs$, $\Second i$ is therefore als included in this case.}

  \case{$e=t_1\Or t_2 , i=\First i$}
       {Given that $\userule{H-FirstOr}$ we have by definition that $\Inputs(t_1\Or t_2,s) = \set{\First\ i \mid i \in \Inputs(t_1,s)} \cup \set{\Second\ i \mid i \in \Inputs(t_2,s)}$.
       By the induction hypothesis, we have that $i\in\Inputs(t_1,s)$, and by definition of $\Inputs$, $\First i$ is therefore als included in this case.}

  \case{$e=t_1\Or t_2 , i=\Second i$}
       {Given that $\userule{H-FirstOr}$ we have by definition that $\Inputs(t_1\Or t_2,s) = \set{\First\ i \mid i \in \Inputs(t_1,s)} \cup \set{\Second\ i \mid i \in \Inputs(t_2,s)}$.
       By the induction hypothesis, we have that $i\in\Inputs(t_2,s)$, and by definition of $\Inputs$, $\Second i$ is therefore als included in this case.}
  %
  % \noindent\textbf{Case} $e=u\At t, i$\\
  %  Given that $\userule{H-Appoint}$, we can directly apply the induction hypothesis to prove this case.\\
\end{proof}



\end{document}
