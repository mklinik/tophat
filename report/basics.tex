% !TEX root=../report.tex

\section{Our base language}

Let us start by defining our host language.
It is a simple $\lambda$-calculus.
Next to abstraction, application and variables,
it contains constants for booleans, integers and strings
and some basic operations on them.
We will embed our task language \emph{into} this basic language.
This means we can use all the power of the host language,
such as arithmetic operations and (higher order) functions,
in our task language.
Because all the constructs in our host language are standard,
we will not give typing rules and semantic rules here.
For this we refer to standard work like \textcite{books/Pierce02TAPL} and \textcite{books/Harper16PFPL}.


\subsection{Syntax}
\label{sec:syntax}

\begin{grammar}
  Expressions
    & e &::= & \lambda x:\tau.\ e & – abstraction \\
    &   &\mid& e_1\ e_2           & – application \\
    &   &\mid& x                  & – variable \\
    &   &\mid& c                  & – constant \\
    &   &\mid& e_1 \star e_2      & – operator \\
    &   &\mid& \If{e_1}{e_2}{e_3} & – branching \\
    &   &\mid& p                  & – pretask \\
  Constants
    & c &::= & B                  & – boolean \\
    &   &\mid& I                  & – integer \\
    &   &\mid& S                  & – string \\
  Pretasks
    & p &::= & \ldots             & (to be defined) \\
\end{grammar}
Expressions are typed abstractions, applications, variables or constants.
Constants can be booleans, integers or strings.
The $\star$ represents a general notion of operators.
With $\If{}{}{}$ one can do basic branching.
The syntactic category of pretasks will be defined in the upcoming sections.
Pretasks $p$ are tasks \emph{in making},
just as expressions are values in making.
A \emph{task} will be introduced in \autoref{sec:evaluation}.

We will use double quotation marks ($\str{}$) to denote strings.
Integers are denoted by their decimal representation,
and booleans are written $\True$ and $\False$.
In the remaining of this report,
we will freely make use of the logic operators $\Not$, $\land$, and $\lor$,
arithmetic operators $+$, $-$, $\times$,
and the string append operator $\pp$.
Also, we will use the standard comparison operations $<$, $\le$, $\equiv$, $\not\equiv$, $\ge$, and $>$
as builtins.
Only when referring to operators in general we will use $\star$.


\subsection{Typing}

\begin{grammar}
  Types
    & \tau &::= & \tau_1 \to \tau_2 & – function type \\
    &      &\mid& \beta             & – basic type \\
    &      &\mid& \Task \tau        & – task type \\
  Basic types
    &\beta &::= & \Bool             & – boolean type \\
    &      &\mid& \Int              & – integer type \\
    &      &\mid& \String           & – string type \\
\end{grammar}
Typing of our expressions $e$ is as to be expected,
and won't be given in this document.
There is one big difference with respect to standard work.
This is the addition of a type for tasks $\Task \tau$.
Typing rules are of the form $\boxed{\Gamma \infers e : \tau}$,
which we read as \enquote{in environment $\Gamma$, expression $e$ has type $\tau$}.


\subsection{Evaluation}
\label{sec:evaluation}

To express evaluation,
we use the big step semantics of our host language.
Turning an expression $e$ into a value $v$ is denoted by $\boxed{e \evaluate v}$.
The rules to evaluate expressions $e$ do not differ from standard work.

\begin{grammar}
  Values
    & v &::= & \lambda x.\ e  & – abstractions \\
    &   &\mid& c              & – constants \\
    &   &\mid& t              & – tasks \\
  Tasks
    & t &::= & \ldots         & – (to be defined for each $p$) \\
\end{grammar}
Regarding our base language, lambda abstractions are values, as are all constants.
Each \emph{pretask} $p$ we will encounter in the upcoming sections,
will be accompanied by a corresponding \emph{task} $t$.
Tasks are also values with respect to evaluation.
This means pretasks will be eagerly evaluated to tasks by the relation $\evaluate$.\footnote{
  The same applies to pairs,
  which we will add to our host language in \autoref{sec:pairing}.
}
This suggestion by \textcite[p. 323]{books/Harper16PFPL}
makes it easier to reason about data constructors.
