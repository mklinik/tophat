% !TEX root=../report.tex

\section{Case study}

There are three possible observations we can do on tasks:
\begin{enumerate}
  \item Does it have a value? ($\Value(t) \neq \bot$)
  \item Is it possible to send events to it? ($\Inputs(t) \neq \nothing$)
  \item Is it succeeding? ($\Failing(t) = \False$)
\end{enumerate}
Below a table of possible results when applying these observations to different tasks.

\begin{equation*}
  \begin{array}{llll}
    \toprule
                                                         & \Value           & \Inputs                                                    & \Failing \\
    \midrule
    \Fail                                                & \bot             & \nothing                                                   & \False \\
    \addlinespace
    \Edit v                                              & v                & \set{v', \Clear}                                           & \True \\
    \View v                                              & v                & \nothing                                                   & \True \\
    \Enter \tau                                          & \bot             & \set{v'}                                                   & \True \\
    \Let l = \Ref v \In \Update l                        & v                & \set{v'}                                                   & \True \\
    \Let l = \Ref v \In \Watch l                         & v                & \nothing                                                   & \True \\
    \addlinespace
    \Edit v \Then \lambda x. \Fail                       & \bot             & \set{v', \Clear}                                           & \True \\
    \View v \Then \lambda x. \Fail                       & \bot             & \nothing                                                   & \True \\
    \Enter \tau \Then \lambda x. \Fail                   & \bot             & \set{v'}                                                   & \True \\
    \Let l = \Ref v \In \Update l \Then \lambda x. \Fail & \bot             & \set{v'}                                                   & \True \\
    \Let l = \Ref v \In \Watch l \Then \lambda x. \Fail  & \bot             & \nothing                                                   & \True \\
    \addlinespace
    \Edit v_1 \And \Edit v_2                             & \tuple{v_1, v_2} & \set{\Left v_1', \Left \Clear, \Right v_2', \Right \Clear} & \True \\
    \View v_1 \And \Edit v_2                             & \tuple{v_1, v_2} & \set{\Right v_2', \Right \Clear}                           & \True \\
    \View v_1 \And \View v_2                             & \tuple{v_1, v_2} & \nothing                                                   & \True \\
    \Enter \tau_1 \And \Edit v_2                         & \bot             & \set{\Left v_1', \Right v_2', \Right \Clear}               & \True \\
    \Enter \tau_1 \And \View v_2                         & \bot             & \set{\Left v_1'}                                           & \True \\
    \Enter \tau_1 \And \Enter \tau_2                     & \bot             & \set{\Left v_1', \Right v_2'}                              & \True \\
    \Fail \And \Edit v_2                                 & \bot             & \set{\Right v_2', \Right \Clear}                           & \True \\
    \Fail \And \View v_2                                 & \bot             & \nothing                                                   & \True \\
    \Fail \And \Fail                                     & \bot             & \nothing                                                   & \False \\
    \addlinespace
    t_1                                                  & \bot             & \nothing                                                   & \True \\
    \bottomrule
  \end{array}
\end{equation*}

\begin{equation*}
  \begin{split}
    t_1 := & \Let l_1 = \Ref \False \In \\
           & \Let l_2 = \Ref \False \In \\
           & u_1 @\ (\Watch l_1 \Then \lambda b. \If{b}{\Update l_2}{\Fail}) \\
           & \quad \And \\
           & u_2 @\ (\Watch l_2 \Then \lambda b. \If{b}{\Update l_1}{\Fail})
  \end{split}
\end{equation*}

Conclusions:
\todo{
  What todo?
}
\begin{itemize}
  \item
    When we have always writeable editors, $\Inputs$ and $\Failing$ agree.
  \item
    Introducing readonly editors will make $\Inputs(t) = \nothing$ but keeps $\Failing = \True$.
  \item
    Taking $\Inputs$ as a predicate to step to a task is tempting,
    but as we just reasoned,
    we will not be able to step to a readonly editor\ldots
  \item
    On the other hand,
    taking $\Failing$ as a predicate to step to a task,
    will still prohibit to step to $\Fail$ or pairs thereof.
    However, it is possible to step to a locked task like $t_1$.
    It is not immediately observable as failing,
    but it clearly does not have any possible events it can handle.
\end{itemize}
