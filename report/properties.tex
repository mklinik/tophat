% !TEX root=../report.tex

\section{Properties}

\statefultrue

\subsection{Equality}

\begin{align*}
\forall t_1, t_2 .  t_1 \sim t_2 \iff &&\forall i . t_1 \handle{i} t_1' \implies& & t_2 \handle{i} t_2'\\
&&&\wedge & \Value{(t_1')} \equiv \Value{(t_2')}\\
&&&\wedge & t_1' \sim t_2'\\
&\wedge& \forall i . t_2 \handle{i} t_2' \implies& & t_1 \handle{i} t_1'\\
&&&\wedge & \Value{(t_1')} \equiv \Value{(t_2')}\\
&&&\wedge & t_1' \sim t_2'
\end{align*}

\subsection{Equivalence}

\begin{align*}
\forall t_1, t_2 .  t_1 \simeq t_2 \iff \forall i_1 .\exists i_2 . t_1 \handle{i_1} t_1'\implies& & t_2 \handle{i_2} t_2'\\
                                                    & \wedge & \Value{(t_1')} \equiv \Value{(t_2')}\\
                                                    & \wedge & t_1' \simeq t_2'\\
                                                    & \wedge & i_1 \bumpeq i_2\\
                                      \wedge\forall i_2.\exists i_1 . t_2 \handle{i_2} t_2'\implies && t_1 \handle{i_1} t_1'\\
                                      & \wedge & \Value{(t_1')} \equiv \Value{(t_2')}\\
                                      & \wedge & t_1' \simeq t_2'\\
                                      & \wedge & i_1 \bumpeq i_2\\
\end{align*}
\todo{What is missing still: $i_2$ can also be/contain empty/be a normalisation (???)}
\todo{Define the equivalence of i's $\bumpeq$}
\subsection{Progress}

\begin{align*}
  \forall e:\tau,i & e \handle{i} e' &\implies& \text{ either Fail}(e') \text{ or } \exists e'',i' \text{ such that } e'\handle{i'}e''&\\
\end{align*}

\subsection{Preservation}

\begin{theorem}[preservation under handling]
  $\forall e:\tau,i  e \handle{i} e' \implies e':\tau$
  \label{thmpreshandle}
\end{theorem}

\begin{theorem}[preservation under normalization]
    $\forall e:\tau  e \normalise e' \implies e':\tau$
    \label{thmpresnorm}
\end{theorem}

\begin{theorem}[preservation under evaluation]
      $\forall e:\tau  e \evaluate e' \implies e':\tau$
      \label{thmpreseval}
\end{theorem}

\begin{proof}
  We prove Theorem~\ref{thmpreseval} by induction on $e$:\\
  \noindent\textbf{Case} $e=\lambda x:\tau.e, e_1 e_2, x, c, l, e_1 \star e_2,
                            \If{e_1}{e_2}{e_3},\tuple{e_1, e_2},\unit,\Ref e,!e,
                            e_1 := e_2,e_1; e_2$
      preservation has been proven for these cases by \todo{insert cite}\\

  \noindent\textbf{Case} $\userule{E-Edit}$
      By T-Edit we have $\Edit e:\Task \tau$ and $e:\tau$. The induction
      hypothesis gives us that $e\evaluate v$ also preserves, and thus $v:\tau$.
      Therefore $\Edit v:\Task\tau$.\\

  \noindent\textbf{Case} $\userule{E-Fill}$
      Evaluation does not alter the expression, therefore this case holds tivially.\\

  \noindent\textbf{Case} $\userule{E-Update}$
      By T-Update we have $\Edit e:\Task \tau$ and $e:\Ref \tau$. The induction
      hypothesis gives us that $e\evaluate l$ also preserves, and thus $l:\Ref\tau$.
      Therefore $\Update l:\Task\tau$\\

  \noindent\textbf{Case} $\userule{E-Fail}$
      Evaluation does not alter the expression, therefore this case holds tivially.\\

  \noindent\textbf{Case} $\userule{E-Then}$
      Given that $e_1\Then e_2:\Task \tau$, T-Then gives us that $e_1:\Task\tau_1$
      and $e_2:\tau_1 \to \Task \tau$. By the induction hypothesis, we know that
      $e_1\evaluate t_1$ preserves and thus $t_1:\Task\tau_1$. Therefore
      $t_1\Then e_2:\Task\tau$.\\

  \noindent\textbf{Case} $\userule{E-Next}$
      Given that $e_1\Next e_2:\Task \tau$, T-Then gives us that $e_1:\Task\tau_1$
      and $e_2:\tau_1 \to \Task \tau$. By the induction hypothesis, we know that
      $e_1\evaluate t_1$ preserves and thus $t_1:\Task\tau_1$. Therefore
      $t_1\Next e_2:\Task\tau$.\\

  \noindent\textbf{Case} $\userule{E-And}$
      Given that $e_1\And e_2:\Task(\tau_1\times\tau_2)$, T-And gives us that
      $e_1:\Task\tau_1$ and $e_2:\Task\tau_2$. By the induction hypothesis, we
      know that both $e_1\evaluate t_1$ and $e_2\evaluate t_2$ preserve and thus
      $t_1:\Task\tau_1$ and $t_2:\Task\tau_2$. Therfore
      $t_1\And t_2:\Task(\tau_1\times\tau_2)$\\

  \noindent\textbf{Case} $\userule{E-Or}$
      Given that $e_1\Or e_2:\Task\tau$, T-Or gives us that $e_1:\Task\tau$ and
      $e_2:\Task\tau$. By the induction hypothesis, we have that both
      $e_1\evaluate t_1$ and $e_2\evaluate t_2$ preserve and thus $t_1:\Task\tau$
      and $t_2:\Task\tau$. Therefore $t_1\Or t_2:\Task\tau$.\\

  \noindent\textbf{Case} $\userule{E-Xor}$
      Evaluation does not alter the expression, therefore this case holds tivially.
\end{proof}

Before we can show that normalisation also preserves, we need to show that the following lemma holds:\\

\begin{lemma}[Task value preserves]
  $\forall e:\Task\tau . \Value{(e)}=v \implies v:\tau$
  \label{lemmavaluepreserves}
\end{lemma}
\begin{proof}
  \noindent\textbf{Case} $\Value{(\Edit v,s)}=v$ By T-Edit, if $\Edit v:\Task\tau$, then $v:\tau$.\\

  \noindent\textbf{Case} $\Value{(\Enter \tau,s)}=\bot$ Since this case does not lead to a value, the lemma holds trivially.\\

  \noindent\textbf{Case} $\Value{(\Update l,s)}=s(l)$ \todo{Which rule do we use here to show that s(l) has the same type as l???}\\

  \noindent\textbf{Case} $\Value{(\Fail)}=\bot$ Since this case does not lead to a value, the lemma holds trivially.\\

  \noindent\textbf{Case} $\Value{(t_1\Then e_2)}=\bot$ Since this case does not lead to a value, the lemma holds trivially.\\

  \noindent\textbf{Case} $\Value{(t_2\Next e_2)}=\bot$ Since this case does not lead to a value, the lemma holds trivially.\\

  \noindent\textbf{Case} $\Value{(t_1\And t_2)}=\tuple{v_1, v_2}, \Value{(t_1)}=v_1\wedge\Value{(t_2)}=v_2$ By T-And we have that $t_1\And t_2:\Task(\tau_1\times\tau_2)$ and $t_1:\tau_1$ and $t_2:\tau_2$. By the induction hypothesis, $ \Value{(t_1)}=v_1$ and $\Value{(t_2)}=v_2$ preserve, and thus $v_1:\tau_1$ and $v_2:\tau_2$. This gives us that $\tuple{v_1, v_2}:\Task(\tau_1\times\tau_2)$ \\

  \noindent\textbf{Case} $\Value{(t_1\And t_2)}=bot, \not\Value{(t_1)}=v_1\wedge\Value{(t_2)}=v_2$ Since this case does not lead to a value, the lemma holds trivially.\\

  \noindent\textbf{Case} $\Value{(t_1\Or t_2)}=v_1, \Value{(t_1)}=v_1$ By T-Or we have that $t_1\Or t_2:\Task\tau$, and $t_1:\Task\tau$ and $t_2:\Task\tau$. By the induction hypothesis, we have that $v_1:\tau$.\\

  \noindent\textbf{Case} $\Value{(t_1\Or t_2)}=v_2, \Value{(t_1)}=\bot\wedge\Value{(t_2)}=v_2$ By T-Or we have that $t_1\Or t_2:\Task\tau$, and $t_1:\Task\tau$ and $t_2:\Task\tau$. By the induction hypothesis, we have that $v_2:\tau$.\\

  \noindent\textbf{Case} $\Value{(t_1\Or t_2)}=\bot, \Value{(t_1)}=\bot\wedge\Value{(t_2)}=\bot$ Since this case does not lead to a value, the lemma holds trivially.\\

  \noindent\textbf{Case} $\Value{(t_1\Xor t_2)}=\bot$ Since this case does not lead to a value, the lemma holds trivially.\\
\end{proof}

\begin{proof}
  We prove Theorem~\ref{thmpresnorm} by induction on $e$:\\

  \noindent\textbf{Case} $\userule{N-Fail}$ Since this case does not alter the
  expression, the theorem holds trivially.\\

  \noindent\textbf{Case} $\userule{N-Xor}$ Since this case does not alter the
  expression, the theorem holds trivially.\\

  \noindent\textbf{Case} $\userule{N-Update}$ Since this case does not alter
  the expression, the theorem holds trivially.\\

  \noindent\textbf{Case} $\userule{N-Fill}$ Since this case does not alter the
  expression, the theorem holds trivially.\\

  \noindent\textbf{Case} $\userule{N-Edit}$ Since this case does not alter the
  expression, the theorem holds trivially.\\

  \noindent\textbf{Case} $\userule{N-And}$ Given that $t_1\And t_2:\Task(\tau_1\times\tau_2)$, by T-And we have $t_1:\tau_1$ and $t_2:\tau_2$. By the induction hypothesis, we also have $t_1':\tau_1$ and $t_2':\tau_2$. This gives us that $t_1'\And t_2':\Task(\tau_1\times\tau_2)$.\\

  \noindent\textbf{Case} $\userule{N-Next}$ Given that $e_1\Next e_2:\Task \tau$, T-Then gives us that $t_1:\Task\tau_1$
  and $e_2:\tau_1 \to \Task \tau$. By the induction hypothesis, we know that
  $t_1\normalise t_1'$ preserves and thus $t_1':\Task\tau_1$. Therefore
  $t_1'\Next e_2:\Task\tau$.\\\\

  \noindent\textbf{Case} $\userule{N-OrLeft}$ Given that $t_1\Or t_2:\Task\tau$,
  by T-Or we have $t_1:\Task\tau$. By the induction hypothesis, we know that
  $t_1\normalise t_1'$ preserves and thus $t_1':\Task\tau$.\\

  \noindent\textbf{Case} $\userule{N-OrRight}$ Given that $t_1\Or t_2:\Task\tau$,
  by T-Or we have $t_2:\Task\tau$. By the induction hypothesis, we know that
  $t_2\normalise t_2'$ preserves and thus $t_2':\Task\tau$. \todo{there might be
  more to be said here. t1 obviously can to stuff too, but we don't say anyting about this.}\\

  \noindent\textbf{Case} $\userule{N-OrNone}$ Given that $t_1\Or t_2:\Task\tau$,
  by T-Or we have $t_1:\Task\tau$ and $t_2:\Task\tau$. By the induction hypothesis,
  we know that $t_1\normalise t_1'$ and $t_2\normalise t_2'$ preserve, and thus
  $t_1'\Or t_2':\Task\tau$.\\

  \noindent\textbf{Case} $\userule{N-ThenStay}$ Given that $t_1\Then e_2:\Task\tau$,
  by T-Then we have $t_1:\Task\tau_1$ and $e_2:\tau_1\to\Task\tau$. By the induction
  hypothesis, we know that $t_1\normalise t_1'$ preserves, and thus $t_1'\Then e_2:\Task\tau$.\\

  \noindent\textbf{Case} $\userule{N-ThenFail}$ Given that $t_1\Then e_2:\Task\tau$,
  by T-Then we have $t_1:\Task\tau_1$ and $e_2:\tau_1\to\Task\tau$. By the induction
  hypothesis, we know that $t_1\normalise t_1'$ preserves, and thus $t_1'\Then e_2:\Task\tau$.\\

  \noindent\textbf{Case} $\userule{N-ThenCont}$Given that $t_1\Then e_2:\Task\tau$,
  by T-Then we have $t_1:\Task\tau_1$ and $e_2:\tau_1\to\Task\tau$. By the induction
  hypothesis, we know that $t_1\normalise t_1'$ preserves. By
  Lemma~\ref{lemmavaluepreserves}, we know that $\Value{(t_1')}=v_1$ preserves.
  By Theorem~\ref{thmpreseval} we know that $e_2 v_1\evaluate t_2$ preserves. And
  finally by the induction hypothesis, we know that $t_2\normalise t_2'$ preserves.
  Therefore $t_2':\Task\tau$.\\

\end{proof}

\begin{proof}
  We prove Theorem~\ref{thmpreshandle} by induction on $e$:

  \noindent\textbf{Case} $\userule{H-Change}$ Given that $\Edit v:\Task\beta$ and $v,v':\beta$, we have that $\Edit v':\Task\beta$\\

  \noindent\textbf{Case} $\userule{H-Empty}$ Given that $\Edit v:\Task\beta$, then by T-Fill we have $\Fill \beta:\Task\beta$ \\

  \noindent\textbf{Case} $\userule{H-Fill}$ \\

  \noindent\textbf{Case} $\userule{H-Up}$ \\

  \noindent\textbf{Case} $\userule{H-PickLeft}$ \\

  \noindent\textbf{Case} $\userule{H-PickRight}$ \\

  \noindent\textbf{Case} $\userule{H-Next}$ \\

  \noindent\textbf{Case} $\userule{H-PassThen}$ \\

  \noindent\textbf{Case} $\userule{H-PassNext}$ \\

  \noindent\textbf{Case} $\userule{H-FirstAnd}$ \\

  \noindent\textbf{Case} $\userule{H-SecondAnd}$ \\

  \noindent\textbf{Case} $\userule{H-FirstOr}$ \\

  \noindent\textbf{Case} $\userule{H-SecondOr}$
\end{proof}
\subsection{???}

\begin{align*}
  \forall e:\tau & e \normalise e' &\implies& \text{ either Fail}(e') \text{ or } \exists e'',i' \text{ such that } e'\handle{i'}e''&\\
  \forall e:\tau, i & e\handle{i}e' &\implies& i\in\Inputs{(e)}
\end{align*}

\begin{theorem}
  \forall e:\tau & i\in\Inputs{(e)} &\implies& e\handle{i}e'
\end{theorem}

\begin{proof}
  \noindent\textbf{Case} $\Edit v:\Task\tau$ By definition, $\Inputs{(\Edit v)}=\{v',\Empty\}$. For $i=v'$, we have that $\userule{H-Change}$ handles the input, and for $i=\Empty$, we have $\userule{H-Empty}$.\\

  \noindent\textbf{Case} $\Enter \tau$ By definition, $\Inputs{(\Enter \tau)}=\{v':\tau\}$. For $i=v'$, we have that $\userule{H-Fill}$ handles the input.\\

  \noindent\textbf{Case} $\Update l:\Task\tau$ By definition, $\Inputs{(\Update\tau)}=\{v'\}$. For $i=v'$, we have that $\userule{H-Update}$ handles the input.\\

  \noindent\textbf{Case} $\Fail$ By definition, $\Inputs{(\Fail)}=\{\}$. The theorem holds trivially.\\

  \noindent\textbf{Case} $t_1\Then e_2$ By definition, $\Inputs{(t_1\Then e_2)}=\Inputs{(t_1)}$. By the induction hypothesis, we have that $i\in\Inputs{(t_1)}$, then $t_1\handle{i}t_1'$. Therefore, we also have $\userule{H-PassThen}$.\\

  \noindent\textbf{Case} $t_1\Next e_2$ By definiton, $\Inputs(t_1 \Next e_2) = \Inputs(t_1) \cup \set{\Continue\mid \Value{(t_1)}\neq \bot \wedge \neg\Failing{(e_2 \Value{(t_1)}\normalise)}}$. If $i=\Continue$, then we know that $\Value{(t_1)}\neq \bot \wedge \neg\Failing{(e_2 \Value{(t_1)}\normalise)}$ holds, and therefore H-Next applies. Otherwise, we have by induction hypothesis that $i\in\Inputs{(t_1)}$, then $t_1\handle{i}t_1'$. Therefore, we also have $\userule{H-PassNext}$.\\

  \noindent\textbf{Case} $t_1\And t_2$ By definiiton, $\Inputs(t_1\And t_2) = \set{\First\ i \mid i \in \Inputs(t_1)} \cup \set{\Second\ i \mid i \in \Inputs(t_2)}$. By the induction hypothesis, we have both that if $i\in\Inputs(t_1)$ then $t_1\handle{i}t_1'$ and that if $i\in\Inputs(t_2)$ then $t_2\handle{i}t_2'$. Therefore, if $i\in\Inputs(t_1\And t_2)$, then either $\userule{H-FirstAnd}$ or $\userule{H-SecondAnd}$.\\

  \noindent\textbf{Case} $t_1\Or t_2$ By definition, $\Inputs(t_1\Or t_2) = \set{\First\ i \mid i \in \Inputs(t_1)} \cup \set{\Second\ i \mid i \in \Inputs(t_2)}$. By the induction hypothesis, we have both that if $i\in\Inputs(t_1)$ then $t_1\handle{i}t_1'$ and that if $i\in\Inputs(t_2)$ then $t_2\handle{i}t_2'$. Therefore, if $i\in\Inputs(t_1\Or t_2)$, then either $\userule{H-FirstOr}$ or $\userule{H-SecondOr}$.\\

  \noindent\textbf{Case} $t_1\Xor t_2$ By definition, $\Inputs(t_1\Xor t_2) = \set{\Pick \Left, \Pick \Right}$. In case $i=\Pick\Left$, H-PickLeft applies, in case $i=\Pick\Right$, H-PickRight applies.

\end{proof}
