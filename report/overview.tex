% !TEX root=../report.tex

\section{An overview}

\statefultrue


\subsection{Expressions}

\begin{grammar}
  Expressions
    & e    &::= & \lambda x:\tau.\ e   & – abstraction \\
    &      &\mid& e_1\ e_2             & – application \\
    &      &\mid& x                    & – variable \\
    &      &\mid& c                    & – constant \\
    &      &\mid& e_1 \star e_2        & – operator \\
    &      &\mid& \If{e_1}{e_2}{e_3}   & – branching \\
    &      &\mid& \{e_1, e_2\}         & – pair \\
    &      &\mid& \{\}                 & – unit \\
    &      &\mid& e_1; e_2             & – sequence \\
    &      &\mid& \Ref e               & – reference \\
    &      &\mid& !e                   & – dereference \\
    &      &\mid& e_1 := e_2           & – assign \\
    &      &\mid& l                    & – location \\
    &      &\mid& p                    & – pretask \\
  Constants
    & c    &::= & B                    & – boolean \\
    &      &\mid& I                    & – integer \\
    &      &\mid& S                    & – string \\
  Pretasks
    & p    &::= & \Edit e              & – valued editor \\
    &      &\mid& \Fill \beta          & – unvalued editor \\
    &      &\mid& \Store e             & – stored editor \\
    &      &\mid& \Fail                & – fail task \\
    &      &\mid& e_1 \Then e_2        & – step \\
    &      &\mid& e_1 \Next e_2        & – user step \\
    &      &\mid& e_1 \And e_2         & – composition \\
    &      &\mid& e_1 \Or e_2          & – choice \\
    &      &\mid& e_1 \Xor e_2         & – user choice \\
\end{grammar}



\subsection{Events and actions}

\begin{grammar}
  Events
    & h    & ::=& a                    & – action \\
    &      &\mid& \First h             & – pass to first \\
    &      &\mid& \Second h            & – pass to second \\
  Actions
    & a    & ::=& v                    & – change editor to value \\
    &      &\mid& \Clear               & – clear an editor \\
    &      &\mid& \Continue            & – continue with next task \\
    &      &\mid& \Pick r              & – pick route \\
  Routes
    & r    & ::=& \Left r              & – go left \\
    &      &\mid& \Right r             & – go right \\
    &      &\mid& \Here                & – stay here \\
\end{grammar}



\subsection{Types}

\begin{grammar}
  Types
    & \tau &::= & \tau_1 \to \tau_2    & – function type \\
    &      &\mid& \tau_1 \times \tau_2 & – product type \\
    &      &\mid& \Unit                & – unit type \\
    &      &\mid& \Reference \tau      & – reference type \\
    &      &\mid& \Task \tau           & – task type \\
    &      &\mid& \beta                & – basic type \\
    &      &\mid& \alpha               & – type variable \\
  Basic types
    &\beta &::= & \Bool                & – boolean type \\
    &      &\mid& \Int                 & – integer type \\
    &      &\mid& \String              & – string type \\
\end{grammar}

\begin{equation*}
  \boxed{\RelationT}
\end{equation*}

\begin{mathpar}
  \userule{T-Edit} \qquad \userule{T-Fill} \qquad \userule{T-Store} \\
  \userule{T-Fail} \\
  \userule{T-Then} \qquad \userule{T-Next} \\
  \userule{T-And} \\
  \userule{T-Or} \qquad \userule{T-Xor}
\end{mathpar}



\subsection{Evaluation}

\begin{grammar}
  Values
    & v    &::= & \lambda x.\ e        & – abstraction \\
    &      &\mid& \{v_1, v_2\}         & – pair value \\
    &      &\mid& \{\}                 & – unit \\
    &      &\mid& c                    & – constant \\
    &      &\mid& l                    & – location \\
    &      &\mid& t                    & – task \\
  Tasks
    & t    &::= & \Edit v              & – valued editor \\
    &      &\mid& \Fill \beta          & – unvalued editor \\
    &      &\mid& \Store l             & – stored editor \\
    &      &\mid& \Fail                & – fail task \\
    &      &\mid& t_1 \Then e_2        & – step \\
    &      &\mid& t_1 \Next e_2        & – user step \\
    &      &\mid& t_1 \And t_2         & – composition \\
    &      &\mid& t_1 \Or t_2          & – choice \\
    &      &\mid& t_1 \Xor t_2         & – user choice \\
\end{grammar}

Notes:
\begin{itemize}
  \item $\Then$ and $\Next$ are strict in their first argument, lazy in their second.
    It doesn't matter what the continuation is.
  \item $\And$ and $\Or$ is strict in both arguments.
    \todo{Explain why?}
  \item $\Xor$ is lazy in both arguments.
    Evaluating one or both options of $e_1 \Xor e_2$ before the user makes a choice will result in untimely reference updates.
  % \item Although internal steps should be evaluated without user interaction,
  %   $t \Then e$ \emph{is} a task value, iff $\Value(t) = \nothing$.
  %   Take for example $t = \Fill \Int$.
\end{itemize}

\begin{equation*}
  \boxed{\RelationV}
\end{equation*}

\begin{mathpar}
  \userule{E-Edit} \qquad \userule{E-Fill} \qquad \userule{E-Store} \\
  \userule{E-Fail} \\
  \userule{E-Then} \qquad \userule{E-Next} \\
  \userule{E-And} \\
  \userule{E-Or} \qquad \userule{E-Xor}
\end{mathpar}
(Standard constructs as defined by \textcite{books/Pierce02TAPL})


\endinput







\subsection{Task observations}

\begin{equation*}
  \begin{array}{@{}lcl}
    \multicolumn{3}{c}{\boxed{\Value : t \times s \to v^?}} \\
    \Value(\Edit v\st{s})      &=& v \\
    \Value(\Change l\st{s})     &=& s(l) \\
    \Value(t_1 \And t_2\st{s}) &=& \{\Value(t_1\st{s}), \Value(t_2\st{s})\} \\
    \Value(\_)                 &=& \nothing
  \end{array}
\end{equation*}


\todo{Is $\Succeeding$ preserved by $\normalise$?}

\newpage
\begin{fullwidth}
\renewcommand*{\arraystretch}{4}


\subsection{Task normalisation}

\begin{equation*}
  \boxed{\RelationN}
\end{equation*}


\paragraph{Simplifying}
\todo{Is \refrule{N-And} needed or can we just use $\Value(t)$ to check if we have a pair?}

\begin{mathpar}
  \divert{\userule{N-Then}} \\
  \userule{N-And}
\end{mathpar}


\paragraph{Stepping}

\begin{mathpar}
  \begin{array}{@{}ll}
    \userule{N-WhenStay} & \userule{N-WhenFail} \\
                         & \userule{N-WhenNext}
  \end{array}
\end{mathpar}


\paragraph{Evaluation}

(Complementing above conditional rules)

\begin{mathpar}
  \divert{\userule{N-ThenEval}} \\
  \userule{N-AndEval} \qquad \userule{N-OrEval}
\end{mathpar}


\paragraph{Values}

\begin{mathpar}
  \userule{N-Pure} \qquad \userule{N-Fail} \\
  \userule{N-Edit} \qquad \userule{N-Fill} \qquad \userule{N-Watch}
\end{mathpar}


\newpage
\subsection{Event handling}

\begin{equation*}
  \boxed{\RelationH}
\end{equation*}


\paragraph{Editors and watches}

\begin{mathpar}
  \userule{H-Change} \qquad \userule{H-Enter} \qquad \userule{H-Clear} \\
  \userule{H-Store}
\end{mathpar}


\paragraph{Stepping}

(This is the ugly part\ldots)

\begin{mathpar}
  \begin{array}{@{}ll}
    \divert{\userule{H-Stay'}} & \divert{\userule{H-Next'}} \\
    \userule{H-Stay}           & \userule{H-Next} \\
                               & \userule{H-Fail}
  \end{array}
\end{mathpar}


\paragraph{Choosing}

\begin{mathpar}
  \userule{H-First} \qquad \userule{H-Second} \qquad \userule{H-Other}
\end{mathpar}


\paragraph{Passing}

(We use $\AndOr \in \{\And, \ExOr\}$.
Note that \refrule{H-Pass} is the place where we reuse $\normalise$.)

\begin{mathpar}
  \divert{\userule{H-PassS}} \qquad \userule{H-Pass} \\
  \userule{H-Left} \qquad \userule{H-Right}
\end{mathpar}


\paragraph{Fallback}

(Only applicable when no other rules apply!)

\begin{mathpar}
  \userule{H-Fallback}
\end{mathpar}

\end{fullwidth}


\newpage
