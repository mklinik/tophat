% !TEX root=../report.tex

\section{Black holes and values}

We have seen editors,
which are key to user interaction.
Before we continue with constructs to combine tasks,
we first need to introduce the other leaf task in our language:
the \emph{failure task} $\Fail$.

\begin{grammar}
  Pretasks
    & p & ::=& \ldots & \\
    &   &\mid& \Fail  & – fail \\
\end{grammar}
Conceptually,
the failure task is a task which \emph{never ends} and \emph{never has a value}.
Its type is therefore polymorphic.
\begin{equation*}
  \userule{T-Fail}
\end{equation*}
For this,
we need to extend our type syntax with polymorphic types.
\begin{grammar}
  Types
    & \tau &::= & \ldots             & \\
    &      &\mid& \alpha             & – type variable \\
\end{grammar}

The failure task evaluates to itself.
\begin{equation*}
  \userule{E-Fail}
\end{equation*}
Therefore,
it is a value with respect to evaluation and thus a syntactic task.
\begin{grammar}
  Tasks
    & t & ::=& \ldots & \\
    &   &\mid& \Fail  & – fail \\
\end{grammar}

Every event $h$ users send to the failure task will be swallowed
and no answer will ever be given.
\begin{equation*}
  \userule{H-Fail}
\end{equation*}
This is why we say that the failure task is a \emph{black hole}.
You can throw anything in it,
but nothing will ever come out!


\subsection{Failing and succeeding}
\label{sec:succeeding}

So, what is the usage of this simple and dull addition to our task language?
Think of $\Fail$ as the error value \type{Nothing} from Haskell's \type{Maybe} data type
or \type{None} from ML's \type{option} lifted into the task world.
Note the difference between $\Fail$ and $\Fill\beta$.
The first is the failing task,
the second an unvalued editor waiting for users to fill in a value.
Editors are tasks that clearly have a future and thus are \emph{succeeding}.

We will need to distinguish tasks which succeed and tasks which fail in the remaining of this report.
To do this,
we introduce the predicate $\boxed{\Succeeding(t)}$,
which tells us if a task is succeeding or not.
\begin{equation*}
  \begin{array}{lcl}
    \multicolumn{3}{l}{\Succeeding : \mathrm{Tasks} \to \mathrm{Booleans}} \\
    \Succeeding(\Edit v)    &=& \True \\
    \Succeeding(\Fill \tau) &=& \True \\
    \Succeeding(\Fail)      &=& \False
  \end{array}
\end{equation*}
For now,
only $\Succeeding(\Fail)$ does not hold.
We will step-by-step extend this definition in future sections.


\subsection{The value of a task}
\label{sec:value}

Above, we stated that the failure task \emph{will never have a value}.
But we never defined what it means for a task to have a value\ldots\
Intuitively,
a valued editor obviously has a value,
namely the one contained in it.
Along the same line,
an unvalued editor does not have a value
and the same holds for the failure task.

To make this intuition more formal,
we define the semantic function $\boxed{\Value(t)}$.
Given a task $t$, this function will answer the question:
\enquote{Does task $t$ have a value?}.
\begin{equation*}
  \begin{array}{lcl}
    \multicolumn{3}{l}{\Value : \mathrm{Tasks} \rightharpoonup \mathrm{Values}} \\
    \Value(\Edit v)    &=& v \\
    \Value(\Fill \tau) &=& \bot \\
    \Value(\Fail)      &=& \bot
  \end{array}
\end{equation*}
The harpoon in the type definition denotes that $\Value$ is a \emph{partial function}.
We use the symbol $\bot$ to denote an undefined mapping.
