% !TEX root=../report.tex

\section{Giving more choice}

\begin{grammar}
  Pretasks
    & p & ::=& \ldots       & \\
    &   &\mid& e_1 \Or e_2  & – task choosing \\
    &   &\mid& e_1 \Xor e_2 & – user choosing \\
  Tasks
    & t & ::=& \ldots       & \\
    &   &\mid& t_1 \Or t_2  & – task choosing \\
    &   &\mid& t_1 \Xor t_2 & – user choosing \\
  Actions
    & a & ::=& \ldots       & \\
    &   &\mid& \Pick r      & – pick route \\
  Routes
    & r & ::=& \Left r      & – go left \\
    &   &\mid& \Right r     & – go right \\
    &   &\mid& \Here        & – stay here \\
\end{grammar}

\begin{equation*}
  \userule{T-Or} \qquad \userule{T-Xor}
\end{equation*}

\begin{equation*}
  \userule{E-Or} \qquad \userule{E-Xor}
\end{equation*}

\begin{equation*}
  \userule{S-OrLeft} \qquad \userule{S-OrRight} \qquad \userule{S-OrNone}
\end{equation*}

\begin{equation*}
  \userule{S-Xor}
\end{equation*}

\begin{equation*}
  \begin{array}{lcl}
    \Value(t_1 \Or t_2) &=& \left\{
      \begin{array}{ll}
        v_1  & \when\ \Value(t_1) = v_1 \\
        v_2  & \when\ \Value(t_1) = \bot \lor \Value(t_2) = v_2 \\
        \bot & \otherwise
      \end{array}
    \right.
  \end{array}
\end{equation*}

\begin{equation*}
  \begin{array}{lcl}
    \Failing(t_1 \Or t_2) &=& \Failing(t_1) \lor \Failing(t_2)
  \end{array}
\end{equation*}

\begin{margintext}{Aside: Pairing in iTasks}
  $\Or$ is similar to \type{-||-}
  and $\Xor$ does not exist in iTasks.
\end{margintext}

\begin{equation*}
  \userule{H-FirstOr} \qquad \userule{H-SecondOr}
\end{equation*}

\begin{equation*}
  \userule{H-PickLeft} \qquad \userule{H-PickRight} \qquad \userule{H-PickHere}
\end{equation*}


\subsection{Three principles of choosing}


\subsection{Intermezzo: alternative functor and its laws}

\begin{equation*}
  \begin{array}{rclr}
    \Fail \Or t
      &\approxeq& t
      & \text{(left identity)} \\
    t \Or \Fail
      &\approxeq& t
      & \text{(right identity)} \\
    r \Or\ (s \Or t)
      &\approxeq& (r \Or s) \Or t
      & \text{(associativity)} \\
    & & & \\
    \Fail \Then g
      &\approxeq& \Fail
      & \text{(left absorption)} \\
    \Edit x \Or t
      &\approxeq& \Edit x
      & \text{(left catch)}
  \end{array}
\end{equation*}



\subsection{Example: a vending machine}

Let us model a vending machine.
In a simple process algebra we can write
\begin{equation*}
  ?\text{2euro}; (!\text{tea}\ \oplus\ ?\text{1euro}; !\text{coffee}).
\end{equation*}
Which reads:
users should insert 2 Euros,
after which they need to make a choice between tea,
or entering another Euro which results in coffee.
So $?\text{2euro}$ and $?\text{1euro}$ mean an input of respectively 2 and 1 Euro,
and $!\text{tea}$ and $!\text{coffee}$ are the outputs.
We use $\oplus$ as an external choice operation.

We can model this vending machine using tasks as follows.
\begin{equation*}
  \begin{array}{l}
    \Enter \Euro \Then \lambda n.
    \IF n \equiv 2.00 \THEN \\
     \quad \Edit \str{tea}
      \Xor
        \Enter \Euro \Then \lambda m.
        \IF m \equiv 1.00 \THEN
          \Edit \str{coffee}
        \ELSE
          \Fail \\
    \ELSE \Fail
  \end{array}
\end{equation*}
Where we use a new type $\Euro$ to represent amounts of money.

We see that in the task world we reason with concrete \emph{values} of a specified \emph{type}.
Only when the user inserted exactly two euros,
we will step to the next task.
In all other cases,
the step will \emph{fail} and we stay where we are.
After the first step,
users need to make an \emph{explicit} choice by selecting tea or selecting the option to enter more coins.

The task world explicitly captures that users can, for example, insert five euros:
The step will not be taken.
Also, it implicitly captures that the user can get his money back.
This can be implemented by a button on the vending machine which sends the \emph{empty} event to the task.
Next to an input field in a graphical user interface,
a coin insertion component is another way to present the concept of an editor to an end user.


% \begin{TASK}
%   fill Euro >>= \ n.
%     if n == 2 then
%         edit "tea"
%       <?>
%         fill Euro >>= \ m.
%         if m == 1 then
%           edit "coffee"
%         else
%           fail
%     else
%       fail
% \end{TASK}
