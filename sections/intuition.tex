% !TEX root=../main.tex



\section{Intuition}
\label{sec:intuition}

This section gives an overview of the core concepts of task-oriented programming.



\paragraph{Tasks model collaboration}
%NOTE: Heb toch maar deze kop toegevoegd, onderstaande gaat er duidelijk over, en de sprong van het voorbeeld naar "Tasks are reusable" is anders abrupt!

The central objective of \TOP is to \emph{coordinate collaboration}.
The basic building blocks of \TOPHAT for expressing collaboration are task combinators.
They express ways in which people can work together.
Tasks can be executed after each other, at the same time, or conditionally.
This motivates the combinators step, parallel, and choice.

\begin{example}[Breakfast]
\label{exm:breakfast}

The following program shows the different collaboration operators in the setting of making breakfast.
Users have a choice ($\Xor$) whether they want tea or coffee.
They always get an egg.
The drink and the food are prepared in parallel ($\And$).
When both the drink and the food are prepared, users can step ($\Next$) to eating the result.

\lstset{emph={mkDrink,mkFood}}

% \begin{TASK}
%   let mkBrkfst : Task Drink -> Task Food -> Task Unit =
%     \makeDrink. \makeFood. makeDrink <&> makeFood >>? \<<d, f>>. eat d f
%   let main : Task Unit =
%     mkBrkfst (makeTea <?> makeCoffee) makeEgg
% \end{TASK}

\begin{TASK}
  let mkBrkfst : Task Drink -> Task Food -> Task <<Drink,Food>>
    = \mkDrink. \mkFood. mkDrink <&> mkFood in
  mkBrkfst (mkTea <?> mkCoffee) mkEgg >>? enjoyMorning
\end{TASK}

% \begin{TASK}
%   let mkBrkfst : Task Drink -> Task Food -> Task Breakfast =
%     \makeDrink. \makeFood. makeDrink <&> makeFood
%   let main : Task Unit =
%     mkBrkfst (makeTea <?> makeCoffee) makeEgg >>? \breakfast. eat breakfast
% \end{TASK}

The way the combinators are defined matches real life closely.
When we want to have breakfast, we have to complete several other tasks first before we can do so.
We decide what we want to have and then prepare it.
We can prepare the different items we have for breakfast in parallel, but not at the same time.
For example, it is impossible to scramble eggs, and put on the kettle for tea simultaneously.
Instead, what is meant by parallel is that \emph{the order in which we do tasks and the smaller tasks that they are composed of, does not matter}.
%XXX: Alejandro has placed a line and exclamation under this sentence. I'm not sure what to do here. --NN
% I think he thinks it is an essential part of understanding tasks. I emphasised it. --TS
Then finally, only when every item we want to have for breakfast is ready, can we sit down and enjoy it.
\end{example}



\paragraph{Tasks are reusable}

There are three ways in which tasks are modular.
First, larger tasks are composed of smaller ones.
Second, tasks are first-class, they can be arguments and results of functions.
Third, tasks can be values of other tasks.
These aspects make it possible for programmers to model custom collaboration patterns.
%
\Cref{exm:breakfast} demonstrates how tasks can be parameterised by other tasks: \TS{mkBrkfst} is a collaboration pattern that always works the same way, regardless of which food and drink are being prepared.



\paragraph{Tasks are driven by user input}

Input events drive evaluation of tasks.
When the system receives a valid event, it applies this event to the current task, which results in a new task.
In this way the system communicates with the environment.
Inputs are synchronous.
We assume systems react instantaneously to each of those events.

In \TOPHAT, \emph{editors} are the basic method of communication with the environment.
Editors are modelled after input widgets from graphical user interfaces.
There are different editors, denoted by different box symbols.
Take for example an editor holding the integer seven: $\Edit 7$.
Such an editor reacts to two kinds of events.
The first are change events, for example the values $42$ or $37$, which are of the same type.
The second is a clear event $\Empty$, which empties the editor.
This results is an empty editor $\Enter \Int$, annotated to only hold integers.

%XXX: Alejandro: unclear
%XXX: Alejandro: What is a value? What is a type?
% He has marked the above paragraph as being unclear. --NN
% Vraag is alleen wat dan precies... :-( Ik heb er een concreet voorbeeld van gemaakt met integers. --TS

The sole purpose of editors is to interact with users by remembering the last value that has been sent to them.
There are no output events.
As values of editors can be observed, for example by a user interface, editors facilitate both input and output.
An empty editor ($\Enter$) stands for a prompt to input data,
while a filled editor ($\Edit$) can be seen either as outputting a value, or as an input that comes with a default value.



\begin{example}[Vending machine]
\label{Vending machine}

This example demonstrates external communication and choice.
It is a vending machine that dispenses a biscuit for one coin and a chocolate bar for two coins.
\begin{TASK}
  let vend : Task Snack = enter Int >>? \n. if n == 1 then edit Biscuit
    else if n == 2 then edit ChocolateBar else fail
\end{TASK}
The editor $\Enter \Int$ asks the user to enter an amount of money.
This editor stands for a coin slot in a real machine that freely accepts and returns coins.
There is a continue button that is initially disabled, due to the fact that the left hand side of the step combinator has no value.
When the user has inserted exactly 1 or 2 coins, the continue button becomes enabled.
When the user presses the continue button, the machine dispenses either a biscuit or a chocolate bar, depending on the amount of money.
Snacks are modelled using a custom type.

\end{example}



\paragraph{Tasks can be observed}

Several observations can be made on tasks.
One of those is determining the value of a task.
%NOTE: Should we change the type of Value to Maybe v? Both Johan and Alejandro mention that it is not completely clear to them. --NN
% I don't think that we should use Maybe. We should use the term partial function and bottom. --MKL
% I agree! I made the examples here also concrete instead of introducing the value function. --TS
Not all tasks have a value, the empty editor for instance, which makes value observation partial.
I.e., the value of $\Edit 7$ is $7$, but the value of $\Enter \Int$ is $\bot$.

Another observation is the set of input events a task can react to.
% This is important for generating user interfaces.
For example, the task $\Edit 7$ can react to value events and empty events, as discussed before.

In order to render a task, we need to observe a task's user interface.
This is done compositionally.
User interfaces of combined tasks are composed of the user interfaces of the components.
For example, of two tasks combined with a step combinator, only the left hand side is rendered.
Two parallel tasks are rendered next to each other.
Combining this information with the task's value and possible inputs,
we can display the current state of the task,
together with buttons that show the actions a user can engage in.

The final observation is to determine whether a task results in a failure, denoted by $\Fail$.
The step combinator $\Next$ and the choice combinator $\Xor$ use this to prevent users from picking a failing task.



\paragraph{Tasks are never done}

Tasks never terminate, they always keep reacting to events.
Editors can always be changed or cleared, and step combinators move on to new tasks.

In a step $t \Next e$, the decision to move on from a task $t$ to its continuation $e$ is taken by $\Next$, not by~$t$.
The decision is based on a speculative evaluation of $e$.
The step combinator in $t \Next e$ passes the value $v$ of $t$ to the continuation $e$.
Steps act like $t$ as long as the step is guarded.
A step is guarded if either the left task has no value, or the speculative evaluation of $e$ applied to $v$ yields the failure task $\Fail$.
Once it becomes unguarded, the step continues as the result of $e\ v$.
Speculative evaluation is designed so that possible side effects are undone.
The task $t \Next e$ additionally requires a continue event $\Continue$ to proceed.
% Its cousin $\Then$ makes a step without user interaction.

Step combinators give rise to a form of internal communication.
They represent data flow that \emph{follows} control flow.


\paragraph{Tasks can share information}

The step combinator is one form of internal communication, where task values are passed to continuations.
Another form of internal communication is shared data.
Shared data enables data flow \emph{across} control flow, in particular between parallel tasks.
Shared data sources are assignable references whose changes are immediately visible to all tasks interested in them.
Users can not directly interact with shared data, a shared editor is required for that.
If $x$ is a reference of type $\tau$, then $\Update x$ is an editor whose value is that of $x$.

The semantics of \TOPHAT requires all updates to shared data and all enabled internal steps to be processed before any further communication with the environment can take place.



\begin{example}[Cigarette smokers]

The cigarette smokers problem \cite{books/Downey08LBOS} is a surprisingly tricky synchronisation problem.
We study it here because it demonstrates the capabilities of guarded steps.
% The problem requires waiting for two conditions, waking up only if both conditions are satisfied.
The problem is stated as follows.
In order to smoke a cigarette, three ingredients are required: tobacco, paper, and a match.
There are three smokers, each having one of the ingredients and requiring the other two.
There is an agent that randomly provides two of those.
The difficulty lies in the requirement that only the smoker may proceed whose missing ingredients are present.

Downey models availability of the ingredients with a semaphore for each ingredient.
The agent randomly signals two of the three.
The solution proposed by Downey involves an additional mutex, three additional semaphores, three additional threads called \emph{pushers}, and three regular Boolean variables.
The job of the pushers is to record availability of their ingredient in their Boolean variable, and check availability of other resources, waking the correct smoker when appropriate.

% The details of the solution are not important here.
What is important is that the implementation of what is essentially deadlock-free waiting for two semaphores requires a substantial amount of additional synchronisation, together with non-trivial conditional statements.
%
\TOPHAT allows a simple solution to this problem, using guarded steps.
Steps can be guarded with arbitrary expressions.
The parallel combinator can be used to watch two shared editors at the same time.
Let \TS{match}, \TS{paper}, and \TS{tobacco} be references to Booleans.
The smokers are defined as follows.
% The problem is then specified as follows.
\begin{TASK}
  let continue = \<<x,y>>. if x /\ y then smoke else fail in
  let tobaccoSmoker = (update match <&> update paper) >>? continue in
  let paperSmoker = (update tobacco <&> update match) >>? continue in
  let matchSmoker = (update tobacco <&> update paper) >>? continue in
  tobaccoSmoker <&> paperSmoker <&> matchSmoker
\end{TASK}
When the agent supplies two of the ingredients by setting the respective shares to \TS{True}, only the step of the smoker that waits for those becomes enabled.

\end{example}



\paragraph{Tasks are predictable}

Let $t_1$ and $t_2$ be tasks.
The parallel combination $t_1 \And t_2$ stands for two independent tasks carried out at the same time.
This operator introduces interleaving concurrency.
For the system it does not matter if the tasks are executed by two people actually in parallel, or by one person who switches between the tasks.
The inputs sent to the component tasks are interleaved into a serial stream, which is sent to the parallel combinator.
We assume that such a serialization is always possible.
The tasks are truly independent of each other, all interleavings are permitted.
The environment must prefix events to $t_1$ and $t_2$ respectively by $\First$ (first) and $\Second$ (second).
This unambiguously renames the inputs, removing any source of nondeterminism.

With concurrency comes the need for synchronisation, in situations where only some but not all interleavings are desired.
The basic method for synchronisation in \TOPHAT is built into the step combinator.
The task $t \Next e$ can only continue execution when two conditions are met:
Task $t$ must have a value $v$, and $e\ v$ must not evaluate to $\Fail$.
Programmers can encode arbitrary conditions in $e\ v$, which are evaluated atomically between interaction steps.
This allows a variety of synchronisation problems to be solved in an intuitive and straight-forward manner.

\Citet{books/Hoare85CSP} states that nondeterminism is only ever useful for \emph{specifying} systems, never for implementing them.
\TOPHAT is meant solely for implementation and does not have any form of nondeterminism.
Input events for parallel tasks are disambiguated, internal steps ($\Then$) have a well-defined evaluation order, and internal choice ($\Or$) is left-biased.

% \lstset{emph={text}}
% \begin{example}[Abstract submission]
% \label{exm:abstract}
%
% The task of writing an abstract for a conference is never done.
% Let \TS{abstract} be a reference to a string.
%
% % \begin{TASK}
% %   let submit : String -> Task Unit =
% %     \text. delay 30 (sendTo "icfp19@sigplan.org" text)
% %   let write : Task Unit =
% %     edit "Abstract: ..." >>= \abstract. if 500 <= words abstract <= 700 then submit abstract else fail
% % \end{TASK}
%
% \begin{TASK}
%   let submit : Ref String -> Task String = \text. sendTo "icfp19@sigplan.org" text in
%   update abstract <|> delay 30 (submit abstract)
% \end{TASK}
%
% The \TS{delay} task is specified in \cref{exm:delay} in \cref{sec:language}.
%
% \end{example}


\paragraph{Recap}

Collaboration in the real world consists of three aspects: communication, concurrency, and synchronisation.
These aspects are reflected in \TOP on a high level of abstraction, hiding the details of communication.
For example, the cigarette smokers communicate with each other, but the programs do not explicitly mention sending or receiving events.

By focusing on collaboration instead of communication, \TOP leads to specifications closer to real-world workflows which, at the same time, can be used to generate multi-user applications to support such workflows.
