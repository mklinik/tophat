% !TEX root=../icfp2019.tex



\section{Intuition}
\label{sec:intuition}

This section gives an overview of the core concepts of task oriented programming (\TOP).



\subsection{Tasks model real world collaborations between people and machines}

The central objective of \TOP is to \emph{coordinate collaboration}.
To this end, \TOPHAT has basic tasks called editors, with which users can
interact, and combinators to express ways in which people can work together.
Tasks can be executed after each other, at the same time, or
conditionally. This motivates the task combinators for steps, parallel
combination, and choice.
%One can say that tasks are a way to model \emph{workflows}.
% \TOPHAT is a language to specify workflows.



\subsection{Tasks are reusable}

Tasks are modular in the following ways.
First, larger tasks are composed from smaller ones.
Second, tasks are first-class, they can be arguments and results of functions.
Third, tasks can be values of other tasks.
These aspects make it possible for programmers to model custom collaboration patterns.

\begin{example}[Twice]

A higher-order function which executes a given task twice.
The task value of the first execution is ignored.

\fixme{Is that our official syntax?}
\begin{TASK}
  let twice : Task $\tau$ -> Task $\tau$ = \t : Task $\tau$. t >>? \x : $\tau$
\end{TASK}
\qed

\end{example}



\subsection{Tasks are driven by user input}


Input events drive evaluation of tasks.
When the system receives a valid event, it applies the event to the current task, which results in a new task.
In this way the system communicates with the environment.

In \TOPHAT, editors are the basic method of communication with the environment.
There are different editors, denoted by different box symbols.
An editor of the form $\Edit v$, where $v$ is of type $\tau$, reacts to two kinds of events:
First, change events $v$ with a value of type $\tau$ and second, clear events $\Empty$ which empty the editor.

The sole purpose of editors is to interact with the user by retaining the last value that has been sent to them.
There are no output events.
As values of editors can be observed, for example by a user interface, editors serve as facilities for both input and output.
An empty editor $\Enter \tau$ stands for a prompt to input data, while a filled editor can be seen either as outputting its value, or as an input that comes with a default value.

\begin{example}[Vending machine]
\label{Vending machine}
% The following \CSP program models a vending machine that dispenses a small biscuit for one coin and a large biscuit for two coins. \todo{remove CSP?}
% \begin{equation*}
%   \textit{coin} \to (\textit{small} \to \text{STOP} \mid \textit{coin} \to \textit{large} \to \text{STOP})
% \end{equation*}
% % An equivalent reactive system in \ESTEREL could be defined as\fixme{validate}
% % \begin{equation*}
%   % V_\text{\ESTEREL} = \text{await } \textit{coin}; ( \text{await } \textit{button}; \text{emit } \textit{small} \parallel \text{await } \textit{coin}; \text{await } \textit{button}; \text{emit } \textit{large})
% % \end{equation*}
% A vending machine in \TOPHAT that is close in spirit to the one above looks as follows.

This example demonstrates external communication and choice.
A vending machine in \TOPHAT that dispenses a small biscuit for one coin and a large biscuit for two coins can be specified as follows
\begin{TASK}
  let vend : Task String = enter Int >>? \n. if n == 1 then edit "small"
    else if n == 2 then edit "large"
    else fail
\end{TASK}
The task asks the user to enter an amount of money.
This editor stands for a coin slot in a real machine that freely accepts and returns coins.
There is a continue button that is initially disabled, due to the fact that the right hand side of the step combinator results in $\Fail$.
When the user has inserted exactly 1 or 2 coins, the continue button becomes enabled.
When the user presses the continue button, the machine dispenses either a large or a small biscuit, depending on the amount of money.

% This machine does not exhibit exactly the same behaviour as the ones in \CSP, nor is it intended to.
% Instead, it just demonstrates how the same choice
%%NOTE: Problem is already stated --TS
% of buying a small biscuit for one coin or a large biscuit for two coins
% can be expressed in \TOPHAT.
\end{example}



\subsection{Tasks can be observed}

Several observations can be made of a task. The most obvious and basic
observation is determining the value of a task. Not all tasks have a value
though, for example, the empty editor $\Enter \tau$ does not have a value.

We can also observe what potential events a task can take. This is essential in
generating a (graphical) user interface for tasks. For example, the task
$\Edit 5$ can take a new value as input, or it can be cleared, resulting in an
empty editor of type Int.

In order to complete the user interface, we also need to be able to render a
task. This is done in a modular and recursive fashion. Every combinator renders
their sub-tasks in a certain way. For example, when rendering a task that is
composed of a step combinator, we simply render the left side of it.

When we are able to render tasks, and we are able to list all possible inputs,
we can obviously construct a user interface, that displays the current state of
the system, together with buttons that show the potential actions a user can
apply.

Another useful observation is to determine whether a task results in $\Fail$.
This is used by the step combinator, as demonstrated in
Example~\ref{Vending machine}, and the choice combinator, to prevent users from
picking a task that fails.


\subsection{Tasks are never done}

%NOTE: depends on: editors
Users can continuously interact with tasks, they will never terminate. The event
send by a user leads to a new task that can always receive a new event.

The new task can either be the same task, but with a new value for the changed
editor, or the system can decide to move on to the next task.

The decision to move on to the next task is not taken by the task itself.
It is based on a calculation to determine the next tasks.
The step combinators in $t \Then e$ and $t \Next e$ pass the value of $t$ to the continuation $e$.
They both act like $t$ as long as the step is guarded.
Once it becomes unguarded, the task continues as $e\ v$, where $v$ is the observed value of $t$.
% The continuation $e$ is a function which calculates the next task or $\Fail$, signalling no continuation is present.
The task $t \Next e$ requires an input event $\Continue$ (Continue) in addition to the step being unguarded. % in order to step.

The step combinators give rise to a method of internal communication.
They represent data flow that \emph{follows} control flow.

% The intuition about sequential composition is to first do one thing, and once it is done, do some other thing.
% The notion of sequential composition exists in two forms: the external and the internal step.



\subsection{Tasks can share information}

We have already seen communication between tasks by means of the step
combinator. The resulting value from the left task is passed on to the task on
the right. The second method of internal communication between tasks is shared
data.
Shared data represents data flow \emph{across} control flow.
Shared data sources are assignable references, whose changes in value are immediately visible to all tasks interested in them.
The semantics of \TOPHAT requires all updates to shared data and all enabled internal steps to be processed before any further communication with the environment can take place.


\begin{example}[Cigarette smokers]

The cigarette smokers problem \cite{books/Downey08LBOS} is a surprisingly tricky synchronisation problem.
We study it here because it demonstrates the capabilities of guarded steps.
% The problem requires waiting for two conditions, waking up only if both conditions are satisfied.
The problem is stated as follows.
In order to smoke a cigarette, three ingredients are required: tobacco, paper, and a match.
There are three smokers, each having one of the ingredients and requiring the other two.
There is an agent that randomly provides two of the ingredients.

Downey models availability of the ingredients using a semaphore for each ingredient.
The agent randomly signals two of the three semaphores.
% The problem asks writing code for the smokers so that the smoker who requires the two signalled semaphores wakes up and proceeds with smoking.
%
% A naïve attempt where each smoker first waits for one and then the other semaphore can lead to deadlock.
% The following pseudo-code is a non-solution.
% Let \emph{paper}, \emph{tobacco} and \emph{match} be semaphores, initialised to 0.
% \begin{TASK}
%   tobaccoSmoker = wait(match); wait(paper); smoke();
%   paperSmoker = wait(tobacco); wait(match); smoke();
%   matchSmoker = wait(tobacco); wait(paper); smoke();
% \end{TASK}
% To see what can go wrong, imagine the three smokers running in parallel.
% Let the agent signal the two semaphores \emph{tobacco} and \emph{paper}.
% One of \emph{paperSmoker} and \emph{matchSmoker} can take their first step, but if \emph{paperSmoker} takes the step, the system deadlocks.
%
The solution proposed by Downey involves an additional mutex, three additional semaphores, three additional threads called \emph{pushers}, and three regular Boolean variables.
The job of the pushers is to record availability of their ingredient in their Boolean variable, and check availability of other resources, waking the correct smoker when appropriate.

The details of the solution are not important here.
What is important is that the implementation of what is essentially waiting for two semaphores requires a substantial amount of additional synchronisation, together with non-trivial conditional statements.

% A similar effort as in Downey's solution must be made in order to solve the problem using synchronised actions in \CSP.
% This is because in \CSP, processes can only synchronise on one action at a time.

\TOPHAT allows a simple solution to this problem, using guarded steps.
Steps can be guarded with arbitrary expressions, and the parallel combinator can be used to watch two shared editors ($\Update$) at the same time.
Let \TS{match}, \TS{paper}, and \TS{tobacco} be references to Booleans.
The smokers are defined as follows.
\begin{TASK}
  let continue = \<<x,y>>. if x /\ y then smoke else fail in
  let tobaccoSmoker = (update match <&> update paper) >>? continue in
  let paperSmoker = (update tobacco <&> update match) >>? continue in
  let matchSmoker = (update tobacco <&> update paper) >>? continue
\end{TASK}
When the agent supplies two of the ingredients by setting the respective shares to \TS{True}, only the step of the smoker that waits for those becomes enabled.

\end{example}



\subsection{Tasks are predictable}


Concurrency is the source of parallelism: two things are done at the same time.
%
Using tasks $t_1$ and $t_2$, the parallel combination $t_1 \And t_2$ stands for two independent tasks carried out at the same time.
For the system it does not matter if the tasks are executed by two people actually in parallel, or by one person who switches between the tasks.
All inputs sent to the combined task are interleaved into a serial stream of inputs.
The tasks are truly independent of each other, all interleavings are permitted.
Inputs to $t_1$ and $t_2$ must be prefixed by $\First$(irst) and $\Second$(econd) respectively.
This unambiguously renames the inputs, removing any source of nondeterminism.

The basic method for synchronisation in \TOP is built into the step combinator.
The task $t \Next e$ can only continue execution when two conditions are met:
Task $t$ must have a value $v$, and $e\ v$ must not evaluate to $\Fail$.
Programmers can encode arbitrary conditions in $e\ v$, which are evaluated atomically between interaction steps.
This allows a variety of synchronisation problems to be solved in an intuitive and straight-forward manner.

\TOPHAT is meant solely for implementation and does not have any form of nondeterminism.
Input events for parallel tasks are disambiguated, internal steps have a well-defined evaluation order, and internal choice is left-biased.


\subsection{Recap}

The focus on collaboration gives rise to concepts like communication, concurrency and synchronisation.
Using tasks, however, they are meant to be used on a higher level of abstraction, which allows concise declarative description of patterns of collaboration, without worrying about the implementation details of communication.
By focussing on collaboration, tasks and \TOPHAT lead to high-level specifications closer to real-world workflows which, at the same time, can be used to generate distributive, multi-user applications to support such workflows.

To employ these patterns of collaboration, users need to communicate.
This motivates different forms of communication: with the environment, along control flow, and across control flow.
