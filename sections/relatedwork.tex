% !TEX root=../icfp2019.tex



\section{Related work}
\label{sec:relatedwork}



\paragraph{iTasks}

As mentioned earlier, iTasks is an implementation of \TOP. iTasks has many
features, and its basic combinators are versatile and powerful. Simpler
combinators are implemented by restricting the powerful ones. This is useful for
everyday programming, where having lots of functionality at one's fingertips is
convenient and efficient.

There have been two previous papers that describe semantics of iTasks, by
\citet{conf/ifl/KoopmanPA08} and \citet{conf/ppdp/PlasmeijerLMAK12}.
Both give a different semantics in the form of minimal implementations of a
subset of the interface of iTasks. These semantics however do not make an
explicit distinction between the host language and task language and they do not
provide a formal semantics, which makes formal reasoning difficult.


% Our work differs in three major ways.
% First, we make an explicit distinction between the host language and the task
% language, to emphasise their boundaries.
% Second, we make use of separate semantic functions for querying the current task
% value, determining the continuation after an input event, and producing a user
% interface.
% In iTasks and its aforementioned semantics, these are integrated into the
% \CL{Task} datatype.
% Third, our system does not have the notion of task stability.
% When desired, stability can be introduced by the programmer, but the system
% itself makes no use of it.
% We argue that this does not impact the expressiveness of our language.

% mTasks is another implementation of TOP, designed for coordination of IoT devices. \alert{cite MSc Lubbers}
% mTasks has been used in a demonstrator for home automation. \alert{cite MSc Andrade}


\subsection{Worfklow modeling}

Much research has been done into workflow modeling. These works focus on
describing the collaboration between subsystems, rather than the communication
between them.
The systems described here follow a \emph{boxes and arrows} model of specifying workflows.
Control flow, represented by arrows, usually can go unrestricted from anywhere to anywhere else in a workflow.
We see \TOP as the functional programming of workflows, as opposed to this GOTO style.


\paragraph{Workflow patterns}

Workflow patterns are regarded as special kind of the design patterns in
software engineering. They identify recurring patterns in workflow systems, much
like the combinators defined by \TOPHAT. Work by Van der
Aalst~\cite{journals/dpd/AalstHKB03} defines a comprehensive list of these
pattens, and examines their availability in industry workflow software.
Workflow patterns are usually described in terms of control flow graphs, and no
formal specification is given, which makes comparison and formal reasoning more
difficult.

\paragraph{Workflow nets}

Workflow nets~\cite{journals/jcsc/Aalst98} allow for the modeling and analysis
of business processes. They are graphical in nature, and clearly display how
every component is related to each other. A downside of Workflow nets is that
they do not facilitate higher order constructs and that they are often not
directly executable.

\paragraph{YAWL}

This is not always the case however. A language based on workflow nets that is
actually directly executable is YAWL by van der
Aalst~\cite{DBLP:journals/is/AalstH05}. It facilitates modeling and execution of
dynamic workflows, with support for AND, OR and XOR workflow patterns. As
mentioned, YAWL programs consist of Workflow Nets, and are therefore programmed
visually.

\paragraph{BPEL}

BPEL~\cite{bpel} is another popular business process language. The standardized
language allows for the specification of actions within business processes,
using an XML format. Processes specified in BEPL are executable, just like YAWL.


\subsection{Related web services}


\paragraph{IFTTT}

If This Then That (IFTTT)~\cite{IFTTT} is a web service that allows users to write and
configure simple conditional statements called applets, that can interact with
web services and IoT devices. These applets can be seen as tiny workflows, that
facilitate collaboration between different services and platforms. The
complexity of the flow is very limited however, and is not geared towards humans
collaborating.

\paragraph{Google forms \& Wufoo}

Google forms~\cite{googleforms} and Wufoo~\cite{wufoo} are web based survey administration apps that allows
users to compose forms and gather data from them. They are comparable to TOP and
iTasks in that they provide an easy way for users to construct forms on the web.
However, Google forms and Wufoo do not allow users to define what the system
then should do with this information, and can therefore not be regarded as a
workflow or business process modeling systems.

\paragraph{Chorus} % (http://www.chorus-home.org/)

Chorus~\cite{chen2017chorus} is a visual programming environment for online
collaboration apps. The goal of Chorus is to allow users to program their own
application, though which a collective task can be coordinated. Users work on a
preset data-type that drives the behavior of the user defined applications.


\subsection{Functional Reactive Programming}

Functional Reactive Programming (FRP) is a paradigm to describe dynamic changes of values in a declarative way.
This is done by specifying networks of values, called behaviors, that can depend on each other and on external events.
Behaviors can change over time, or triggered by events.
When a behavior changes, all other behaviors that depend on it are updated automatically.
The underlying implementation that takes care of the updating usually can tie input devices, like mouse and keyboard, to event streams and behaviors to output facilities, like text fields.
This allows for declarative specifications of applications with user interfaces.

The idea of FRP was pioneered by \citet{conf/icfp/ElliottH97}, in the meantime there are many variants and implementations, with reactive-banana one of the most well-known \cite{reactive-banana}.

FRP and \TOP are different systems that have different goals in mind.
Whereas FRP expresses automatically updating data dependencies, \TOP expresses collaboration patterns.
\TOP has no notion of time.
Tasks can not change over time, while behaviors can.
The biggest conceptual difference between a workflow in \TOP and a data network in FRP is that an event to a task only causes updates up until the next step, while an event in FRP propagates through the whole network.

That being said, there are some concepts that are similar in \TOP and FRP.
The \emph{stepper} behavior for example is associated with an event and yields the value of the most recent event.
This is similar to editors in \TOP.
Furthermore, both systems can be used to declaratively program user interfaces.
