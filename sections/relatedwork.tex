% !TEX root=../main.tex



\section{Related work}
\label{sec:relatedwork}

The work presented in this paper lies on the boundary of many areas of study.
People have looked at the problem of how to model and coordinate collaboration from many different perspectives.
The following subsections give an overview of related work from the many different areas.


\subsection{Task-oriented programming implementations}

\paragraph{iTasks}
As mentioned earlier, iTasks is an implementation of \TOP.
iTasks has many features, and its basic combinators are versatile and powerful.
Simpler combinators are implemented by restricting the powerful ones.
This is useful for everyday programming, where having lots of functionality at one's fingertips is convenient and efficient.
\TOPHAT on the other hand does not include the many different variations of the step- and parallel combinators of \ITASKS.
To name two examples, the combinators \verb+(>>|)+ and \verb+(||-)+ are variations of step and parallel that ignore the value of the left task.

There have been two previous papers that describe semantics of iTasks, by \citet{conf/ifl/KoopmanPA08} and \citet{conf/ppdp/PlasmeijerLMAK12}.
Both give a different semantics in the form of minimal implementations of a subset of the interface of iTasks.
These semantics however do not make an explicit distinction between the host language and task language and they do not provide an explicit formal semantics.
Therefore, the do not lend itself well for formal reasoning.

% Our work differs in three major ways.
% First, we make an explicit distinction between the host language and the task
% language, to emphasise their boundaries.
% Second, we make use of separate semantic functions for querying the current task
% value, determining the continuation after an input event, and producing a user
% interface.
% In iTasks and its aforementioned semantics, these are integrated into the
% \CL{Task} datatype.
% Third, our system does not have the notion of task stability.
% When desired, stability can be introduced by the programmer, but the system
% itself makes no use of it.
% We argue that this does not impact the expressiveness of our language.



\paragraph{mTasks}

The \MTASKS framework \cite{koopman2018task} is an implementation of \TOP geared towards \IOT devices.
As \TOPHAT its basic combinators are a subset of \ITASKS.
They are similar to those of \TOPHAT.
However, on \IOT devices it is useful to continue running tasks endlessly,
which is done in \MTASKS using a \CL{forever} combinator.
This is currently not possible in \TOPHAT.

As for \ITASKS, there is currently no formal semantics for \MTASKS.



\subsection{Worfklow modelling}

Much research has been done into workflow modelling.
This work focusses on describing the collaboration between subsystems,
rather than the communication between them.
The systems described here follow a \emph{boxes and arrows} model of specifying workflows.
Control flow, represented by arrows, usually can go unrestricted from anywhere to anywhere else in a workflow.
We see \TOP as the functional programming of workflows, as opposed to this \emph{goto}-style.

\paragraph{Workflow patterns}

Workflow patterns are regarded as special kind of the design patterns in
software engineering. They identify recurring patterns in workflow systems, much
like the combinators defined by \TOPHAT. Work by \citet{journals/dpd/AalstHKB03}
defines a comprehensive list of these
pattens, and examines their availability in industry workflow software.
Workflow patterns are usually described in terms of control flow graphs, and no
formal specification is given, which makes comparison and formal reasoning more
difficult.

\paragraph{Workflow Nets \& YAWL}

Workflow Nets~(\WFN) \cite{journals/jcsc/Aalst98} allow for the modelling and analysis of business processes.
They are graphical in nature, and clearly display how every component is related to each other.
A downside of \WFN is that they do not facilitate higher order constructs.
Also, they are often not directly executable.

A language based on \WFN that is actually directly executable is \YAWL by \citet{DBLP:journals/is/AalstH05}.
It facilitates modelling and execution of dynamic workflows, with support for \emph{and}, \emph{or} and \emph{xor} workflow patterns.
As mentioned, \YAWL programs consist of \WFN, and are therefore programmed visually.

\paragraph{BPEL}

\BPEL~\cite{bpel} is another popular business process calculus. The standardised
language allows for the specification of actions within business processes,
using an \XML format.
The language is mainly used for coordinating web services.
Two workflow patterns are supported; execution of services can be done sequential or in parallel.
On top of that, processes can be guarded by conditionals.
There is no support for higher order processes however.
Processes described in \BPEL can be regarded as activity graphs, and they can also be rendered as such.
The specified processes in \BPEL are directly executable, just like \YAWL.



\subsection{Process algebras}

\paragraph{Differences}
There are two main differences between \TOP and process algebras.
The first is a difference in scope.
Process algebras focus on modelling the input/output behaviour of processes, by explicitly stating which actions are sent and received at certain points in the program.
The goal of process algebras is formal reasoning about the interaction between processes.
Typically, one wishes to prove properties such as deadlock-freedom, liveness, or adherence to a protocol specification.

The focus of \TOP on the other hand is to model collaboration patterns, with the explicit goal of not having to specify how exactly subtasks communicate.
The declarative specification of data dependencies between subtasks enables \TOP to hide such details.

The second difference concerns internal communication.
There are two forms of communication between tasks: Passing values to continuations and sharing data.
This is different from communication in process algebras, which is based on message-passing.


\paragraph{Similarities}
There are some aspects that are similar in \TOPHAT and process algebras.
Internal communication in Hoare's CSP \cite{books/Hoare85CSP} is introduced with the concealment operator.
The semantics of CSP requires that all concealed actions are handled to exhaustion before any action with the environment can take place.
This is somewhat similar to \TOPHAT, where all enabled internal steps must be taken until the system can react to input events again.
Contrast this with Milner's CCS \cite{books/Milner89CAC}, where concealed actions are visible to the outside as $\tau$-actions, and can be interleaved with external communication.

Another similarity between \TOPHAT and process algebras, or any system with concurrency for that matter, is the need for synchronisation.
Broadly speaking, concurrency means that different parts of a program can interact with the environment independently, in an interleaved manner.
Synchronisation means that only some, but not all, of the possible interleavings are desirable.
The semantics of the step combinators in \TOPHAT, together with the fact that internal communication happens atomically, allows for concise and intuitive synchronisation code.



\subsection{Reactive programming}

\paragraph{HipHop \& Esterel}

HipHop \cite{berry2011hiphop,journals/corr/BerryS13} is a programming language tailored to the development of synchronous reactive web systems.
From a single source, both server and client applications can be generated.
Programs are written in the Hop language, a Scheme dialect.
Communication is based on a reactive layer embedded in Hop.
The set of HipHop reactive statements is based on those of the Esterel language \cite{boussinot1991esterel,journals/scp/BerryG92}.
Each reactive component starts by specifying possible input and output events.
The component then proceeds as a state machine.

Input events are sent to such a machine programmatically using Hop, or are explicitly wired to events from the client.
They are optionally associated with a Hop value.
As Hop is a dynamic language, and HipHop uses strings to identify events, events and their possible associated values are not statically checked.
Events are aggregated until the moment the machine is asked to react.
% then are perceived as conceptually simultaneous.
The machine is executed and reacts by building a multi-set of output events.
The execution of a HipHop machine is atomic.
The set of inputs is not influenced by the current computations.

As with \TOPHAT, HipHop is a \DSL embedded in a general purpose programming language.
% the same way \TOPHAT is embedded into the simply typed $\lambda$-language.
Another similarity is that both specifications lead to executable server and client applications from a single source.
However, both HipHop and Esterel are more low level regarding their specification.
Where \TOPHAT takes tasks and collaboration as a starting point,
HipHop focusses on synchronous communication and atomic execution of reactive machines.

This difference in focus shows in the way both systems define events.
In HipHop programmers can define and use their own events.
Inputs in \TOPHAT are not extensible and not visible to the developer.
They are a completely separate entity living on the semantic level.

Another important difference is the way in which both systems handle events.
In HipHop the programmer decides when a machine should process its events.
This could be just one event, or a multi-set of events that are processed simultaneously.
\TOPHAT always processes an input the moment it occurs and only handles a single event in one instance.



\paragraph{Functional reactive programming}

Functional Reactive Programming (\FRP) is a paradigm to describe dynamic changes of values in a declarative way.
This is done by specifying networks of values, called behaviours, that can depend on each other and on external events.
Behaviours can change over time, or triggered by events.
When a behaviour changes, all other behaviours that depend on it are updated automatically.
The underlying implementation that takes care of the updating usually can tie input devices, like mouse and keyboard, to event streams and behaviours to output facilities, like text fields.
This allows for declarative specifications of applications with user interfaces.

The idea of \FRP was pioneered by \citet{conf/icfp/ElliottH97}.
In the meantime there are many variants and implementations, where reactive-banana \cite{reactive-banana}, FrTime \cite{CooperK04}, and Flapjax \cite{conf/oopsla/MeyerovichGBCGBK09} belong to the most well-known.

\FRP and \TOP are different systems that have different goals in mind.
Whereas \FRP expresses automatically updating data dependencies, \TOP expresses collaboration patterns.
\TOP has no notion of time.
Tasks cannot change spontaneous over time, while behaviours can.
Only input events can change task values.
The biggest conceptual difference between a workflow in \TOP and a data network in \FRP is that an event to a task only causes updates up until the next step, while an event in \FRP propagates through the whole network.

That being said, there are some concepts that are similar in \TOP and \FRP.
The \emph{stepper} behaviour, for example, is associated with an event and yields the value of the most recent event.
This is similar to editors in \TOP.
Furthermore, both systems can be used to declaratively program user interfaces, albeit in \FRP the programmer has to construct the \GUI elements manually, and connect inputs and outputs to the correct events and behaviours.
In \TOP graphical user interfaces are automatically derived.



\subsection{Session types}

Session types are a type discipline that can be used to check whether communicating programs conform to a certain protocol.
Session types are expressions in some process calculus that describe the input/output behaviour of such programs.
Session types are useful for programming languages where modules communicate with each other via messages, like \CSP, \PICALC, or Go, to name a few.
The only form of messages in \TOP are input events which drive execution, but modules do not communicate using messages.
Therefore, session types are not applicable to \TOP in the sense used in the literature.

Formal reasoning about \TOP programs is one of our future goals for \TOPHAT.
The ideas and techniques of session types could be useful for specifying that a list of inputs of a certain form leads to desired task values.
The details are a topic for future work.



\subsection{Related web services}


\paragraph{IFTTT}

If This Then That (\IFTTT)~\cite{IFTTT} is a web service that allows users to write and
configure simple conditional statements called applets, that can interact with
web services and IoT devices. These applets can be seen as tiny workflows, that
facilitate collaboration between different services and platforms. The
complexity of the flow is very limited however, and is not geared towards humans
collaborating.

\paragraph{Google forms \& SurveyMonkey}

Google Forms~\cite{googleforms} and SurveyMonkey~\cite{wufoo} (and their various other tools like Wufoo, Precise Polling and Zoomerang) are web based survey administration apps that allows
users to compose forms and gather data from them. They are comparable to \TOP and
iTasks in that they provide an easy way for users to construct forms on the web.
However, Google Forms and Wufoo do not allow users to define what the system
then should do with this information, and can therefore not be regarded as a
workflow or business process modelling systems.

\paragraph{Chorus} % (http://www.chorus-home.org/)

Chorus~\cite{chen2017chorus} is a visual programming environment for online
collaboration apps. The goal of Chorus is to allow users to program their own
application, though which a collective task can be coordinated. Users work on a
preset data-type that drives the behaviour of the user defined applications.
Chorus can not be regarded as a fully fledged workflow system, nor does it aim
to be. The data-type that drives the behaviour of the app does not change during
execution, and therefore heavily limits what flows can be encoded.
