% !TEX root=../pldi2019.tex



\section{Related work}

There have been two previous papers that describe semantics of iTasks, by \citet{conf/ifl/KoopmanPA08} and \citet{conf/ppdp/PlasmeijerLMAK12}.
Both give a different semantics in the form of minimal implementations of a subset of the interface of iTasks.
Our work is based on them, but differs in three major ways.
First, we make an explicit distinction between the host language and the task language, to emphasize their boundaries.
Second, we make use of separate semantic functions for querying the current task value, determining the continuation after an input event, and producing a user interface.
In iTasks and its aforementioned semantics, these are integrated into the Task datatype.
Third, our system does not have the notion of task stability.
When desired, stability can be introduced by the programmer, but the system itself makes no use of it.
We argue that this does not impact the expressiveness of our language.


% mTasks is another implementation of TOP, designed for coordination of IoT devices. \alert{cite MSc Lubbers}
% mTasks has been used in a demonstrator for home automation. \alert{cite MSc Andrade}
