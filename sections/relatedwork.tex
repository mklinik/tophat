% !TEX root=../pldi2019.tex



\section{Related work}
\label{sec:relatedwork}



\paragraph{iTasks}

As mentioned earlier, iTasks is an implementation of \TOP. iTasks has many
features, and its basic combinators are versatile and powerful.vSimpler
combinators are implemented by restricting the powerful ones. This is useful for
everyday programming, where having lots of functionality at one's fingertips is
convenient and efficient.

There have been two previous papers that describe semantics of iTasks, by
\citet{conf/ifl/KoopmanPA08} and \citet{conf/ppdp/PlasmeijerLMAK12}.
Both give a different semantics in the form of minimal implementations of a
subset of the interface of iTasks. These semantics however do not make an
explicit distinction between the host language and task language and they do not
provide a formal semantics, which makes formal reasoning difficult.


% Our work differs in three major ways.
% First, we make an explicit distinction between the host language and the task
% language, to emphasize their boundaries.
% Second, we make use of separate semantic functions for querying the current task
% value, determining the continuation after an input event, and producing a user
% interface.
% In iTasks and its aforementioned semantics, these are integrated into the
% \CL{Task} datatype.
% Third, our system does not have the notion of task stability.
% When desired, stability can be introduced by the programmer, but the system
% itself makes no use of it.
% We argue that this does not impact the expressiveness of our language.

% mTasks is another implementation of TOP, designed for coordination of IoT devices. \alert{cite MSc Lubbers}
% mTasks has been used in a demonstrator for home automation. \alert{cite MSc Andrade}

\paragraph{Workflow patterns}

Much research has been done into workflow patterns. They are regarded as special
kind of the design patterns in software engineering. They identify recurring patterns
in workflow systems, much like the combinators defined by \TOPHAT. Work by
Van der Aalst~\cite{journals/dpd/AalstHKB03} defines a comprehensive list of
these pattens, and examines their availability in industry workflow software.
Workflow patterns are usually described in terms of control flow graphs, and no
formal specification is given, which makes comparison and formal reasoning more
difficult.

\paragraph{Petri net}

Petri nets, and more specifically Workflow nets, have also been used to describe
workflows~\cite{journals/jcsc/Aalst98}. They allow for the modelling and
analysis of business processes. A downside of using workflow nets is that the
model is not directly implementable, which is possible with the \TOPHAT calculus.
