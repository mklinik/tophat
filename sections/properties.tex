% !TEX root=../pldi2019.tex

\section{Properties}

\subsection{Equality}

The intuitive idea of task equality is that of observational equivalence.
We consider two tasks equal if the user can do essentially the same things with them.
The notion of \emph{essentially the same things} is made precise in the following definition.

\begin{definition}
Two events are \emph{essentially the same}, written $h_1 \approx h_2$ if they are syntactically equivalent after stripping Ls and Rs. \qed
\end{definition}

\begin{definition}
(Bisimulation) A binary relation $R$ is a bisimulation relation between tasks iff the following conditions hold for all $t_1 \mathrel{R} t_2$.
\begin{itemize}
\item $\mathcal{V}(t_1) = \mathcal{V}(t_2)$
\item $\forall h_1 . \exists h_2 . h_1 \approx h_2 \wedge t_1 \implies t_1' \wedge t_2 \implies t_2' \wedge t_1' \mathrel{R} t_2'$
\item $\forall h_2 . \exists h_1 . h_1 \approx h_2 \wedge t_1 \implies t_1' \wedge t_2 \implies t_2' \wedge t_1' \mathrel{R} t_2'$ \qed
\end{itemize}
\end{definition}

\begin{definition}
(Task equality) Two tasks are equal iff there exists a bisimulation relation containing them. \qed
\end{definition}

\subsection{Safety}

In order to validate our semantics, we show that our evaluation, normalisation
and handling semantics is type preserving. We additionally prove a progress
theorem for our small-step handling semantics.

Additionally, we show that the normalisation semantics is a big-step semantics.
While at first sight, this might seem obvious, the fact that we are dealing with
state complicates matters.

We show that our failing function $\Failing$ indeed only indicates expressions
that can not be normalised and that allow no further interaction.

Finally we prove that the function to compute all possible inputs $\Inputs$ is sound and complete.
\subsubsection{Preservation}

We show that the following three preservation Theorems hold.

\begin{theorem}[preservation under evaluation]
      For all $e$ and $s$ such that
      $\Gamma,\Sigma\infers e:\tau$ and $\Gamma\St{}\infers s$\\
      if $e,s\evaluate e',s'$
      then $\Gamma,\Sigma\infers e':\tau$ and $\Gamma,\Sigma\infers s'$
      \label{thmpreseval}
\end{theorem}

Theorem~\ref{thmpreseval} is shown to be correct by induction over $e$. The full
proof can be found in the appendix.


Moving on, we show that normalisation also preserves, by showing that the
following Theorem holds.

\begin{theorem}[preservation under normalisation]
    For all $e$ and $s$ such that $\Gamma,\Sigma\infers e:\tau$ and $\Gamma,\Sigma\infers s$\\
    if   $e,s \normalise e',s'$ then $\Gamma,\Sigma\infers e':\tau$ and $\Gamma,\Sigma\infers s'$
    \label{thmpresnorm}
\end{theorem}

Since this semantics makes use of the value function $\Value$, we first need to
show that this function also preserves types.

\begin{lemma}[Task value preserves]
  For all $e$ and $s$ such that $\Gamma,\Sigma\infers e:\Task\tau$ and $\Gamma,\Sigma\infers s$\\
  if $\Value{(e,s)}=v$ then $v:\tau$
  \label{lemmavaluepreserves}
\end{lemma}

Lemma~\ref{lemmavaluepreserves} states exactly this poperty, and is proven in the
appendix by induction over $e$. This subseqently allows us to prove
Theorem~\ref{thmpresnorm}, again by induction over $e$.

\begin{theorem}[preservation under handling]
  For all $e$, $s$ and $i$ such that $\Gamma,\Sigma\infers e:\tau$ and $\Gamma,\Sigma\infers s$\\
  if $ e,s \handle{i} e's'$ then $\Gamma,\Sigma\infers e':\tau$ and $\Gamma,\Sigma\infers s'$
   \label{thmpreshandle}
\end{theorem}

\subsubsection{Progress}

\begin{theorem}[Progress of handling]
 For all $e$, $s$ and $i$ such that $\Gamma,\Sigma\infers e:\tau$ and $\Gamma,\Sigma\infers s$\\
 if $e,s \drive{i} e',s'$ then either $\Failing(e')$ or there exists an $e''$ and $i'$ such that $e',s'\handle{i'}e'',s''$
\end{theorem}

which we prove by showing that the following Theorem holds:
\begin{theorem}
  For all $e$ and $s$ such that $\Gamma,\Sigma\infers e:\tau$ and $\Gamma,\Sigma\infers s$\\
  if $e,s \normalise e',s'$ then either $\Failing(e')$ or there exists an $e''$, $s''$ and $i$ such that $e',s'\handle{i}e'',s''$
  \label{thmprogressnorm}
\end{theorem}

\subsubsection{Normalisation is Big-Step}

\begin{theorem}
  For all $e$ and $s$ such that $\Gamma,\Sigma\infers e:\tau$ and $\Gamma,\Sigma\infers s$\\
  if $e,s\normalise e',s'$ and $e',s'\normalise e'',s''$, then $e'\equiv e''$ and $s'\equiv s''$.
\end{theorem}

\subsubsection{Failing means no interaction possible}

\begin{theorem}
  For all $e$ and $s$ such that $\Gamma,\Sigma\infers e:\tau$ and $\Gamma,\Sigma\infers s$\\
  if $\Failing(e,s)\equiv \True$, then $e,s\normalise e',s'$ implies $e\equiv e'$ and $s\equiv s'$\\
  and there is no $i$ such that $e,s\handle{i}e',s'$ for some $e'$ and $s'$.
\end{theorem}


\subsubsection{Safety of Inputs function}

\begin{theorem}
  For all $e$ and $s$ such that $\Gamma,\Sigma\infers e:\tau$ and $\Gamma,\Sigma\infers s$\\
  if $i\in\Inputs{(e)}$ then $e,s\handle{i}e',s'$
  \label{thmsafetyi1}
\end{theorem}

\begin{theorem}
  For all $e$, $s$ and $i$ such that $\Gamma,\Sigma\infers e:\tau$ and $\Gamma,\Sigma\infers s$\\
  if $e,s\handle{i}e',s'$ then  $i\in\Inputs{(e)}$
  \label{thmsafetyi2}
\end{theorem}
\subsection{Laws}
