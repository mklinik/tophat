% !TEX root=../main.tex



\section{Properties}
\label{sec:properties}



% \subsection{Safety}

In order to validate our semantics, we show that our evaluation, normalisation
and handling semantics is type preserving. We additionally prove a progress
theorem for our small-step handling semantics.
We show that our failing function $\Failing$ indeed only indicates expressions
that can not be normalised and that allow no further interaction.
Finally, we prove that the function to compute all possible inputs $\Inputs$ is sound and complete.


\subsection{Type preservation}
\label{sub:preservation}

We show that the following three preservation Theorems hold.

\begin{theorem}[Type preservation under evaluation]
  % For all well typed expressions $e$ and states $\sigma$,
  For all expressions $e$ and states $\sigma$
  such that $\Gamma,\Sigma \infers e:\tau$ and $\Gamma,\Sigma \infers \sigma$,
  if $e,\sigma \evaluate e',\sigma'$,
  then $\Gamma,\Sigma \infers e':\tau$ and $\Sigma \infers \sigma'$.
  \label{thm:pres-eval}
\end{theorem}

Where $\Sigma \infers \sigma$ means that for all $l\in \sigma$, it holds that
$\Sigma(l)=\Gamma,\Sigma\infers \sigma(l)$.

\Cref{thm:pres-eval} is shown to be correct by induction over $e$. The full
proof can be found in the appendix.


Moving on, we show that normalisation also preserves types, by showing that the following Theorem holds.

\begin{theorem}[Type preservation under normalisation]
  % For all well typed expressions $e$ and states $\sigma$,
  For all expressions $e$ and states $\sigma$
  such that $\Gamma,\Sigma \infers e:\Task\tau$ and $\Sigma \infers \sigma$,
  if $e,\sigma \normalise e',\sigma'$,
  then $\Gamma,\Sigma \infers e':\Task\tau$ and $\Sigma \infers \sigma'$.
  \label{thm:pres-norm}
\end{theorem}

Since this semantics makes use of the value function $\Value$ and the striding semantics $\stride$,
we first need to show that preservation also holds for these two semantics.

\begin{lemma}[Task value preserves types]
  % For all well typed expressions $e$ and states $\sigma$,
  For all expressions $e$ and states $\sigma$
  such that $\Gamma,\Sigma \infers e:\Task\tau$ and $\Sigma \infers \sigma$,
  if $\Value{(e,\sigma)}=v$,
  then $v:\tau$.
  \label{lem:presvalue}
\end{lemma}

\begin{lemma}[Striding preserves types]
  For all expressions $e$ and states $\sigma$
  such that $\Gamma,\Sigma \infers e:\Task\tau$ and $\Sigma \infers \sigma$,
  if $e,\sigma \stride e',\sigma'$,
  then $\Gamma,\Sigma \infers e':\Task\tau$ and $\Sigma \infers \sigma'$.
  \label{thm:pres-stride}
\end{lemma}

\Cref{lem:presvalue} and \cref{thm:pres-stride} are proven in the
appendix by induction over $e$. This subsequently allows us to prove
\Cref{thm:pres-norm}, again by induction over $e$.

This brings us finally to the type preservation property of the handling semantics.

\begin{theorem}[Type preservation under handling]
  % For all well typed expressions $e$, states $\sigma$, and inputs $i$,
  For all expressions $e$, states $\sigma$ and inputs $i$
  such that $\Gamma,\Sigma \infers e:\Task\tau$ and $\Sigma \infers \sigma$,
  if $ e,\sigma \handle{i} e',\sigma'$,
  then $\Gamma,\Sigma\infers e':\Task\tau$ and $\Sigma\infers \sigma'$.
  \label{thm:pres-handle}
\end{theorem}

And again, this is proven by induction over $e$.

From \cref{thm:pres-handle} and \cref{thm:pres-norm} we directly
obtain that the driving semantics also preserves types.

\subsection{Progress}

A well-typed term of a task type is guaranteed to progress after normalisation,
unless it is failing.

We define what we mean with progress in \cref{thm:prog-norm}.
\begin{theorem}[Progress under handling]
  For all well typed expressions $e$ and states $\sigma$,
  % For all expressions $e$ and states $\sigma$
  % such that $\Gamma,\Sigma \infers e:\Task\tau$ and $\Sigma \infers \sigma$,
  if $e,\sigma \normalise e',\sigma'$,
  then either $\Failing(e', \sigma')$
  or there exist $e''$, $\sigma''$, and $i$ such that $e',\sigma'\handle{i} e'',\sigma''$.
  \label{thm:prog-norm}
\end{theorem}

Where a well typed expression $e$ means that $\Gamma,\Sigma\infers e:\tau$ for
some type $\tau$, and a well typed state means that $\Sigma\infers \sigma$.

If an expression $e$ and state $\sigma$ are well-typed, then after normalisation, the pair $e',\sigma'$
either fails, or there exists some input $i$ that can be handled by it under the handling semantics.
In order to prove this Theorem, we require an additional theorem.

% \paragraph{Failing}

We need to show that the failing function $\Failing$ behaves as desired.

\begin{theorem}[Failing means no interaction possible]
  For all well typed expressions $e$ and states $\sigma$,
  % For all expressions $e$ and states $\sigma$
  % such that $\Gamma,\Sigma \infers e:\Task\tau$ and $\Sigma \infers \sigma$,
  and $e,\sigma \normalise e',\sigma'$,
  we have that $\Failing(e',\sigma') = \True$,
  if and only if there is no input $i$
  such that $e',\sigma'\handle{i} e'',\sigma''$ for some $e''$ and $\sigma''$.
  \label{thm:failing}
\end{theorem}

The Theorem above states that an expression $e$ and state $\sigma$ are failing, if,
after normalisation, there exists no input that can be handled by it.
We prove the theorem to be true by induction on $e'$.

% \paragraph{Proof}

We now have the ingredients to prove \cref{thm:prog-norm}.

\begin{proof}
  Given $\Gamma,\Sigma\infers e:\Task\tau$ and $\Sigma\infers \sigma$ and after
  normalisation $e,\sigma \normalise e',\sigma'$, we find ourselves in either one of the
  following situations:

  There exists an $i$ such that $e',\sigma'\handle{i} e'',\sigma''$.

  There does not exist an $i$ such that $e',\sigma'\handle{i} e'',\sigma''$. In this case, we
  know that $\Failing(e',\sigma')$ must be true, by \cref{thm:failing}.
\end{proof}



\subsection{Soundness and Completeness of Inputs}

In order to validate the function that calculates all possible inputs $\Inputs$,
we want to show that the set of possible inputs it produces is both sound and complete with respect to the handle semantics.
By sound we mean that all inputs in the set of possible inputs can actually be handled by the handle semantics,
and by complete we mean that the set of possible inputs contains all inputs that can be handled by the handle semantics.
\Cref{thm:safety-i} expresses exactly this property.

\begin{theorem}[Inputs function is sound and complete]
  For all well typed expressions $e$, states $\sigma$, and inputs $i$,
  % For all expressions $e$, states $\sigma$, and inputs $i$
  % such that $\Gamma,\Sigma \infers e:\tau$ and $\Sigma \infers \sigma$,
  we have that $i \in \Inputs{(e,\sigma)}$ if and only if $e,\sigma \handle{i} e',\sigma'$.
  \label{thm:safety-i}
\end{theorem}

We prove the above theorem by induction over $e$. The proof is listed in the
appendix.
