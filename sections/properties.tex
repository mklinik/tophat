% !TEX root=../pldi2019.tex



\section{Properties}



% \subsection{Safety}

In order to validate our semantics, we show that our evaluation, normalisation
and handling semantics is type preserving. We additionally prove a progress
theorem for our small-step handling semantics.

Additionally, we show that the normalisation semantics is a big-step semantics.
While at first sight, this might seem obvious, the fact that we are dealing with
state complicates matters.

We show that our failing function $\Failing$ indeed only indicates expressions
that can not be normalised and that allow no further interaction.

Finally we prove that the function to compute all possible inputs $\Inputs$ is sound and complete.



\subsection{Preservation}
\label{sub:preservation}

We show that the following three preservation Theorems hold.

\begin{theorem}[preservation under evaluation]
      For all $e$ and $s$ such that
      $\Gamma,\Sigma\infers e:\tau$ and $\Gamma\St{}\infers s$\\
      if $e,s\evaluate e',s'$
      then $\Gamma,\Sigma\infers e':\tau$ and $\Sigma\infers s'$
      \label{thmpreseval}
\end{theorem}

Theorem~\ref{thmpreseval} is shown to be correct by induction over $e$. The full
proof can be found in the appendix.


Moving on, we show that normalisation also preserves, by showing that the
following Theorem holds.

\begin{theorem}[preservation under normalisation]
    For all $e$ and $s$ such that $\Gamma,\Sigma\infers e:\tau$ and $\Sigma\infers s$\\
    if   $e,s \stride e',s'$ then $\Gamma,\Sigma\infers e':\tau$ and $\Sigma\infers s'$
    \label{thmpresnorm}
\end{theorem}

Since this semantics makes use of the value function $\Value$, we first need to
show that this function also preserves types.

\begin{lemma}[task value preserves]
  For all $e$ and $s$ such that $\Gamma,\Sigma\infers e:\Task\tau$ and $\Sigma\infers s$\\
  if $\Value{(e,s)}=v$ then $v:\tau$
  \label{lemmavaluepreserves}
\end{lemma}

Lemma~\ref{lemmavaluepreserves} states exactly this poperty, and is proven in the
appendix by induction over $e$. This subseqently allows us to prove
Theorem~\ref{thmpresnorm}, again by induction over $e$.

This brings us finally to the type preservation property of the handling semantics.

\begin{theorem}[preservation under handling]
  For all $e$, $s$ and $i$ such that $\Gamma,\Sigma\infers e:\tau$ and $\Sigma\infers s$\\
  if $ e,s \handle{i} e's'$ then $\Gamma,\Sigma\infers e':\tau$ and $\Sigma\infers s'$
   \label{thmpreshandle}
\end{theorem}

And again, this is proven by induction over $e$.



\subsection{Progress}

Furthermore, a well-typed term of a task type is guaranteed to progress after
normalisation, unless it is failing.

We define what we mean with progress in Theorem~\ref{thmprogressnorm}.

\begin{theorem}
  For all $e$ and $s$ such that $\Gamma,\Sigma\infers e:\Task\tau$ and $\Sigma\infers s$\\
  if $e,s \bar{\stride} e',s'$ then either $\Failing(e')$ or there exists an $e''$, $s''$ and $i$ such that $e',s'\handle{i}e'',s''$
  \label{thmprogressnorm}
\end{theorem}

If an expression $e$ and state $s$ are well-typed, then after normalisation, it
either fails, or there exists some input $i$ that can be handled by it.
In order to prove this Theorem to be true, we require two additional theorems.

% \paragraph{Normalisation is Big-Step}

The normalisation semantics is a big-step stemantics. This would be quite
straight-forward, if we did not have to deal with state. Consider the following
example.

$(\Update l \Then \lambda x:\Bool .\ \If{x}{e}{\Fail})\And (\Edit l:=\True)$ with $s=\{l\mapsto \False\}$

When we apply S-And in order to normalise the expression above, we obtain
$(\Update l \Then \lambda x:\Bool .\ \If{x}{e}{\Fail})\And (\Edit \unit)$ with $s'=\{l\mapsto \True\}$

an expression which in fact is not normalised. The issue here lies in the fact
that $s$ gets updated, and allows the first component of $\And$ to be further
normalised, in this case to $e$. To overcome this problem, the N-Done and
N-Stride rules have been added. They ensure that normalisation is applied untill
the shared memory $s$ has become stable and no further normalisation can be
applied.

To prove that this is true, and that therefore normalisation is a big-step
semantics, we show the following theorem to be true.

\begin{theorem}
  For all $e$ and $s$ such that $\Gamma,\Sigma\infers e:\Task\tau$ and $\Sigma\infers s$\\
  if $e,s\normalise e',s'$ and $e',s'\normalise e'',s''$, then $e'= e''$ and $s'= s''$.
  \label{thmnormisbigstep}
\end{theorem}

An additional lemma is required.

\begin{lemma}
  For all $e$ and $s$ such that $\Gamma,\Sigma\infers e:\Task\tau$ and $\Sigma\infers s$\\
  if $e,s\evaluate t,s'$, $t,s'\stride t',s''$, $s=s''$ and $t',s''\evaluate t'',s'''$, then $t'=t''$ and $s''=s'''$
  \label{lemmastridedoesnotevaluate}
\end{lemma}

% \paragraph{Failing}

We then show that the failing function $\Failing$ behaves as desired.

\begin{theorem}[Failing means no interaction possible]
  For all $e$ and $s$ such that $\Gamma,\Sigma\infers e:\Task\tau$ and $\Sigma\infers s$\\
  and $e,s\stride e',s'$, then $\Failing(e',s')\equiv \True$, if and only if there is no $i$ such that $e',s'\handle{i}e'',s''$ for some $e''$ and $s''$.
  \label{thmfailing}
\end{theorem}

The Theorem above states that an expression $e$ and state $s$ are failing, if,
after normalisation, there exists no input that can be handled by it.
We prove the theorem to be true by induction on $e'$.

% \paragraph{Proof}

We now have the ingredients to prove Theorem~\ref{thmprogressnorm}.

\begin{proof}
  Given $\Gamma,\Sigma\infers e:\Task\tau$ and $\Sigma\infers s$ and after
  normalisation $e,s \stride e',s'$, we find ourselves in either one of the
  following situations:\\

  There exists an $i$ such that $e',s'\handle{i}\_,\_$. \qed\\

  There does not exist an $i$ such that $e',s'\handle{i}\_,\_$. In this case, we
  know that $\Failing(e',s')$ must be true, since Theorem~\ref{thmnormisbigstep}
  gives us that $e',s'$ can not be normalised further and we assumed that there
  is no possible input that can be handled.
\end{proof}



\subsection{Safety of Inputs function}

\begin{theorem}
  For all $e$, $s$ and $i$ such that $\Gamma,\Sigma\infers e:\tau$ and $\Sigma\infers s$\\
  then $i\in\Inputs{(e)}$ if and only if $e,s\handle{i}e',s'$
  \label{thmsafetyi}
\end{theorem}
