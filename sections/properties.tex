% !TEX root=../pldi2019.tex



\section{Properties}



% \subsection{Safety}

In order to validate our semantics, we show that our evaluation, normalisation
and handling semantics is type preserving. We additionally prove a progress
theorem for our small-step handling semantics.

Additionally, we show that the normalisation semantics is a big-step semantics.
While at first sight, this might seem obvious, the fact that we are dealing with
state complicates matters.

We show that our failing function $\Failing$ indeed only indicates expressions
that can not be normalised and that allow no further interaction.

Finally we prove that the function to compute all possible inputs $\Inputs$ is sound and complete.



\subsection{Preservation}
\label{sub:preservation}

We show that the following three preservation Theorems hold.

\begin{theorem}[Preservation under evaluation]
  % For all well typed expressions $e$ and states $s$,
  For all expressions $e$ and states $s$
  such that $\Gamma,\Sigma \infers e:\tau$ and $\Gamma,\Sigma \infers s$,
  if $e,s \evaluate e',s'$,
  then $\Gamma,\Sigma \infers e':\tau$ and $\Sigma \infers s'$.
  \label{thm:pres-eval}
\end{theorem}

Where $\Sigma \infers s$ means that for all $l\in s$, it holds that
$\Sigma(l)=\Gamma,\Sigma\infers s(l)$.

Theorem~\ref{thm:pres-eval} is shown to be correct by induction over $e$. The full
proof can be found in the appendix.


Moving on, we show that normalisation also preserves, by showing that the
following Theorem holds.

\begin{theorem}[Preservation under normalisation]
  % For all well typed expressions $e$ and states $s$,
  For all expressions $e$ and states $s$
  such that $\Gamma,\Sigma \infers e:\tau$ and $\Sigma \infers s$,
  if $e,s \normalise e',s'$,
  then $\Gamma,\Sigma \infers e':\tau$ and $\Sigma \infers s'$.
  \label{thm:pres-norm}
\end{theorem}

Since this semantics makes use of the value function $\Value$ and the striding semantics $\stride$. We first need to
show that preservation also holds for these two semantics.

\begin{lemma}[Task value preserves]
  % For all well typed expressions $e$ and states $s$,
  For all expressions $e$ and states $s$
  such that $\Gamma,\Sigma \infers e:\Task\tau$ and $\Sigma \infers s$,
  if $\Value{(e,s)}=v$,
  then $v:\tau$.
  \label{lem:presvalue}
\end{lemma}

\begin{lemma}[Striding preserves]
  For all expressions $e$ and states $s$
  such that $\Gamma,\Sigma \infers e:\Task\tau$ and $\Sigma \infers s$,
  if $e,s \stride e',s'$,
  then $\Gamma,\Sigma \infers e':\tau$ and $\Sigma \infers s'$.
  \label{thm:pres-stride}
\end{lemma}

Lemma~\ref{lem:presvalue} states exactly this property, and is proven in the
appendix by induction over $e$. This subsequently allows us to prove
Theorem~\ref{thm:pres-norm}, again by induction over $e$.

This brings us finally to the type preservation property of the handling semantics.

\begin{theorem}[Preservation under handling]
  % For all well typed expressions $e$, states $s$, and inputs $i$,
  For all expressions $e$, states $s$ and inputs $i$
  such that $\Gamma,\Sigma \infers e:\tau$ and $\Sigma \infers s$,
  if $ e,s \handle{i} e's'$,
  then $\Gamma,\Sigma\infers e':\tau$ and $\Sigma\infers s'$.
  \label{thm:pres-handle}
\end{theorem}

And again, this is proven by induction over $e$.



\subsection{Progress}

Furthermore, a well-typed term of a task type is guaranteed to progress after
normalisation, unless it is failing.

We define what we mean with progress in Theorem~\ref{thm:prog-norm}.
\begin{theorem}[Progress under handling]
  For all well typed expressions $e$ and states $s$,
  % For all expressions $e$ and states $s$
  % such that $\Gamma,\Sigma \infers e:\Task\tau$ and $\Sigma \infers s$,
  if $e,s \normalise e',s'$,
  then either $\Failing(e', s')$
  or there exist $e''$, $s''$, and $i$ such that $e',s'\handle{i} e'',s''$.
  \label{thm:prog-norm}
\end{theorem}

If an expression $e$ and state $s$ are well-typed, then after normalisation, it
either fails, or there exists some input $i$ that can be handled by it.
In order to prove this Theorem to be true, we require two additional theorems.

% \paragraph{Normalisation is Big-Step}

The normalisation semantics is a big-step semantics. This would be quite
straight-forward, if we did not have to deal with state. Consider the following
example:
$(\Update l \Then \lambda x:\Bool.\allowbreak \If{x}{e}{\Fail}) \And (l:=\True; \Edit \unit)$ with $s=\set{l\mapsto\False}$.
When we apply \refrule{S-And} in order to normalise the expression above, we obtain
$(\Update l \Then \lambda x:\Bool .\ \If{x}{e}{\Fail})\And (\Edit \unit)$ with $s'=\set{l\mapsto\True}$,
an expression which in fact is not normalised. The issue here lies in the fact
that $s$ gets updated, and allows the first component of $\And$ to be further
normalised, in this case to $e$. To overcome this problem, the \refrule{N-Done} and
\refrule{N-Repeat} rules have been added. They ensure that normalisation is applied until
the shared memory $s$ has become stable and no further normalisation can be
applied.

To prove that this is true, and that therefore normalisation is a big-step
semantics, we show the following theorem to be true.

\begin{theorem}[Normalisation is big step]
  For all well typed expressions $e$ and states $s$,
  % For all expressions $e$ and states $s$
  % such that $\Gamma,\Sigma \infers e:\Task\tau$ and $\Sigma\infers s$,
  if $e,s \normalise e',s'$ and $e',s' \normalise e'',s''$,
  then $e'= e''$ and $s'= s''$.
  \label{thm:norm-is-bigstep}
\end{theorem}

An additional lemma is required.

\begin{lemma}[]
  For all well typed expressions $e$ and states $s$,
  % For all expressions $e$ and states $s$
  % such that $\Gamma,\Sigma \infers e:\Task\tau$ and $\Sigma \infers s$,
  if $e,s \evaluate t,s'$, $t,s' \stride t',s''$, $s'=s''$ and $t',s'' \evaluate t'',s'''$,
  then $t'=t''$ and $s''=s'''$.
  \label{lem:stride-does-not-eval}
\end{lemma}

% \paragraph{Failing}

We then show that the failing function $\Failing$ behaves as desired.

\begin{theorem}[Failing means no interaction possible]
  For all well typed expressions $e$ and states $s$,
  % For all expressions $e$ and states $s$
  % such that $\Gamma,\Sigma \infers e:\Task\tau$ and $\Sigma \infers s$,
  and $e,s \normalise e',s'$,
  we have that $\Failing(e',s') = \True$,
  if and only if there is no input $i$
  such that $e',s'\handle{i} e'',s''$ for some $e''$ and $s''$.
  \label{thm:failing}
\end{theorem}

The Theorem above states that an expression $e$ and state $s$ are failing, if,
after normalisation, there exists no input that can be handled by it.
We prove the theorem to be true by induction on $e'$.

% \paragraph{Proof}

We now have the ingredients to prove Theorem~\ref{thm:prog-norm}.

\begin{proof}
  Given $\Gamma,\Sigma\infers e:\Task\tau$ and $\Sigma\infers s$ and after
  normalisation $e,s \normalise e',s'$, we find ourselves in either one of the
  following situations:

  There exists an $i$ such that $e',s'\handle{i}\_,\_$.

  There does not exist an $i$ such that $e',s'\handle{i}\_,\_$. In this case, we
  know that $\Failing(e',s')$ must be true, since Theorem~\ref{thm:norm-is-bigstep}
  gives us that $e',s'$ can not be normalised further and we assumed that there
  is no possible input that can be handled.
\end{proof}



\subsection{Completeness of inputs}

\begin{theorem}[Inputs are complete]
  \todo{Is this a correct title?}
  For all well typed expressions $e$, states $s$, and inputs $i$,
  % For all expressions $e$, states $s$, and inputs $i$
  % such that $\Gamma,\Sigma \infers e:\tau$ and $\Sigma \infers s$,
  we have that $i \in \Inputs{(e)}$ if and only if $e,s \handle{i} e',s'$.
  \label{thm:safety-i}
\end{theorem}
