% !TEX root=../pldi2019.tex

\appendix
\section{Appendix}

  \subsection{Rules}

    \begin{mathpar}
      \boxed{\RelationT} \\
      \userule{T-Var} \quad
      \userule{T-Loc} \\
      \userule{T-Abs} \quad
      \userule{T-App} \\
      \userule{T-If} \\
      \userule{T-Pair} \\
      \userule{T-Ref} \quad
      \userule{T-Deref} \\\
      \userule{T-Assign}
    \end{mathpar}

    \begin{mathpar}
      \boxed{\RelationE} \\
      \userule{E-Value} \\
      \userule{E-App} \\
      \userule{E-IfTrue} \\
      \userule{E-IfFalse} \\
      \userule{E-Pair} \\
      \userule{E-Ref} \quad
      \userule{E-Deref} \\
      \userule{E-Assign}
    \end{mathpar}

  \subsection{Proofs}

  \subsubsection{Theorem~\ref{thm:pres-eval}}
\begin{proof}
  We prove Theorem~\ref{thm:pres-eval} by induction on $e$:\\

  \noindent\textbf{Case} $e=\lambda x:\tau.e, e_1 e_2, x, c, l, e_1 \star e_2,
      \If{e_1}{e_2}{e_3},$\\
      $\tuple{e_1, e_2},\unit,\Ref e,!e,e_1 := e_2$ preservation has
      been proven for these cases by \todo{insert cite}\\

  \noindent\textbf{Case} $\userule{E-Edit}$
      Given that $\Gamma,\Sigma\infers\Edit e:\Task \tau$ and $\Sigma\infers s$,
      \refrule{T-Edit} gives us that $\Gamma,\Sigma\infers e:\tau$. The induction
      hypothesis gives us that $e,s\evaluate v,s'$ also preserves, and thus
      $\Gamma,\Sigma\infers v:\tau$ and $\Sigma\infers s'$. Therefore
      $\Gamma,\Sigma\infers\Edit v:\Task\tau$.\\

  \noindent\textbf{Case} $\userule{E-Enter}$
      Evaluation does not alter $e$ and $s$, therefore this case holds tivially.\\

  \noindent\textbf{Case} $\userule{E-Update}$
      Given that $\Gamma,\Sigma\infers \Edit e:\Task \tau$ and
      $\Sigma\infers s$, \refrule{T-Update} gives us that $\Gamma,\Sigma\infers e:\Ref \tau$.
      The induction hypothesis gives us that $e,s\evaluate l,s'$ also preserves,
      and thus $\Gamma,\Sigma\infers l:\Ref\tau$ and $\Sigma\infers s'$.
      Therefore $\Gamma,\Sigma\infers\Update l:\Task\tau$\\

  \noindent\textbf{Case} $\userule{E-Fail}$
      Evaluation does not alter $e$ and $s$, therefore this case holds tivially.\\

  \noindent\textbf{Case} $\userule{E-Then}$
      Given that $\Gamma,\Sigma\infers e_1\Then e_2:\Task \tau$ and
      $\Sigma\infers s$, \refrule{T-Then} gives us that
      $\Gamma,\Sigma\infers e_1:\Task\tau_1$ and
      $\Gamma,\Sigma\infers e_2:\tau_1 \to \Task \tau$. By the induction
      hypothesis, we know that $e_1,s\evaluate t_1,s'$ preserves and thus
      $\Gamma,\Sigma\infers t_1:\Task\tau_1$ and $\Sigma\infers s'$. Therefore
      $\Gamma,\Sigma\infers t_1\Then e_2:\Task\tau$.\\

  \noindent\textbf{Case} $\userule{E-Next}$
      Given that $\Gamma,\Sigma\infers e_1\Next e_2:\Task \tau$ and
      $\Sigma\infers s$, \refrule{T-Next} gives us that
      $\Gamma,\Sigma\infers e_1:\Task\tau_1$ and
      $\Gamma,\Sigma\infers e_2:\tau_1 \to \Task \tau$. By the induction
      hypothesis, we know that $e_1,s\evaluate t_1,s'$ preserves and thus
      $\Gamma,\Sigma\infers t_1:\Task\tau_1$ and $\Sigma\infers s'$. Therefore
      $\Gamma,\Sigma\infers t_1\Next e_2:\Task\tau$.\\

  \noindent\textbf{Case} $\userule{E-And}$
      Given that $\Gamma,\Sigma\infers e_1\And e_2:\Task(\tau_1\times\tau_2)$
      and $\Sigma\infers s$, \refrule{T-And} gives us that
      $\Gamma,\Sigma\infers e_1:\Task\tau_1$ and
      $\Gamma,\Sigma\infers e_2:\Task\tau_2$. By the induction hypothesis, we
      know that both $e_1,s\evaluate t_1,s'$ and $e_2,s'\evaluate t_2,s''$
      preserve and thus $\Gamma,\Sigma\infers t_1:\Task\tau_1$,
      $\Sigma\infers s'$, $\Gamma,\Sigma\infers t_2:\Task\tau_2$ and
      $\Sigma\infers s''$. Therfore
      $\Gamma,\Sigma\infers t_1\And t_2:\Task(\tau_1\times\tau_2)$\\

  \noindent\textbf{Case} $\userule{E-Or}$
      Given that $\Gamma,\Sigma\infers e_1\Or e_2:\Task\tau$ and $\Sigma\infers s$, \refrule{T-Or} gives us that $\Gamma,\Sigma\infers e_1:\Task\tau$ and
      $\Gamma,\Sigma\infers e_2:\Task\tau$. By the induction hypothesis, we have that both
      $e_1,s\evaluate t_1,s'$ and $e_2,s'\evaluate t_2,s''$ preserve and thus $\Gamma,\Sigma\infers t_1:\Task\tau$, $\Sigma\infers s'$,
      $\Gamma,\Sigma\infers t_2:\Task\tau$ and $\Sigma\infers s''$. Therefore $\Gamma,\Sigma\infers t_1\Or t_2:\Task\tau$.\\

  \noindent\textbf{Case} $\userule{E-Xor}$
      Evaluation does not alter $e$ and $s$, therefore this case holds tivially.\\

  \noindent\textbf{Case} $\userule{E-Appoint}$
      Given that $\Gamma,\Sigma\infers u \At e:\Task\tau$ and $\Sigma\infers s$, the induction hypothesis gives us that $e,s\evaluate t,s'$ also preserves, and therefore by \refrule{T-Appoint} $\Gamma,\Sigma\infers u \At t:\Task\tau$.
\end{proof}



\subsubsection{Lemma~\ref{lem:presvalue}}
\begin{proof}
  We prove Lemma~\ref{lem:presvalue} by induction over $e$.\\

  \noindent\textbf{Case} $\Value{(\Edit v,s)}=v$ By T-Edit, if
  $\Gamma,\Sigma\infers \Edit v:\Task\tau$, then $\Gamma,\Sigma\infers v:\tau$.\\

  \noindent\textbf{Case} $\Value{(\Enter \tau,s)}=\bot$ Since this case does not
  lead to a value, the lemma holds trivially.\\

  \noindent\textbf{Case} $\Value{(\Update l,s)}=s(l)$ Given that
  $\Gamma,\Sigma\infers\Update l:\Task \tau$ and $\Sigma\infers s$, we know that
  $\Gamma,\Sigma\infers s(l):\tau$ by definiton.\\

  \noindent\textbf{Case} $\Value{(\Fail,s)}=\bot$ Since this case does not lead
  to a value, the lemma holds trivially.\\

  \noindent\textbf{Case} $\Value{(t_1\Then e_2,s)}=\bot$ Since this case does
  not lead to a value, the lemma holds trivially.\\

  \noindent\textbf{Case} $\Value{(t_2\Next e_2,s)}=\bot$ Since this case does
  not lead to a value, the lemma holds trivially.\\

  \noindent\textbf{Case} $\Value{(t_1\And t_2,s)}=\tuple{v_1, v_2}$ given that
  $\Value{(t_1,s)}=v_1\wedge\Value{(t_2,s)}=v_2$\\ By \refrule{T-And} we have that
  $\Gamma,\Sigma\infers t_1\And t_2:\Task(\tau_1\times\tau_2)$ and
  $\Gamma,\Sigma\infers t_1:\tau_1$ and $\Gamma,\Sigma\infers t_2:\tau_2$. By the
  induction hypothesis, $ \Value{(t_1,s)}=v_1$ and $\Value{(t_2,s)}=v_2$ preserve,
  and thus $\Gamma,\Sigma\infers v_1:\tau_1$ and $\Gamma,\Sigma\infers v_2:\tau_2$.
  This gives us that $\Gamma,\Sigma\infers \tuple{v_1, v_2}:\Task(\tau_1\times\tau_2)$ \\

  \noindent\textbf{Case} $\Value{(t_1\And t_2,s)}=\bot$ given that
  $\neg(\Value{(t_1,s)}=v_1\wedge\Value{(t_2,s)}=v_2)$\\ Since this case does
  not lead to a value, the lemma holds trivially.\\

  \noindent\textbf{Case} $\Value{(t_1\Or t_2,s)}=v_1$ given that
  $\Value{(t_1,s)}=v_1$\\
  By \refrule{T-Or} we have that $\Gamma,\Sigma\infers t_1\Or t_2:\Task\tau$,
  and $\Gamma,\Sigma\infers t_1:\Task\tau$ and
  $\Gamma,\Sigma\infers t_2:\Task\tau$. By the induction hypothesis, we have
  that $\Gamma,\Sigma\infers v_1:\tau$.\\

  \noindent\textbf{Case} $\Value{(t_1\Or t_2,s)}=v_2$ given that
  $\Value{(t_1,s)}=\bot\wedge\Value{(t_2,s)}=v_2$\\
  By \refrule{T-Or} we have that $\Gamma,\Sigma\infers t_1\Or t_2:\Task\tau$,
  and $\Gamma,\Sigma\infers t_1:\Task\tau$ and
  $\Gamma,\Sigma\infers t_2:\Task\tau$. By the induction hypothesis, we have
  that $\Gamma,\Sigma\infers v_2:\tau$.\\

  \noindent\textbf{Case} $\Value{(t_1\Or t_2,s)}=\bot$ given that
  $\Value{(t_1,s)}=\bot\wedge\Value{(t_2,s)}=\bot$\\ Since this case does not
  lead to a value, the lemma holds trivially.\\

  \noindent\textbf{Case} $\Value{(t_1\Xor t_2,s)}=\bot$ Since this case does not
  lead to a value, the lemma holds trivially.\\

  \noindent\textbf{Case} $\Value{(u\At t,s)}=\Value(t,s)$ This case follows
  directly from the induction hypothesis.\\
\end{proof}



\subsubsection{Theorem~\ref{thm:pres-stride}}
\begin{proof}
  We prove Theorem~\ref{thm:pres-stride} by induction on $e$:\\

  \noindent\textbf{Case} $\userule{S-Fail}$ Since this case does not alter the
  expression, the theorem holds trivially.\\

  \noindent\textbf{Case} $\userule{S-Xor}$ Since this case does not alter the
  expression, the theorem holds trivially.\\

  \noindent\textbf{Case} $\userule{S-Update}$ Since this case does not alter
  the expression, the theorem holds trivially.\\

  \noindent\textbf{Case} $\userule{S-Fill}$ Since this case does not alter the
  expression, the theorem holds trivially.\\

  \noindent\textbf{Case} $\userule{S-Edit}$ Since this case does not alter the
  expression, the theorem holds trivially.\\

  \noindent\textbf{Case} $\userule{S-And}$ Given that
  $\Gamma,\Sigma\infers t_1\And t_2:\Task(\tau_1\times\tau_2)$, by
  \refrule{T-And we} have $\Gamma,\Sigma\infers t_1:\tau_1$ and
  $\Gamma,\Sigma\infers t_2:\tau_2$. By the induction hypothesis, we also have $\Gamma,\Sigma\infers t_1':\tau_1$ and $\Gamma,\Sigma\infers t_2':\tau_2$.
  This gives us that
  $\Gamma,\Sigma\infers t_1'\And t_2':\Task(\tau_1\times\tau_2)$.\\

  \noindent\textbf{Case} $\userule{S-Next}$ Given that
  $\Gamma,\Sigma\infers e_1\Next e_2:\Task \tau$, \refrule{T-Then} gives us that $\Gamma,\Sigma\infers t_1:\Task\tau_1$ and
  $\Gamma,\Sigma\infers e_2:\tau_1 \to \Task \tau$. By the induction hypothesis,
  we know that $t_1\stride t_1'$ preserves and thus
  $\Gamma,\Sigma\infers t_1':\Task\tau_1$. Therefore
  $\Gamma,\Sigma\infers t_1'\Next e_2:\Task\tau$.\\\\

  \noindent\textbf{Case} $\userule{S-OrLeft}$ Given that
  $\Gamma,\Sigma\infers t_1\Or t_2:\Task\tau$, by \refrule{T-Or} we have
  $\Gamma,\Sigma\infers t_1:\Task\tau$. By the induction hypothesis, we know that
  $t_1\stride t_1'$ preserves and thus $\Gamma,\Sigma\infers t_1':\Task\tau$.\\

  \noindent\textbf{Case} $\userule{S-OrRight}$ Given that
  $\Gamma,\Sigma\infers t_1\Or t_2:\Task\tau$, by \refrule{T-Or} we have $\Gamma,\Sigma\infers t_2:\Task\tau$. By the induction hypothesis, we know that
  $t_2\stride t_2'$ preserves and thus $\Gamma,\Sigma\infers t_2':\Task\tau$.\\

  \noindent\textbf{Case} $\userule{S-OrNone}$ Given that
  $\Gamma,\Sigma\infers t_1\Or t_2:\Task\tau$, by \refrule{T-Or} we have
  $\Gamma,\Sigma\infers t_1:\Task\tau$ and $\Gamma,\Sigma\infers t_2:\Task\tau$.
  By the induction hypothesis, we know that $t_1\stride t_1'$ and
  $t_2\stride t_2'$ preserve, and thus
  $\Gamma,\Sigma\infers t_1'\Or t_2':\Task\tau$.\\

  \noindent\textbf{Case} $\userule{S-ThenStay}$ Given that
  $\Gamma,\Sigma\infers t_1\Then e_2:\Task\tau$,
  by \refrule{T-Then} we have $\Gamma,\Sigma\infers t_1:\Task\tau_1$ and
  $\Gamma,\Sigma\infers e_2:\tau_1\to\Task\tau$. By the induction
  hypothesis, we know that $t_1\stride t_1'$ preserves, and thus
  $\Gamma,\Sigma\infers t_1'\Then e_2:\Task\tau$.\\

  \noindent\textbf{Case} $\userule{S-ThenFail}$ Given that
  $\Gamma,\Sigma\infers t_1\Then e_2:\Task\tau$,
  by \refrule{T-Then} we have $\Gamma,\Sigma\infers t_1:\Task\tau_1$ and
  $e_2:\tau_1\to\Task\tau$. By the induction
  hypothesis, we know that $t_1\stride t_1'$ preserves, and thus
  $\Gamma,\Sigma\infers t_1'\Then e_2:\Task\tau$.\\

  \noindent\textbf{Case} $\userule{S-ThenCont}$Given that
  $\Gamma,\Sigma\infers t_1\Then e_2:\Task\tau$, by \refrule{T-Then} we have
  $\Gamma,\Sigma\infers t_1:\Task\tau_1$ and
  $\Gamma,\Sigma\infers e_2:\tau_1\to\Task\tau$. By the induction hypothesis, we
  know that $t_1\stride t_1'$ preserves. By Lemma~\ref{lem:presvalue}, we know
  that $\Value{(t_1')}=v_1$ preserves. By Theorem~\ref{thm:pres-eval} we know
  that $e_2 v_1\evaluate t_2$ preserves. And finally by the induction hypothesis,
  we know that $t_2\stride t_2'$ preserves. Therefore
  $\Gamma,\Sigma\infers t_2':\Task\tau$.\\

  \noindent\textbf{Case} $\userule{S-Appoint}$ Given that
  $\Gamma,\Sigma\infers u\At t:\Task\tau$, by \refrule{T-Appoint} we have
  $t:\Task\tau$. By the induction hypothesis, we know that $t,s\stride t',s'$
  preserves, and thus $\Gamma,\Sigma\infers u\At t':\Task\tau$.\\

\end{proof}

\subsubsection{Theorem~\ref{thm:pres-norm}}

\begin{proof}
  We prove Theorem~\ref{thm:pres-norm} by induction on $e$:\\

  \noindent\textbf{Case} $\userule{N-Done}$ Given that
  $\Gamma,\Sigma\infers e:\Task\tau$ and $\Sigma\infers s$, we know that
  $\Gamma,\Sigma\infers t:\Task\tau$ and $\Sigma\infers s'$ by
  Theorem~\ref{thm:pres-eval}. Then by Theorem~\ref{thm:pres-stride}, we have
  $\Gamma,\Sigma\infers t':\Task\tau$ and $\Sigma\infers s''$.\\

  \noindent\textbf{Case} $\userule{N-Repeat}$ Given that
  $\Gamma,\Sigma\infers e:\Task\tau$ and $\Sigma\infers s$, we know that
  $\Gamma,\Sigma\infers t:\Task\tau$ and $\Sigma\infers s'$ by
  Theorem~\ref{thm:pres-eval}. Then by Theorem~\ref{thm:pres-stride}, we have
  $\Gamma,\Sigma\infers t':\Task\tau$ and $\Sigma\infers s''$. Then by the
  induction hypothesis, we finally obtain that
  $\Gamma,\Sigma\infers t'':\Task\tau$ and $\Sigma\infers s'''$.\\

\end{proof}

\subsubsection{Theorem~\ref{thm:pres-handle}}

We require the following Lemma for this proof.

\begin{lemma}
  Given that $\Sigma\infers s$, $\Sigma(l)=\tau$ and $\Gamma,\Sigma\infers v:\tau$, it holds that $\Sigma\infers s[l\mapsto v]$
  \label{lemmasigmaconsistent}
\end{lemma}
This lemma follows immediately from definition.

\begin{proof}
  We prove Theorem~\ref{thm:pres-handle} by induction on $e$:

  \noindent\textbf{Case} $\userule{H-Change}$ Given that
  $\Gamma,\Sigma\infers\Edit v:\Task\tau$ and $\Sigma\infers s$, the
  \refrule{H-Change} rule additionally gives us that $v,v':\tau$. Therefore by
  \refrule{T-Edit} we have that $\Gamma,\Sigma\infers\Edit v':\Task\tau$\\

  \noindent\textbf{Case} $\userule{H-Empty}$ Given that
  $\Gamma,\Sigma\infers\Edit v:\Task\beta$ and $\Sigma\infers s$, the
  \refrule{H-Empty} rule additionally gives us that $v:\tau$. Then by
  \refrule{T-Enter} we have $\Gamma,\Sigma\infers\Enter \tau:\Task\tau$ \\

  \noindent\textbf{Case} $\userule{H-Fill}$ Given that
  $\Gamma,\Sigma\infers\Enter\tau$ and $\Sigma\infers s$, the \refrule{H-Fill}
  rule additionally gives us that $v':\tau$. Then by \refrule{T-Enter} we have
  $\Gamma,\Sigma\infers \Edit v':\Task\tau$.\\

  \noindent\textbf{Case} $\userule{H-Update}$ Given that
  $\Gamma,\Sigma\infers\Update l:\Task\tau$ and $\Sigma\infers s$. This gives us
  that $\Sigma(l)=\tau$, and we additionally obtain $s(l),v':\tau$ by
  \refrule{H-Update}. By application of Lemma~\ref{lemmasigmaconsistent} this
  case holds.\\

  \noindent\textbf{Case} $\userule{H-PickLeft}$ Given that
  $\Gamma,\Sigma\infers t_1\Xor t_2:\Task\tau$ and $\Sigma\infers s$, then by
  \refrule{T-Xor} we have $\Gamma,\Sigma\infers t_1:\Task \tau$.\\

  \noindent\textbf{Case} $\userule{H-PickRight}$ Given that
  $\Gamma,\Sigma\infers t_1\Xor t_2:\Task\tau$ and $\Sigma\infers s$, then by
  \refrule{T-Xor} we have $\Gamma,\Sigma\infers t_2:\Task \tau$. \\

  \noindent\textbf{Case} $\userule{H-Next}$ Given that
  $\Gamma,\Sigma\infers t_1\Next e_2 :\Task\tau$ and $\Sigma\infers s$. Then by
  \refrule{T-Next}, we have $\Gamma,\Sigma\infers t_1:\Task\tau_1$ and
  $\Gamma,\Sigma\infers e_2:\tau_1\to\Task\tau$. Then by \refrule{T-Then} we
  obtain $\Gamma,\Sigma\infers t_1\Then e_2:\Task\tau$.\\

  \noindent\textbf{Case} $\userule{H-PassThen}$ Given that
  $\Gamma,\Sigma\infers t_1\Then e_2:\Task\tau$ and $\Sigma\infers s$,
  \refrule{T-Then} gives us that $\Gamma,\Sigma\infers t_1:\Task\tau_1$ and
  $\Gamma,\Sigma\infers e_2:\tau_1\to\Task\tau$. By the induction hypothesis, we
  know that $t_1,s\handle{i}t_1',s'$ also preserves and thus
  $\Gamma,\Sigma\infers t_1':\Task\tau_1$ and $\Gamma,Sigma\infers s'$. By
  \refrule{T-Then} we now obtain that
  $\Gamma,\Sigma\infers t_1'\Then e_2:\Task\tau$. \\

  \noindent\textbf{Case} $\userule{H-PassNext}$ Given that
  $\Gamma,\Sigma\infers t_1\Next e_2:\Task\tau$ and $\Sigma\infers s$,
  \refrule{T-Next} gives us that $\Gamma,\Sigma\infers t_1:\Task\tau_1$ and
  $\Gamma,\Sigma\infers e_2:\tau_1\to\Task\tau$. By the induction hypothesis, we
  know that $t_1,s\handle{i}t_1',s'$ also preserves and thus
  $\Gamma,\Sigma\infers t_1':\Task\tau_1$ and $\Gamma,Sigma\infers s'$. By
  \refrule{T-Next} we now obtain that
  $\Gamma,\Sigma\infers t_1'\Next e_2:\Task\tau$. \\

  \noindent\textbf{Case} $\userule{H-FirstAnd}$ Given that
  $\Gamma,\Sigma\infers t_1\And t_2:\Task(\tau_1\times\tau_2)$ and
  $\Sigma\infers s$, \refrule{T-And} gives us that
  $\Gamma,\Sigma\infers t_1:\Task\tau_1$ and
  $\Gamma,\Sigma\infers t_2:\Task\tau_2$. By the induction hypothesis, we know
  that $t_1,s\handle{i}t_1',s'$ also preserves and thus
  $\Gamma,\Sigma\infers t_1':\Task\tau_1$ and $\Sigma\infers s'$. Therefore by
  \refrule{T-Next} we obtain
  $\Gamma,\Sigma\infers t_1'\And t_2:\Task(\tau_1\times\tau_2)$.\\

  \noindent\textbf{Case} $\userule{H-SecondAnd}$ Given that
  $\Gamma,\Sigma\infers t_1\And t_2:\Task(\tau_1\times\tau_2)$ and
  $\Sigma\infers s$, \refrule{T-And} gives us that
  $\Gamma,\Sigma\infers t_1:\Task\tau_1$ and
  $\Gamma,\Sigma\infers t_2:\Task\tau_2$. By the induction hypothesis, we know
  that $t_2,s\handle{i}t_2',s'$ also preserves and thus
  $\Gamma,\Sigma\infers t_2':\Task\tau_2$ and $\Sigma\infers s'$. Therefore by
  \refrule{T-Next} we obtain
  $\Gamma,\Sigma\infers t_1\And t_2':\Task(\tau_1\times\tau_2)$.\\

  \noindent\textbf{Case} $\userule{H-FirstOr}$ Given that
  $\Gamma,\Sigma\infers t_1\Or t_2:\Task\tau$ and $\Sigma\infers s$,
  \refrule{T-Or} gives us that $\Gamma,\Sigma\infers t_1:\Task\tau$ and
  $\Gamma,\Sigma\infers t_2:\Task\tau$. By the induction hypothesis we know that
  $t_1,s\handle{i}t_1',s'$ also preserves, and therefore
  $\Gamma,\Sigma\infers t_1':\Task\tau$ and $\Sigma\infers s'$. By
  \refrule{T-Or} we now obtain $\Gamma,\Sigma\infers t_1'\Or t_2:\Task\tau$.\\

  \noindent\textbf{Case} $\userule{H-SecondOr}$ Given that
  $\Gamma,\Sigma\infers t_1\Or t_2:\Task\tau$ and $\Sigma\infers s$,
  \refrule{T-Or} gives us that $\Gamma,\Sigma\infers t_1:\Task\tau$ and
  $\Gamma,\Sigma\infers t_2:\Task\tau$. By the induction hypothesis we know that
  $t_2,s\handle{i}t_2',s'$ also preserves, and therefore
  $\Gamma,\Sigma\infers t_2':\Task\tau$ and $\Sigma\infers s'$. By T-Or we now
  obtain $\Gamma,\Sigma\infers t_1\Or t_2':\Task\tau$.\\

  \noindent\textbf{Case} $\userule{H-Appoint}$ Given that
  $\Gamma,\Sigma\infers u\At t:\Task\tau$, \refrule{T-Assign} gives us that
  $\Gamma,\Sigma\infers t:\Task\tau$. By the induction hypothesis, we know that
  $t,s\handle{i}t',s'$ also preserves, and therefore
  $\Gamma,\Sigma\infers u\At t'$ and $\Sigma\infers s'$.
\end{proof}

% \subsubsection{Theorem~\ref{thmprogressnorm}}
%
% \begin{proof} We prove this Theorem by induction on $e$.\\
%
%   \noindent\textbf{Case} $\userule{S-Edit}$ Given that $\Gamma,\Sigma\infers \Edit v : \Task \tau$, let $e''=\Edit v'$, $s''=s$ and $i=v':\tau$. Then H-Edit gives $\Edit v,s\handle{v'}\Edit v',s$. \\
%
%
%   \noindent\textbf{Case} $\userule{S-Fill}$ Given that $\Gamma,\Sigma\infers \Enter \tau : \Task \tau$, let $e''=\Edit v$, $s''=s$ and $i=v:\tau$. Then H-Fill gives $\Enter \tau,s\handle{v}\Edit v,s$. \\
%
%   \noindent\textbf{Case} $\userule{S-Update}$ Given that $\Gamma,\Sigma\infers \Update l : \Task \tau$, let $e''=\Update l$, $s''=s[l\mapsto v]$ and $i=v:\tau$. Then H-Update gives $\Update l ,s\handle{v}\Update l,s[l\mapsto v]$. \\
%
%   \noindent\textbf{Case} $\userule{S-Fail}$ Since $\Failing(\Fail)=\True$, the theorem holds in this case. \\
%
%   \noindent\textbf{Case} $\userule{S-Xor}$ \todo{Here we have an issue with the defintion of $\Failing$ and the sideconditions of external choice. These definitions should concur, so we can finish the proof, but $\Failing$ is a static property}\\
%
%   \noindent\textbf{Case} $\userule{S-Eval}$ By induction hypothesis we have that there exists an $e'''$, $s'''$ and $i$ such that $e'',s''\handle{i}e''',s'''$.\\
%
%   \noindent\textbf{Case} $\userule{S-ThenStay}$ By the induction hyposthesis we have that there exists an $e''$, $s''$ and $i$ such that $t_1',s'\handle{i}e'',s''$. Then by H-Pass, we know that we can apply the same $i$ to obtain $t_1\Then e_2,s'\handle{i} e''\Then e,s''$.\\
%
%   \noindent\textbf{Case} $\userule{S-ThenFail}$ By the induction hyposthesis we have that there exists an $e''$, $s''$ and $i$ such that $t_1',s'\handle{i}e'',s''$. Then by H-Pass, we know that we can apply the same $i$ to obtain $t_1\Then e_2,s'\handle{i} e''\Then e,s''$.\\
%
%   \noindent\textbf{Case} $\userule{S-ThenCont}$ By the induction hypothesis, we have that there exists an $e''$, $s''''$ and $i$ such that $t_2',s''\handle{i}e'',s''''$.\\
%
%   \noindent\textbf{Case} $\userule{S-OrLeft}$ By the induction hypothesis, we have that there exists an $e''$, $s'$ and $i$ such that $t_1',s\handle{i}e'',s''$.\\
%
%   \noindent\textbf{Case} $\userule{S-OrRight}$ By the induction hypothesis, we have that there exists an $e''$, $s'$ and $i$ such that $t_2',s\handle{i}e'',s''$.\\
%
%   \noindent\textbf{Case} $\userule{S-OrNone}$ \todo{   }\\
%
%   \noindent\textbf{Case} $\userule{S-Next}$ By the induction hypothesis, we have that there exists an $e''$, $s'$ and $i$ such that $t_1',s'\handle{i}e'',s''$. Then by the H-PassNext rule, we know that we can apply the same $i$ to obtain $t_1'\Next e_2,s'\handle{i}e'',s''$.\\
%
%   \noindent\textbf{Case} $\userule{S-And}$ By the induction hypothesis, we have that there exists an $e''$, $s'$ and $i$ such that $t_1',s'\handle{i}e'',s''$. Then by the H-FirstOr rule, we know that we can apply $\First i$ to obtain $t_1'\And t_2',s'\handle{\First i} t_1''\And t_2',s''$
%
% \end{proof}

% \subsubsection{Lemma~\ref{lem:stride-does-not-eval}}
%
% \begin{proof}
%   We prove Lemma~\ref{lem:stride-does-not-eval} by induction on $e$. First we
%   apply evaluation, and we then find ourselves in one of the cases below.\\
%
%   \noindent\textbf{Case} $\userule{S-ThenStay}$\\
%   and $\userule{S-ThenFail}$ The \refrule{E-Then} rule gives us that evaluation
%   has been applied to $t_1$. We therefore can apply the induction hypothesis,
%   from which we obtain that $t_1',s'\evaluate t_1'',s''$ then $t_1'=t_1''$ and
%   $s'=s''$. This, together with the \refrule{E-Then} rule, proves this case.\\
%
%   \noindent\textbf{Case} $\userule{S-ThenCont}$ By application of the induction
%   hypothesis, we obtain that $t_2',s'''\evaluate t_2'',s''''$ then $t_2'=t_2''$
%   and $s'''=s''''$, which immediately proves this case.\\
%
%   \noindent\textbf{Case} $\userule{S-OrLeft}$ The \refrule{E-Or} rule gives us
%   that evaluation has been applied to $t_1$. We therefore can apply the
%   induction hypothesis, from which we obtain that $t_1',s'\evaluate t_1'',s''$
%   then $t_1'=t_1''$ and $s'=s''$, which immediately proves this case.\\
%
%   \noindent\textbf{Case} $\userule{S-OrRight}$The \refrule{E-Or} rule gives us
%   that evaluation has been applied to $t_2$. We therefore can apply the
%   induction hypothesis, from which we obtain that $t_2',s'\evaluate t_2'',s''$
%   then $t_2'=t_2''$ and $s'=s''$, which immediately proves this case.\\
%
%   \noindent\textbf{Case} $\userule{S-OrNone}$ The \refrule{E-Or} rule gives us
%   that evaluation has been applied to both $t_1$ and $t_1$. We assumed that the
%   states do not change, so we know that $s=s'=s''$. We can therefore safely
%   apply the induction hypothesis twice, to obtain that
%   $t_1',s''\evaluate t_1'',s'''$ then $t_1'=t_1''$ and $s''=s'''$, and
%   $t_2',s'''\evaluate t_2'',s''''$ then $t_2'=t_2''$ and $s'''=s''''$. This,
%   together with the \refrule{E-Or} rule proves this case.\\
%
%   \noindent\textbf{Case} $\userule{S-Edit}$ by definiton, $\Edit v$ is a value
%   with respect to evaluation.\\
%
%   \noindent\textbf{Case} $\userule{S-Fill}$ by definition, $\Enter \tau$ is a
%   value with respect to evaluation.\\
%
%   \noindent\textbf{Case} $\userule{S-Update}$ by definiton, $\Update l$ is a
%   value with respect to evaluation.\\
%
%   \noindent\textbf{Case} $\userule{S-Fail}$ by definiton, $\Fail$ is a value
%   with respect to evaluation.\\
%
%   \noindent\textbf{Case} $\userule{S-Xor}$ by definition, $e_1\Xor e_2$ is a
%   value with respect to evaluation.\\
%
%   \noindent\textbf{Case} $\userule{S-Next}$ The \refrule{E-Next} rule gives us
%   that evaluation has been applied to $t_1$. We therefore can apply the
%   induction hypothesis, from which we obtain that $t_1',s'\evaluate t_1'',s''$
%   then $t_1'=t_1''$ and $s'=s''$. This, together with the \refrule{E-Next} rule,
%   proves this case.\\
%
%   \noindent\textbf{Case} $\userule{S-And}$ The \refrule{E-And} rule gives us
%   that evaluation has been applied to both $t_1$ and $t_1$. We assumed that the
%   states do not change, so we know that $s=s'=s''$. We can therefore safely
%   apply the induction hypothesis twice, to obtain that
%   $t_1',s''\evaluate t_1'',s'''$ then $t_1'=t_1''$ and $s''=s'''$, and
%   $t_2',s'''\evaluate t_2'',s''''$ then $t_2'=t_2''$ and $s'''=s''''$. This,
%   together with the \refrule{E-And} rule proves this case.\\
%
%   \noindent\textbf{Case} $\userule{S-Appoint}$ The \refrule{E-Appoint} rule
%   gives us that evaluation has been applied to $t$. We therefore can apply the
%   induction hypothesis, from which we obtain that $t',s'\evaluate t'',s''$ then
%   $t'=t''$ and $s'=s''$. This, together with the \refrule{E-Appoint} rule,
%   proves this case.
%
% \end{proof}





\subsubsection{Theorem~\ref{thm:failing}}

\begin{proof}

  We prove Theorem~\ref{thm:failing} by induction on $e'$.\\

  \noindent\textbf{Case} $e = \Fail$\\
  $\Failing(\Fail,s)=\True$, and there is no handling rule that applies to fail.\\

  \noindent\textbf{Case} $e = \Edit v$\\
   Given that $\Gamma,\Sigma\infers\Edit v:\Task\tau$, $\Failing(\Edit v,s)=\False$, and there exists an input $i$, namely $v':\tau$.\\

  \noindent\textbf{Case} $e = \Enter \tau$\\
  $\Failing(\Enter \tau)=\False$, and there exists an in put i, namely $v:\tau$.\\

  \noindent\textbf{Case} $e = \Update l$\\
  Given that $\Gamma,\Sigma\infers\Update l:\Task\tau$, $\Failing(\Update l,s)=\False$, and there exists an input $i$, namely $v:\tau$.\\

  \noindent\textbf{Case} $e = t_1\Then e_2$\\
   $\Failing(t_1\Then e_2,s)=\Failing(t_1,s)$. If there exitsts an $i$ for $t_1$, then this $i$ also applies to $t_1\Then e_2$. This case therefore holds by the induction hypothesis.\\

  \noindent\textbf{Case} $e = t_1\Next e_2$\\
   $\Failing(t_1\Next e_2,s)=\Failing(t_1,s)$. If there exitsts an $i$ for $t_1$, then this $i$ also applies to $t_1\Next e_2$. This case therefore holds by the induction hypothesis.\\

  \noindent\textbf{Case} $e = e_1\Xor e_2$\\
   We normalise the two expressions first, $e_1,s\stride t_1,s'$, $e_2,s\stride t_2,s'$ and we can then be in two situations. One, we can have that $\Failing(t_1,s')$ and $\Failing(t_2,s')$ are both true. If that is so, then by definiton, we have both $\Failing(e_1\Xor e_2,s)$ and no rule of the hanlding semantics applies, and therefore there exists no input for this case.\\
                                           Or we are in the situation where one or both of the two sub expressions does not fail. In that case, we know that $\Failing(e_1\Xor e_2,s)$ does not hold, and that at least one of the handling rules applies. Therefore, there must be an input $i$, namely $\Left$, $\Right$ or both.

  \noindent\textbf{Case} $e = t_1\And t_2$\\
   We can again find ourselfs in one of two situations. In the first case, both sub expressions fail, $\Failing(t_1,s)$ and $\Failing(t_2,s)$. In that case, we know that $\Failing(t_1\And t_2,s)$ also fails by definition. By the induction hypothesis, we know that for both $t_1$ and $t_2$ there is no input that can be handled. Since the only two rules for $\And$ that handle input just pass this input on to one of the two expressions, we know that indeed no $i$ applies.\\
                                           In the case that one or both sub expressions do not fail, then by definition $t_1\And t_2$ not failing under $s$. Again by induction hypothesis, we know that for one or both of the expressions, there exits an $i$ that can be handled. Then by H-FirstAnd and H-SecondAnd, we know that we can pass this $i$, by prefixing it with either $\First$ or $\Second$.\\

  \noindent\textbf{Case} $e = t_1\Or t_2$\\
  We can again find ourselfs in one of two situations. In the first case, both sub expressions fail, $\Failing(t_1,s)$ and $\Failing(t_2,s)$. In that case, we know that $\Failing(t_1\Or t_2,s)$ also fails by definition. By the induction hypothesis, we know that for both $t_1$ and $t_2$ there is no input that can be handled. Since the only two rules for $\Or$ that handle input just pass this input on to one of the two expressions, we know that indeed no $i$ applies.\\
                                           In the case that one or both sub expressions do not fail, then by definition $t_1\Or t_2$ not failing under $s$. Again by induction hypothesis, we know that for one or both of the expressions, there exits an $i$ that can be handled. Then by H-FirstOr and H-SecondOr, we know that we can pass this $i$, by prefixing it with either $\First$ or $\Second$.\\

  \noindent\textbf{Case} $e = u \At t$\\
   This case follows directly from applying the induction hypothesis, since $\Failing(u \At t,s)=\Failing(t,s)$ and any input that applies to $t$ also applies to $u \At t$.\\
\end{proof}

\subsubsection{Theorem~\ref{thm:safety-i}}
\begin{proof}

  \noindent\textbf{Case} $e=\Edit v:\Task\tau, i= v':\tau$\\
   Given that $\userule{H-Change}$, we have by definition that $\Inputs(\Edit v:\Task\tau,s) = \set{v':\tau,\Empty}$, which includes $v':\tau$.

  \noindent\textbf{Case} $e=\Edit v:\Task\tau, i= \Empty$\\
   Given that $\userule{H-Empty}$, we have by definition that $\Inputs(\Edit v:\Task\tau,s) = \set{v':\tau,\Empty}$, which includes $\Empty$.\\

  \noindent\textbf{Case} $e=\Enter \tau, i= v':\tau$\\
   Given that $\userule{H-Fill}$, we have by definiton that $\Inputs(\Enter \tau,s) = \set{v':\tau }$, which includes $v':\tau$.\\

  \noindent\textbf{Case} $e=\Update l:\Task\tau, i= v':\tau$\\
   Given that $\userule{H-Update}$, we have by definiton that $\Inputs(\Update l:\Task\tau,s)= \set{v':\tau }$, which includes $v':\tau$.\\

  \noindent\textbf{Case} $e=t_1\Xor t_2 , i= \Left$\\
   Given that $\userule{H-PickLeft}$, we have by definition that $\Inputs(t_1\Xor t_2,s) = \set{ \Left,\Right}$, which includes $\Left$.\\

  \noindent\textbf{Case} $e=t_1\Xor t_2 , i= \Right$\\
   Given that $\userule{H-PickRight}$, we have by definition that $\Inputs(t_1\Xor t_2,s) = \set{ \Left,\Right}$, which includes $\Right$.\\

  \noindent\textbf{Case} $e=t_1\Next e_2 , i= \Continue $\\
   Given that $\userule{H-Next}$, we have by definition that $\Inputs(t_1\Next e_2,s) = \Inputs(t_1,s) \cup \set{\Continue\mid \Value{(t_1,s)} = v_1 \wedge \neg\Failing{(e_2 v_1,s\stride)}}$. If the H-Next rule applies, this means that the conditions $\Value{(t_1,s)} = v_1 \wedge \neg\Failing{(e_2 v_1,s\stride)}$ are fulfilled, and therefore $\Continue$ is contained. \\

  \noindent\textbf{Case} $e=t_1\Next e_2 , i\neq\Continue$\\
   Given that $\userule{H-PassNext}$, we have by definition that $\Inputs(t_1\Next e_2,s) = \Inputs(t_1,s) \cup \set{\Continue\mid \Value{(t_1,s)} = v_1 \wedge \neg\Failing{(e_2 v_1,s\stride)}}$. By the induction hypthesis, we have that $i\in \Inputs(t_1,s)$, and by definiton of $\Inputs$, $i$ is therefore also included in this case.\\

  \noindent\textbf{Case} $e=t_1\Then e_2 , i$\\
   Given that $\userule{H-PassThen}$, we have by definition that $\Inputs(t_1\Then e_2,s)= \Inputs(t_1,s)$. By the induction hypthesis, we have that $i\in \Inputs(t_1,s)$, and by definiton of $\Inputs$, $i$ is therefore also included in this case.\\

  \noindent\textbf{Case} $e=t_1\And t_2 , i=\First i$\\
   Given that $\userule{H-FirstAnd}$ we have by definition that $\Inputs(t_1\And t_2,s) = \set{\First\ i \mid i \in \Inputs(t_1,s)} \cup \set{\Second\ i \mid i \in \Inputs(t_2,s)}$. By the induction hypothesis, we have that $i\in\Inputs(t_1,s)$, and by definition of $\Inputs$, $\First i$ is therefore als included in this case.\\

  \noindent\textbf{Case} $e=t_1\And t_2 , i=\Second i$\\
   Given that $\userule{H-SecondAnd}$ we have by definition that $\Inputs(t_1\And t_2) = \set{\First\ i \mid i \in \Inputs(t_1,s)} \cup \set{\Second\ i \mid i \in \Inputs(t_2,s)}$. By the induction hypothesis, we have that $i\in\Inputs(t_2,s)$, and by definition of $\Inputs$, $\Second i$ is therefore als included in this case.\\

  \noindent\textbf{Case} $e=t_1\Or t_2 , i=\First i$\\
   Given that $\userule{H-FirstOr}$ we have by definition that $\Inputs(t_1\Or t_2,s) = \set{\First\ i \mid i \in \Inputs(t_1,s)} \cup \set{\Second\ i \mid i \in \Inputs(t_2,s)}$. By the induction hypothesis, we have that $i\in\Inputs(t_1,s)$, and by definition of $\Inputs$, $\First i$ is therefore als included in this case.\\

  \noindent\textbf{Case} $e=t_1\Or t_2 , i=\Second i$\\
   Given that $\userule{H-FirstOr}$ we have by definition that $\Inputs(t_1\Or t_2,s) = \set{\First\ i \mid i \in \Inputs(t_1,s)} \cup \set{\Second\ i \mid i \in \Inputs(t_2,s)}$. By the induction hypothesis, we have that $i\in\Inputs(t_2,s)$, and by definition of $\Inputs$, $\Second i$ is therefore als included in this case.\\

  \noindent\textbf{Case} $e=u\At t, i$\\
   Given that $\userule{H-Appoint}$, we can directly apply the induction hypothesis to prove this case.\\
\end{proof}
