% !TEX root=../icfp2019.tex



\section{Introduction}

Task-Oriented Programming (\TOP) is a programming paradigm aimed at writing interactive multi-user applications in a declarative way \cite{conf/ppdp/PlasmeijerLMAK12}.
\TOP has two aspects.
Firstly, \TOP defines the primitive building blocks that are useful for high-level descriptions of how users collaborate with each other and with applications.
These building blocks are \emph{editors}, \emph{composition}, and \emph{shared data}.
Secondly, \TOP defines a type-driven way to generate applications, including graphical user interfaces, from workflows modelled with said building blocks.

In this paper we study the essence of \TOP.
We claim that at the heart of \TOP lie modular interactive workflows.
Workflows because \TOP programs coordinate collaboration between people.
Interactive because progress of \TOP programs is driven by user input.
Modular because the embedding in a strongly-typed functional language allows powerful abstractions using type-driven techniques and higher-order functions.

Interactive workflows are called \emph{tasks}.
Tasks stand for units of work in the real world, assigned to people.
People can work together in a number of ways, and this is reflected in \TOP by task combinators.
There is sequential composition, parallel composition, and choice.
People need to communicate in order to engage in these forms of collaboration.
This is reflected in \TOP by three kinds of communication mechanisms.
There is data flow \emph{alongside} control flow, where the result of a task is passed onto the next.
There is data flow \emph{across} control flow, where information is shared between multiple tasks.
Finally, there is communication with the \emph{outside} world, where information is entered into the system via input events.
The end points where the outside world interacts with \TOP applications are called editors.
In generated applications, editors can take many forms, like input fields, selection boxes, or map widgets.

iTasks is an implementation of \TOP, in the form of a shallowly embedded domain-specific language in the lazy functional programming language Clean.
It is a library that provides editors and monadic combinators.
iTasks uses the generic programming facilities of Clean to derive web applications from workflow models.
It has been used to model an incident management tool for the Dutch coast guard~\cite{conf/iscram/LijnseJP12}.
Also it has been used numerous times to prototype ideas for Command and Control~\cite{theses/nlda/Kool17, theses/radboud/Stutterheim17}, and in a case study for the Dutch tax authority \cite{conf/sfp/StutterheimAP17}.
