% !TEX root=../main.tex

In this we investigate a simple system to model interactive workflows using \emph{tasks}.
It is an (informal) introduction to the developed language,
explaining design choices and giving examples on the way.
By using five basic constructs,
each with its own characteristics,
we will show how to express task based systems.
The author claims this language is capable of expressing the majority of programs that can be written in the \ITASKS \cite{plasmeijer2011itasks} system.

In the next sections we will gradually develop a language, dubbed \Lang, for interactive workflow modelling.
Each section will introduce one of the following basic constructs,
making our language more powerful after every addition:
\begin{itemize}
  \item interactive \emph{editors} with a value ($\Edit v$) and without a value ($\Fill$);
  \item parallel composition or \emph{pairing} of tasks ($t_1 \And t_2$);
  \item sequential composition or \emph{stepping} of tasks ($t \Then e$);
  \item making a \emph{choice} between tasks ($t_1 \Or t_2$) and the \emph{failure} task ($\Fail$);
\end{itemize}
Additionally, we will discuss two convenience combinators and one extension using shared data:
\begin{itemize}
  \item giving the user an explicit choice ($t_1 \ExOr t_2$);
  \item continuing to the next task on a user's request ($t_1 \ExThen t_2$);
  \item watching and changing shared data ($\Change l$).
\end{itemize}

Compared to previous attempts,
this system distinguishes itself by the following points:
\begin{itemize}
  \item
    There is a clear distinction between the underlaying \emph{host language}
    and the task layer (\emph{object language}) on top of it.
  \item
    There is no notion of values which can be stable or unstable.
    Our language only has pure values lifted into the task world ($\Edit e$).
  \item
    Tasks do not need to be identified by a task identifier.
    Events are based on the \emph{structure} of the task at hand.
  \item
    Events are not handled by the user, but are build into the semantics.
  \item
    Entering some information into the system is not a one-shot action.
    Editors keep asking for input continually.
    Thus repeatedly asking for information does not need to be modelled with recursion.
    The next task is only executed under preprogrammed internal conditions or external action by the user.
  \item
    We make use of multiple semantic functions that describe different aspects of the system.
    At the core, we assume a big step semantics for the host language.\footnote{
      We use a $\lambda$-calculus with some simple extensions like $\If{}{}{}$, pairs, and references.}
    On top of this there are \emph{normalisation} semantics of tasks,
    performed after each event handle step,
    and the \emph{handling} of events itself.
    Next to these three core semantics,
    there are also semantic functions querying the current value of a task,
    giving enabled options,
    or producing a rudimentary user interface.
\end{itemize}

The remaining of this report is structured as follows.
In the first section we set up our our host language,
the $\lambda$-calculus extended with basic arithmetic.
Thereafter, in the next seven sections, we will gradually define our object language,
extending the properties of our host language when needed.
After some sections we make a small intermezzo,
showing the current properties and laws of our system.
We strive to keep the host language as small as possible.
At the end of the report,
we will see an overview of our our system,
including all static and dynamic rules defined.
