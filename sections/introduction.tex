% !TEX root=../pldi2019.tex

\section{Introduction}

Task-Oriented Programming (TOP) is a programming paradigm aimed at writing interactive multi-user applications in a declarative way.
TOP has two aspects.
Firstly, TOP defines the primitive building blocks that are useful for high-level descriptions of how users collaborate with each other and with applications.
These building blocks are \emph{editors}, \emph{combinators}, and \emph{shared data sources}.
Secondly, TOP defines a type-driven way to derive applications including graphical user interfaces from workflows modelled with said building blocks.

Editors represent end-points where users directly interact with applications.
In a derived application, editors can take many forms, like input fields, selection boxes, or map widgets.
Combinators represent control flow and stand for the ways work can be coordinated in a collaborative environment.
The three most important combinators are sequential composition, parallel composition, and choice.

iTasks is an implementation of TOP, in the form of an embedded domain-specific language (EDSL) in the lazy functional programming language Clean.
It comes in the form of a library that provides editors and monadic combinators.
iTasks uses the generic programming facilities of Clean to derive web applications from workflow models.
