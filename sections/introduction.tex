% !TEX root=../pldi2019.tex



\section{Introduction}

Task-Oriented Programming (\TOP) is a programming paradigm aimed at writing interactive multi-user applications in a declarative way \cite{conf/ppdp/PlasmeijerLMAK12}.
\TOP has two aspects.
Firstly, \TOP defines the primitive building blocks that are useful for high-level descriptions of how users collaborate with each other and with applications.
These building blocks are \emph{editors}, \emph{combinators}, and \emph{shared data sources}.
Secondly, \TOP defines a type-driven way to generate applications, including graphical user interfaces, from workflows modelled with said building blocks.

In this paper we study the essence of Task-Oriented Programming.
We claim that at the heart of \TOP lie modular interactive workflows.
Workflows because \TOP programs coordinate collaboration between people.
Interactive because progress of \TOP programs is driven by user input.
Modular because the embedding in a strongly-typed functional language allows powerful abstractions using type-driven techniques and higher-order functions.

Interactive workflows are called \emph{tasks}.
Tasks stand for units of work in the real world, assigned to people.
People can work together in a number of ways, and this is reflected in \TOP by task combinators.
There is sequential composition, parallel composition, and choice.
People need to communicate in order to engage in these forms of collaboration.
This is reflected in \TOP by three kinds of communication mechanisms.
There is data flow alongside control flow, where the result of a task is passed onto the next.
There is data flow across control flow, where information is shared between multiple tasks.
Finally, there is communication is with the outside world, where information is entered into the system via input events.
The end points where the outside world interacts with applications are called editors.
In generated applications, editors can take many forms, like input fields, selection boxes, or map widgets.

iTasks is an implementation of \TOP, in the form of an embedded domain-specific language in the lazy functional programming language Clean.
It comes in the form of a library that provides editors and monadic combinators.
iTasks uses the generic programming facilities of Clean to derive web applications from workflow models.

iTasks has been used to implement an incident management tool for the Dutch coast guard \cite{conf/iscram/LijnseJP12}.
Furthermore it has been used numerous times to prototype ideas for Command and Control \cite{theses/nlda/Kool17, theses/radboud/Stutterheim17}, and in a case study for the Dutch tax authority \cite{conf/sfp/StutterheimAP17}.

iTasks has many features, and its basic combinators are versatile and powerful.
Simpler combinators are implemented by restricting the powerful ones.
This is useful for everyday programming, where having lots of functionality at one's fingertips is convenient and efficient.
This approach does not lend itself well to formal treatment however, as it obscures what the essential features of \TOP are, and how they interact.



\paragraph{Motivations}
\begin{itemize*}
  \item Formal basis needed for research.
  \item Prove properties of interactive workflows.
  \item Clearly specify the basic concepts needed to model workflows
    and separate these from general purpose programming concepts.
  \item Compare \TOP with processes and process algebra's like \CSP and \CCS.
\end{itemize*}


\paragraph{Contributions}
\begin{itemize*}
  \item A calculus to model workflows, which can be used in a modular way and are driven by user inputs.
  \item Embed this calculus in a simply typed $\lambda$-calculus.
  \item Its operational semantics, presented and defined in a layered way.
  \item Semantic observations on the calculus which can be used to:
  \begin{itemize*}
    \item calculate stuck configurations,
    \item associate a value with a task,
    \item draw a user interface (\UI) for a task,
    \item calculate input possibilities for the user.
  \end{itemize*}
  \item A clear separation between \TOP principles and general purpose programming constructs.
\end{itemize*}


\paragraph{Background}


The goal of this paper is to develop a calculus that serves as a formal foundation for \TOP.
The calculus is called Top Hat, written \TOPHAT.
Starting with combinators that have a single responsibility, we show how more complicated combinators can be expressed in terms of those.
\TOPHAT models a subset of iTasks, striking a balance between covering the features of the actual implementation and being suitable for formal treatment.
In this way we identify the essential basic features of \TOP and study them in isolation.
This approach makes the process algebra-like flavour of iTasks precise, and allows comparison with other works in the field.
