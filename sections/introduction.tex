% !TEX root=../icfp2019.tex



\section{Introduction}



\subsection{Task-oriented programming}

Many applications these days are developed to support workflows in institutions and businesses.
Take for example expenses declarations, order processing, and emergency management.
Not all of these workflows happen inside organisations.
Some occur on the boundary with customers, like flight bookings or tax returns.
What they all have in common is,
that they need to interact with different people (end users, tax officers, customers, etc.)
and they use information from multiple sources (input forms, databases, sensors, etc.).

We call such interactive units of work based on information sources \emph{tasks}.
Tasks model collaboration between users and are driven by work users do.
Users could be people out in the field or sitting behind their desks,
as well as machines doing calculations or fetching data.

Task-oriented programming (\TOP) is a programming paradigm which searches for the sweet spot between faithful modelling workflows
and rapid prototyping of multi-user web applications supporting this workflow.
% Task-oriented programming (\TOP) is a programming paradigm to support these ways of working.
The paradigm takes collaboration patterns as the base of modelling
and anticipates on user's need to interact and share information.
A language that supports \TOP should allow to specify tasks from real world scenarios,
and be able to generate multi-user web applications to support these scenarios.
By choosing the right level of abstraction,
\TOP focusses on creating workflow specifications,
while automatically providing common development duties like designing \GUI\ s, connecting to databases, and communicating between servers and client.

Currently, we know of two frameworks using \TOP: \ITASKS and \MTASKS.
\ITASKS is a full fledged \TOP framework able to generate rich client and server applications from a single source in the functional programming language Clean.
\MTASKS is a subset of \ITASKS,
focussing on \IOT devices and deployment on micro controllers.
Both have been used to model real world scenarios like \fixme{add}.

In this paper we tend to identify the essence of \TOP in formal as well as an informal way.
These core principles lead to a formal system for \TOP called \TOPHAT.
\TOPHAT paves the way to formal reasoning about applications written using the \TOP paradigm,
like those in \ITASKS and \MTASKS.

\TOP applications give rise to concepts like concurrency, synchronisation, \GUI-programming and program generation.
In this paper we identify common concepts with already existing languages and frameworks,
and describe their similarities and differences.



\subsection{Contributions}

Our contributions to workflow modelling, functional programming language design, and rapid application development are as follows.

\begin{enumerate}

  \item
    We informally describe the essential concepts of task-oriented programming (\TOP).
    We have a strong focus on modelling using collaboration
    while keeping a constant desire to rapidly generate executable applications into account.

  \item
    We present a formal calculus for \TOP, arising naturally from above essential concepts.
    Hereby we provide ground work to apply formal reasoning about \TOP specifications in future work.

  \item
    Using both informal and formal descriptions, we compare \TOP and \TOPHAT with multiple related work in the area,
    ranging from business process modelling languages, to process algebras and reactive programming frameworks.

\end{enumerate}



\subsection{Structure}

% !TEX root=../pldi2019.tex



% \paragraph{Overview}

In Section~\ref{sec:example} we demonstrate the functionality of \TOPHAT by means of an example,
Section~\ref{sec:comparison} compares \TOP and \TOPHAT to \CSP.
Section~\ref{sec:language} introduces the \TOPHAT calculus syntax
and Section~\ref{sec:semantics} the semantics.
Then in Section~\ref{sec:properties} we show that certain properties hold for the calculus.
We take a look at related work in Section~\ref{sec:relatedwork}
and conclude in Section~\ref{sec:conclusions}.




\subsection{Old}

% !TEX root=../icfp2019.tex



\section{Introduction}

Task-Oriented Programming (\TOP) is a programming paradigm aimed at writing interactive multi-user applications in a declarative way \cite{conf/ppdp/PlasmeijerLMAK12}.
\TOP has two aspects.
Firstly, \TOP defines the primitive building blocks that are useful for high-level descriptions of how users collaborate with each other and with applications.
These building blocks are \emph{editors}, \emph{composition}, and \emph{shared data}.
Secondly, \TOP defines a type-driven way to generate applications, including graphical user interfaces, from workflows modelled with said building blocks.

In this paper we study the essence of \TOP.
We claim that at the heart of \TOP lie modular interactive workflows.
Workflows because \TOP programs coordinate collaboration between people.
Interactive because progress of \TOP programs is driven by user input.
Modular because the embedding in a strongly-typed functional language allows powerful abstractions using type-driven techniques and higher-order functions.

Interactive workflows are called \emph{tasks}.
Tasks stand for units of work in the real world, assigned to people.
People can work together in a number of ways, and this is reflected in \TOP by task combinators.
There is sequential composition, parallel composition, and choice.
People need to communicate in order to engage in these forms of collaboration.
This is reflected in \TOP by three kinds of communication mechanisms.
There is data flow \emph{alongside} control flow, where the result of a task is passed onto the next.
There is data flow \emph{across} control flow, where information is shared between multiple tasks.
Finally, there is communication with the \emph{outside} world, where information is entered into the system via input events.
The end points where the outside world interacts with \TOP applications are called editors.
In generated applications, editors can take many forms, like input fields, selection boxes, or map widgets.

iTasks is an implementation of \TOP, in the form of a shallowly embedded domain-specific language in the lazy functional programming language Clean.
It is a library that provides editors and monadic combinators.
iTasks uses the generic programming facilities of Clean to derive web applications from workflow models.
It has been used to model an incident management tool for the Dutch coast guard~\cite{conf/iscram/LijnseJP12}.
Also it has been used numerous times to prototype ideas for Command and Control~\cite{theses/nlda/Kool17, theses/radboud/Stutterheim17}, and in a case study for the Dutch tax authority \cite{conf/sfp/StutterheimAP17}.

