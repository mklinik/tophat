% !TEX root=../main.tex

\section{Introduction}

In the next sections we will gradually develop a language to model interactive workflow.
Each section will introduce one of the following basic constructs,
making our language more powerful after every addition:
\begin{itemize}
  \item interactive \emph{editors} with a value ($\Edit v$) and without a value ($\Fill \tau$);
  \item parallel composition or \emph{pairing} of tasks ($t_1 \And t_2$);
  \item sequential composition or \emph{stepping} of tasks ($t_1 \Then e_2$);
  \item making a \emph{choice} between tasks ($t_1 \Or t_2$);
  \item the \emph{failure} task ($\Fail$);
\end{itemize}
Additionally, we will discuss two convenience combinators and one extension using shared data:
\begin{itemize}
  \item giving users an explicit choice ($t_1 \ExOr t_2$);
  \item continuing to the next task upon users' requests ($t_1 \ExThen e_2$);
  \item watching and changing shared data ($\Change l$).
\end{itemize}

Compared to previous attempts \cite{conf/ifl/KoopmanPA08,conf/ppdp/PlasmeijerLMAK12,theses/radboud/VinterHviid18},
our approach distinguishes itself by the following points:
\begin{itemize}
  \item
    There is a clear distinction between the underlaying \emph{host language}
    and the task layer (\emph{object language}) on top of it.
    % For example, \textcite{conf/ppdp/PlasmeijerLMAK12} present a reference implementation of task oriented programming
    % as an embedded domain specific language in Clean \cite{manuals/PlasmeijerE98} which blurs the lines between the host and object language.
  \item
    There is no notion of values which can be stable or unstable.
    Our language has editors which can be valued ($\Edit v$) or unvalued ($\Fill \tau$),
    but values themselves do not have special distinguishing features and are just values,
    as in every other lambda calculus or functional programming language.
  \item
    Tasks are never done!
    A value lifted into the task world can always interactively be changed by end users.
  \item
    All tasks are, in the end, about interaction with end users.
    Not because construct introduced in our language has a way to interact with the user,
    but the leaves are.\footnote{
      There are two leaves in our language: editors ($\Edit v$ and $\Fill \tau$) and fail ($\Fail$).
      Editors are the main focus of user interaction.
      Users can enter and change information using an editor.
      Fail is a special case.
      It acts like a black hole regarding interaction and will swallow every event users will throw at it.
      All other language constructs are combinators.
    }
  \item
    Entering some information into a system is not a one-shot action.
    Editors keep asking for input continually.
    Thus repeatedly asking for information does not need to be modelled with recursion.
    The next task is only executed under preprogrammed internal conditions or external actions by users.
  \item
    Tasks do not need to be identified by a task identifier.
    Sending events is based on the \emph{structure} of the task at hand.
  \item
    Events are not accessible by users, they are build into the semantics.
  \item
    Semantically, tasks do not return a value \emph{and} a continuation
    but \emph{only} a continuation.
    Obtaining the current value of a task is an \emph{observation},
    implemented by an additional semantic function.
    Looking at tasks in this way mitigates the problem of a stream-like type which can not be made into a monad
    (as described in \textcite{theses/radboud/VinterHviid18})
    and make sure some tasks, such as a step, do not have a value.
  \item
    We make use of multiple semantic functions that describe different aspects of the language.
    At the core, we assume a big step \emph{evaluation} semantics for the host language.\footnote{
      We use a $\lambda$-calculus with some primitive types and simple extensions like $\If{}{}{}$, pairs, and references.}
    On top of this there is \emph{normalisation} of tasks
    and \emph{handling} of events.
    Normalisation is done before starting the main event loop
    and after handling an event.
    Next to these three core semantics,
    there are also semantic functions querying the current \emph{value} of a task,
    giving enabled \emph{actions},
    or producing a (rudimentary) \emph{user interface}.
\end{itemize}

The remaining of this report is structured as follows.
In the first section we set up our our host language,
the $\lambda$-calculus extended with primitive types and operations.
Thereafter, in the next \fixme{seven} sections, we will gradually define our object language,
extending the properties of our host language when needed.
After some of them,
we will make a small intermezzo,
showing the current properties and laws of our language.
We strive to keep the host language as small as possible.
At the end of the report,
we will see an overview of our our language,
summarising all static and dynamic rules defined along the way.
