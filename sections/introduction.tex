% !TEX root=../icfp2019.tex



\section{Introduction}

%XXX: Alejandro:
% The introduction seems to be more about why Task-Oriented Programming is nice and useful.
% Personally, I would like it to tell me another story:
% given that some people find TOP useful (as shown by the many examples),
% why should be care about formalising it and which challenges does it pose?
% I think this is what people from ICFP would care about more.
%NOTE: Ik hoop dat de `Challenges` sectie dit wat opvangt. --TS

Many applications these days are developed to support workflows in institutions and businesses.
Take for example expense declarations, order processing, and emergency management.
Some of these workflows occur on the boundary between organisations and customers,
like flight bookings or tax returns.
What they all have in common is
that they need to interact with different people (end users, tax officers, customers, etc.)
and they use information from multiple sources (input forms, databases, sensors, etc.).

\subsection{Tasks}

We call interactive units of work based on information sources \emph{tasks}.
% Tasks stand for units of work in the real world, assigned to people.
Tasks model real world collaboration between users,
are driven by work users do,
and are assigned to some user.
Users could be people out in the field or sitting behind their desks,
as well as machines doing calculations or fetching data.



\subsection{Task-oriented programming}
\label{sec:top}

%XXX: Alejandro: Not entirely convincing
% Hij heeft het hier volgens mij over de titel van de sectie en het woord aims in de eerste zin. Misschien moeten we een hardere claim doen hier? --NN
% Ik heb `aims` vervangen door `targets`. Doet dat wat? --TS

% Task-Oriented Programming (\TOP) is a programming paradigm aimed at writing interactive multi-user applications in a declarative way \cite{conf/ppdp/PlasmeijerLMAK12}.
Task-oriented programming (\TOP) is a programming paradigm which targets the sweet spot between faithful modelling workflows
and rapid prototyping of multi-user web applications supporting these workflows \cite{conf/ppdp/PlasmeijerLMAK12}.
% Task-oriented programming (\TOP) is a programming paradigm to support these ways of working.
\TOP focusses on modelling collaboration patterns.
This gives rise to user's need to interact and share information.
Next to that, \TOP automatically provides solutions to common development jobs like designing \GUI\ s, connecting to databases, and communicating between servers and clients.

Therefore,
a language that supports \TOP should choose the right level of abstraction to support two things.
% \TOP has two aspects.
% First, it should allow to specify tasks from real world scenarios.
Firstly, it should provide primitive building blocks that are useful for high-level descriptions of how users collaborate,
with each other and with machines.
These building blocks are: \emph{editors}, \emph{composition}, and \emph{shared data}.
% Second, it should be able to generate multi-user web applications to support these scenarios.
Secondly, it should be able to generate applications, including graphical user interfaces, from workflows modelled with said building blocks.

Users can work together in a number of ways, and this is reflected in \TOP by task compositions.
There is \emph{sequential} composition, \emph{parallel} composition, and \emph{choice}.
Users need to communicate in order to engage in these forms of collaboration.
This is reflected in \TOP by three kinds of communication mechanisms.
There is data flow \emph{alongside} control flow, where the result of a task is passed onto the next.
There is data flow \emph{across} control flow, where information is shared between multiple tasks.
Finally, there is communication with the \emph{outside} world, where information is entered into the system via input events.
The end points where the outside world interacts with \TOP applications are called editors.
In generated applications, editors can take many forms, like input fields, selection boxes, or map widgets.



\subsection{Utilisation}


Currently, we know of two frameworks implementing \TOP: \ITASKS and \MTASKS.

\ITASKS is an implementation of \TOP, in the form of a shallowly embedded domain-specific language in the lazy functional programming language Clean.
It is a library that provides editors, monadic combinators, and shared data sources.
\ITASKS uses the generic programming facilities of Clean to derive rich client and server applications from a single source.
It has been used to model an incident management tool for the Dutch coast guard~\cite{conf/iscram/LijnseJP12}.
Also it has been used numerous times to prototype ideas for Command and Control~\cite{theses/nlda/Kool17, theses/radboud/Stutterheim17}, and in a case study for the Dutch tax authority~\cite{conf/sfp/StutterheimAP17}.

\MTASKS is a subset of \ITASKS,
focussing on \IOT devices and deployment on micro controllers.
It has been used to control home thermostats and other home automation applications \cite{koopman2018task}.
%
Both implementations currently lack formal semantics which are suited to prove properties about tasks.



\subsection{Challenges}

% \ITASKS is the de-facto reference implementation of \TOP.
Both \ITASKS and \MTASKS have been designed for developing real-world applications.
They are constantly being extended and improved with this goal in mind.
The different variations of task combinators and the details that come with real-world requirements,
make it hard to see what the essence of \TOP is.
% One of the defining aspects of both \ITASKS and \MTASKS is their embedding in a functional programming language.
% This creates a synergy where they profit from the expressivity of Clean, while enhancing the language with functionality for user interaction.
Also, the tight integration of both frameworks with the Clean, makes it difficult to see where the boundaries are.
This makes formal reasoning about \TOP programs impossible.

In this paper, we want to take a step back and look at the spirit of \TOP.
We do this both formally and informally.
Informally in the sense that we give an intuitive description of the features that define task-oriented programming.
Formally in the sense that we develop a calculus which formalises these features as language constructs,
and we give them semantics in the style that is common in programming language research.
We separate the task layer and the underlying host language, both syntactically and semantically.
Thus making explicit which properties of \TOP come from the task layer, and which come from functional programming.
Our challenge, therefore, is to model the properties of \TOP into a calculus
and, pave the way for formal treatment of \TOP programs.
We give this formal calculus the name \TOPHAT (TopHat).



\subsection{Contributions}

Our contributions to workflow modelling, functional programming language design, and rapid application development are as follows.


\begin{itemize}
  \item
    We describe the essential concepts of task-oriented programming.

  \item
    We present a formal calculus for modelling declarative workflows, embedded in a simply typed lambda-calculus, called \TOPHAT.
    It is based on the aforementioned essential \TOP concepts.

  \item
    We develop a layered operational semantics for \TOPHAT that is driven by user input.
    The semantics of the task language is clearly separated from the semantics of the underlying host language.

  \item
    Along with this semantics, we present the following semantic observations on tasks:
    the current value, whether a term is stuck, the current user interface, and the accepted inputs.

  \item
    We prove progress and type preservation for \TOPHAT.

  \item
    Using both the essential concepts and the formal calculus, we compare \TOP with related work in areas ranging from business process modelling, to process algebras and reactive programming.

  \item
    We implemented the whole semantic system in the dependently typed programming language Idris \cite{journals/jfp/Brady13}.

  \item
    To create executable applications, we implemented the task layer of \TOPHAT in iTasks.
    This also demonstrates that the former is a subset of the latter.


\end{itemize}


\subsection{Structure}

In \cref{sec:example} we demonstrate the functionality of \TOPHAT by means of an example,
\Cref{sec:intuition} gives an overview of the essential concepts of \TOP.
\Cref{sec:language} introduces the \TOPHAT calculus syntax
and \cref{sec:semantics} the semantics.
Then in \cref{sec:properties} we show that certain properties hold for the calculus.
We take a look at related work in \cref{sec:relatedwork}
and conclude in \cref{sec:conclusions}.
