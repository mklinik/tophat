% !TEX root=../main.tex

\section{Our base language}

Let us start with defining our host language.
It is a simple $\lambda$-calculus,
just containing abstraction, application, variables,
and some basic values like integers and strings.
We will embed our task language in it.
This means we use all the power of the host language,
such as arithmetic operations and (higher order) functions in our task language.
Because all the constructs in our host language are standard,
we will not define typing rules and semantic rules for them explicitly.
For this we refer to standard work like \textcite{Pierce:2002tp} and \textcite{Harper:2016uj}.


\subsection{Syntax}

\begin{grammar}
  Expressions
    & e &::= & e_1\ e_2       & – application \\
    &   &\mid& e_1 \star e_2  & – operator \\
    &   &\mid& x              & – variable \\
    &   &\mid& v              & – value \\
  Values
    & v &::= & \lambda x.\ e  & – abstractions \\
    &   &\mid& I              & – integers \\
    &   &\mid& S              & – strings \\
    &   &\mid& t              & – tasks \\
  Tasks
    & t &::= & `e             & – pure expression \\
\end{grammar}
Expressions are just applications, variables or values.
We add a general notion of operators, represented by $\star$.
Lambda abstractions are values, as are integers and strings.
Most important, \emph{tasks are values}.

For now, the only way to create a task is by \emph{quoting} a pure expression
and thereby lifting it into the task world.
This means we can create very dull tasks containing values like, for example, $`1$,
which is a task having the integer $1$,
and $`\astring{Hello}$,\footnote{
  Mind the extra quote $`$ in front of the expression!
}
a task always containing the string $\astring{Hello}$.
We will use double quotation marks ($""$) to denote strings.
Integers are denoted by their decimal representation.
In the remaining of this text,
we will make freely use of arithmetic operators like $+$, $-$, $\times$,
and appending strings with $\append$.
Only when referring to operators in general we will use $\star$.


\subsection{Typing}

Typing rules for our expressions $e$ are as to be expected.
Only difference with respect to standard work is the addition of a type for tasks.
\begin{grammar}
  Types
    & \tau &::= & \tau_1 \to \tau_2 & – abstractions \\
    &      &\mid& \Int              & – integers \\
    &      &\mid& \String           & – strings \\
    &      &\mid& \Task \tau        & – tasks \\
\end{grammar}
Typing rules are of the form $\boxed{\Gamma \infers e : \tau}$,
which we read as \enquote{in environment $\Gamma$, expression $e$ has type $\tau$}.
The typing rule for pure tasks is given by
\begin{equation*}
  \TPure
\end{equation*}


\subsection{Evaluation}

To express evaluation,
we use the big step semantics of our host language.
Turning an expression $e$ into a value $v$ is denoted by $\boxed{e \downto v}$.
The rules to evaluate expressions $e$ do not defer from standard work.
Also, as tasks are values, they will not be evaluated further by the $\downto$ operation.
