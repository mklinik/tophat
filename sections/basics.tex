% !TEX root=../main.tex

\section{Our base language}

Let us start with defining our base or language.
We start with a simple $\lambda$-calculus,
just containing abstraction, application, variables,
and some basic values like integers and strings.
We will embed our object language in it.
This means we can make use of all the power of our host language in a task.
Because all the constructs in our base language are standard,
we will not define typing rules and semantic rules for them explicitly.
For this we refer to standard work like \textcite{Pierce:2002tp} and \textcite{Harper:2016uj}.

\begin{grammar}
  Expressions
    & e &::= & e_1\ e_2       & – application \\
    &   &\mid& e_1 \cdot e_2  & – operator \\
    &   &\mid& x              & – variable \\
    &   &\mid& v              & – value \\
  Values
    & v &::= & \lambda x.\ e  & – abstractions \\
    &   &\mid& I              & – integers \\
    &   &\mid& S              & – strings \\
    &   &\mid& t              & – tasks \\
  Tasks
    & t &::= & `e             & – pure \\
\end{grammar}

Expressions are just applications, variables or values.
We add a general notion of operators, represented by $\cdot$.
Lambda abstractions are values, as are integers and strings.
Most important, \emph{tasks are values}.

For now, the only way to create a task is by \emph{quoting} a pure expression
and thereby lifting it into the task world.
This means we can create very dull tasks containing values like, for example, $`1$,
which is a task having the integer $1$,
and $`\astring{Hello}$, a task always containing the string $\astring{Hello}$.\footnote{
  Mind the quote $`$ in front of the expression!
}

We will use double quotation marks ($""$) to denote strings.
Integers are denoted by their decimal representation.
In the remaining of this text,
we will make freely use of arithmetic operators like $+$, $-$, $\times$,
and appending strings with $\append$.
Only when referring to operators in general we will use $\cdot$.


\subsection{Typing}

Typing rules for our expressions $e$ are as to be expected.
Only difference with respect to standard work is the addition of a type for tasks.

\begin{grammar}
  Types
    & \tau &::= & \tau_1 \to \tau_2 & – abstractions \\
    &      &\mid& \Int              & – integers \\
    &      &\mid& \String           & – strings \\
    &      &\mid& \Task \tau        & – tasks \\
\end{grammar}

The typing rule for pure tasks becomes:
\begin{equation*}
  \TPure
\end{equation*}


\subsection{Semantics}

To express the semantics,
we use a big step semantics for our host language.
Turning an expression $e$ into a value $v$ is denoted by $e \downto v$.
The rules to evaluate expressions $e$ do not defer from standard work.
Also, as tasks are values, they will not be evaluated further by the $\downto$ operation.
