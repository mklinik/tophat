% !TEX root=../pldi2019.tex

\section{Comparison}

In this section we compare TOP with Hoare's Communicating Sequential Processes (CSP) \cite{books/Hoare85CSP}.
We chose CSP as an example to stand for the numerous process algebras in existence, which, as far as this section is concerned, are reasonably similar.
Notions like communication with the environment and between subsystems, concurrency, and sequential composition can be found in both TOP and in process algebras.
This raises the question of how these systems relate.

We provide comparisons based on three aspects.
First, we compare the scope of the languages, as intended by the authors.
Second, we show how similar features like concurrency or communication work.
Third, we provide some example problems and illustrate how they can be solved in each language.

\subsection{Scope}

The central goal of CSP is to model the patters of behaviour of processes.
These patterns of behaviour manifest themselves in sequences of actions that actors can perform.

The central goal of TOP is to coordinate the collaboration between people who work together to reach a common goal.

CSP has a formal semantics that allows various kinds of correctness proofs, including equality of processes, and adherence to a specification.
This allows applications in program correctness, proofs of deadlock freedom, liveness, or verification of protocols.

TOP focuses less on formal correctness, and more on practical applicability.
It wants to be a language with intuitive semantics that facilitates communication between programmers and domain experts.
TOP programs should hide implementation details from domain experts while containing enough information to allow automatic generation of executable applications, including user interfaces.

\subsection{Features}

In this section we focus on certain features common to TOP and CSP, and study their respective realization.
When we point out differences, we do not argue that the different realizations are incompatible.
As a matter of fact the primitives can certainly be expressed in terms of each other, with more or less effort.
Instead, we point out how the systems emphasize certain points of view by choosing different basic building blocks.

\subsubsection*{Communication with the environment}

In CSP, communication with the environment is represented by prefixing.
The process $(a \to P)$ can engage in action $a$, after which it continues as process $P$.
In general, if $B$ is a set of events and $P(x)$ is an expression that evaluates to a process for each $x \in B$, the process $(x:B \to P(x))$ can engage in any one of the actions in $B$, after which it continues as the process determined by $P(x)$.
Hoare does not specify the language in which $P(x)$ is to be expressed.
He seems to permit any kind of mathematical formula, or any programming language.

Pure prefixing has no direction of sending and receiving.
The interpretation of whether an action stands for input or output is left to the reader.
For example, a vending machine $P = (\textit{coin} \to \textit{choc} \to P)$ is to be interpreted as taking a coin as input and producing chocolate as output.

Sending and receiving of values is modelled by giving additional structure to the names of actions.
For example, an action name could be of the form $\textit{in}.5$.
Let's take as the set $B$ all actions $\textit{in}.n$ for all $n \in \mathbb{N}$.
Let $P(x)$ be an expression that for a given action of this form employs a parser that extracts the number after the dot, the process $(x:B \to P(x))$ reacts as if it receives the action \textit{in}, parameterized with a number.
To send a value to this process, the environment has to provide an appropriate action with a concrete number, for example $\textit{in}.7$.
Sending is therefore the willingness to engage in one concrete action of a given form, while receiving is the willingness to engage in any action of this form.


\subsubsection*{Communication between subsystems}
Concealment vs. shared data and monadic bind

\subsubsection*{Sequential composition}
Success vs. step

\subsubsection*{Concurrency}
Parallel composition

\subsubsection*{Synchronization}
Synchronized actions vs. guarded transitions

\subsubsection*{Nondeterminism}


\subsection{Examples}

\begin{itemize}
\item Mutual exclusion
\item Semaphores
\item Cigarette smokers
\end{itemize}
