% !TEX root=../icfp2019.tex



\section{Intuition}
\label{sec:comparison}

%NOTE: Use \emph{process} as a general term for a process in process algebras and a reactive system from reactive programming?
%NOTE: Main focus should be on \emph{collaboration} between tasks opposed to \emph{communication} between processes.

\subsection{Collaboration}

\subsection{Communication}

The equivalent process in \ESTEREL is the reactive statement $\text{await} a; P$.
Here, action $a$ is interpreted as an input.
To emit an action, one can use $\text{emit} a$.

\paragraph{Communication between subsystems}

% \subsection{Multiprogramming}

\fixme{Add something about reactive programming.}
\fixme{Add combination function for signals that are emitted at the same instant.}
