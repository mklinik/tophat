% !TEX root=../icfp2019.tex

% Context
Software that models how people work is omnipresent in today's society.
Current languages and frameworks are limited in flexibility and lack a good level of abstraction.
Task-oriented programming (TOP) is a programming paradigm that aims to provide the desired level of abstraction while still being expressive enough to describe real world collaboration.
It prescribes a declarative programming style to specify multi-user workflows.
Workflows can be higher-order.
They communicate through typed values on a local and global level.
Such specifications can be turned into interactive applications for different platforms, supporting collaboration during execution.

% Inquiry
TOP has been around for more than a decade, in the form of iTasks, which is tailored for real-world usability.
So far, it has not been given a formalization in the style common in programming language research.

% Approach
In this paper we give a description of the TOP paradigm and then decompose its rich features into elementary language elements, which makes them suitable for formal treatment.
We use the simply typed lambda calculus, extended with pairs and references, as a base language.
On top of this language, we develop TopHat, a calculus for modular interactive workflows.
We describe TopHat by means of a layered semantics.
These layers consist of multiple big step evaluations on expressions, and two labelled transition systems, handling user inputs.

% Knowledge
With TopHat we prepare a way to formally reason about TOP languages and programs.
We show some interesting properties of the language.
This approach allows for comparison with other work in the field.

% Grounding
%  ??? De taal zelf is eigenlijk het artifact

% Importance
TOP has been employed in projects with the Dutch coast guard, tax office, and navy.
Our work matters because formal program verification is important for mission-critical software, especially for systems with concurrency.



% Context: What is the broad context of the work? What is the importance of the general research area?
% Inquiry: What problem or question does the paper address? How has this problem or question been addressed by others (if at all)?
% Approach: What was done that unveiled new knowledge?
% Knowledge: What new facts were uncovered? If the research was not results oriented, what new capabilities are enabled by the work?
% Grounding: What argument, feasibility proof, artifacts, or results and evaluation support this work?
% Importance: Why does this work matter?
