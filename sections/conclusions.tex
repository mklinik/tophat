% !TEX root=../pldi2019.tex

\section{Conclusion}

\label{sec:conclusions}

In this paper we have developed a domain-specific language for declarative interactive workflows.
We have identified, what we believe to be, the essence of Task-Oriented Programming, and provided corresponding operators in our language.
We have compared our language with \CSP to point out differences and similarities.
Finally, we have proven type safety for our language.

\paragraph{Future work}

There are a couple of ways in which we would like to continue this line of work.

The addition of read-only shared editors would allow watching references without being able to change them.
Some programs could be expressed more faithfully, where we for now require that users do not modify certain shares.

One goal of the formalization of a language is the ability to reason about programs.
In this paper we reason about the language itself, but it would be nice to prove properties about individual programs.
We would like to prove whether certain programs are equal, for example to show that the monad laws hold for our step combinator.
This requires a notion of equality, which in the presence of side effects most certainly needs some form of coalgebraic input-output conformance.

We have implemented the reduction semantics of our language in Idris, whose type system could aid in a formalization of those proofs.

Another form of reasoning about programs is static analysis.
In previous work we have developed a cost analysis for tasks that require resources in order to be executed \cite{conf/ifl/KlinikJP17}.
This analysis was developed for a simpler task calculus, and could be brought over to the one developed here.

In other previous work, we have looked at building a generic feedback system for
rule-based problems \cite{UUCS2017013}. A workflow system typically is rule
based, as outlined in this work. It would be interesting to fit the generic
feedback system to \TOPHAT in order to support end-users working in applications
developed in this calculus.


%\begin{itemize*}
  %\item visualisation
  %\item IDE
%\end{itemize*}
